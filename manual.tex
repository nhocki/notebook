\documentclass[10pt,letterpaper]{article}

%---------------------------------------------------------------
\usepackage[utf8]{inputenc}
\usepackage[spanish]{babel}
\usepackage{listings}
\usepackage[usenames,dvipsnames]{color}
\usepackage{amsmath}
\usepackage{verbatim}
\usepackage{hyperref}
%\usepackage{color}
%---------------------------------------------------------------

\setlength{\topmargin}{-1.0in}
\setlength{\textheight}{9.5in} 
\setlength{\evensidemargin}{0.0in}
\setlength{\oddsidemargin}{0.0in}
\setlength{\textwidth}{6.5in} 

\begin{document}

%---------------------------------------------------------------
\title{Resumen de algoritmos para torneos de programación}
\author{Andrés Mejía}
\date{\today}
\maketitle
%---------------------------------------------------------------

%---------------------------------------------------------------
\tableofcontents
%\lstlistoflistings
\lstloadlanguages{C++}
%---------------------------------------------------------------
%---------------------------------------------------------------
\section{Teoría de números}
%---------------------------------------------------------------
\subsection{Big mod}
% Generator: GNU source-highlight, by Lorenzo Bettini, http://www.gnu.org/software/src-highlite

{\ttfamily \raggedright {
\noindent
\mbox{}\textit{\textcolor{Brown}{//retorna\ (b\textasciicircum{}p)mod(m)}} \\
\mbox{}\textit{\textcolor{Brown}{//\ 0\ $<$=\ b,p\ $<$=\ 2147483647}} \\
\mbox{}\textit{\textcolor{Brown}{//\ 1\ $<$=\ m\ $<$=\ 46340}} \\
\mbox{}\textcolor{ForestGreen}{long}\ \textbf{\textcolor{Black}{f}}\textcolor{BrickRed}{(}\textcolor{ForestGreen}{long}\ b\textcolor{BrickRed}{,}\ \textcolor{ForestGreen}{long}\ p\textcolor{BrickRed}{,}\ \textcolor{ForestGreen}{long}\ m\textcolor{BrickRed}{)}\textcolor{Red}{\{} \\
\mbox{}\ \ \textcolor{ForestGreen}{long}\ mask\ \textcolor{BrickRed}{=}\ \textcolor{Purple}{1}\textcolor{BrickRed}{;} \\
\mbox{}\ \ \textcolor{ForestGreen}{long}\ pow2\ \textcolor{BrickRed}{=}\ b\ \textcolor{BrickRed}{\%}\ m\textcolor{BrickRed}{;} \\
\mbox{}\ \ \textcolor{ForestGreen}{long}\ r\ \textcolor{BrickRed}{=}\ \textcolor{Purple}{1}\textcolor{BrickRed}{;} \\
\mbox{} \\
\mbox{}\ \ \textbf{\textcolor{Blue}{while}}\ \textcolor{BrickRed}{(}mask\textcolor{BrickRed}{)}\textcolor{Red}{\{} \\
\mbox{}\ \ \ \ \textbf{\textcolor{Blue}{if}}\ \textcolor{BrickRed}{(}p\ \textcolor{BrickRed}{\&}\ mask\textcolor{BrickRed}{)} \\
\mbox{}\ \ \ \ \ \ r\ \textcolor{BrickRed}{=}\ \textcolor{BrickRed}{(}r\ \textcolor{BrickRed}{*}\ pow2\textcolor{BrickRed}{)}\ \textcolor{BrickRed}{\%}\ m\textcolor{BrickRed}{;} \\
\mbox{}\ \ \ \ pow2\ \textcolor{BrickRed}{=}\ \textcolor{BrickRed}{(}pow2\textcolor{BrickRed}{*}pow2\textcolor{BrickRed}{)}\ \textcolor{BrickRed}{\%}\ m\textcolor{BrickRed}{;} \\
\mbox{}\ \ \ \ mask\ \textcolor{BrickRed}{$<$$<$=}\ \textcolor{Purple}{1}\textcolor{BrickRed}{;} \\
\mbox{}\ \ \textcolor{Red}{\}} \\
\mbox{}\ \ \textbf{\textcolor{Blue}{return}}\ r\textcolor{BrickRed}{;} \\
\mbox{}\textcolor{Red}{\}} \\

} \normalfont\normalsize
%.tex

\subsection{Criba de Eratóstenes}
\small
\textbf{Field-testing:} 
\begin{itemize}
\item \emph{SPOJ} -  2912 - Super Primes
\item \emph{Live Archive} - 3639 - Prime Path
\end{itemize}

\normalsize
Marca los números primos en un arreglo. Algunos tiempos de ejecución:
\begin{center}
\begin{tabular}{c c}
\hline\hline
SIZE & Tiempo (s) \\ [0.5ex]
\hline
100000 & 0.003 \\
1000000 & 0.060 \\
10000000 & 0.620 \\
100000000 & 7.650 \\ [1ex]
\hline
\end{tabular}
\end{center}

% Generator: GNU source-highlight, by Lorenzo Bettini, http://www.gnu.org/software/src-highlite

{\ttfamily \raggedright {
\noindent
\mbox{}\textbf{\textcolor{RoyalBlue}{\#include}}\ \texttt{\textcolor{Red}{$<$iostream$>$}} \\
\mbox{} \\
\mbox{}\textbf{\textcolor{Blue}{const}}\ \textcolor{ForestGreen}{int}\ SIZE\ \textcolor{BrickRed}{=}\ \textcolor{Purple}{1000000}\textcolor{BrickRed}{;} \\
\mbox{} \\
\mbox{}\textit{\textcolor{Brown}{//criba[i]\ =\ false\ si\ i\ es\ primo}} \\
\mbox{}\textcolor{ForestGreen}{bool}\ criba\textcolor{BrickRed}{[}SIZE\textcolor{BrickRed}{+}\textcolor{Purple}{1}\textcolor{BrickRed}{];} \\
\mbox{} \\
\mbox{}\textcolor{ForestGreen}{void}\ \textbf{\textcolor{Black}{buildCriba}}\textcolor{BrickRed}{()}\textcolor{Red}{\{} \\
\mbox{}\ \ \textbf{\textcolor{Black}{memset}}\textcolor{BrickRed}{(}criba\textcolor{BrickRed}{,}\ \textbf{\textcolor{Blue}{false}}\textcolor{BrickRed}{,}\ \textbf{\textcolor{Blue}{sizeof}}\textcolor{BrickRed}{(}criba\textcolor{BrickRed}{));} \\
\mbox{} \\
\mbox{}\ \ criba\textcolor{BrickRed}{[}\textcolor{Purple}{0}\textcolor{BrickRed}{]}\ \textcolor{BrickRed}{=}\ criba\textcolor{BrickRed}{[}\textcolor{Purple}{1}\textcolor{BrickRed}{]}\ \textcolor{BrickRed}{=}\ \textbf{\textcolor{Blue}{true}}\textcolor{BrickRed}{;} \\
\mbox{}\ \ \textbf{\textcolor{Blue}{for}}\ \textcolor{BrickRed}{(}\textcolor{ForestGreen}{int}\ i\textcolor{BrickRed}{=}\textcolor{Purple}{4}\textcolor{BrickRed}{;}\ i\textcolor{BrickRed}{$<$=}SIZE\textcolor{BrickRed}{;}\ i\ \textcolor{BrickRed}{+=}\ \textcolor{Purple}{2}\textcolor{BrickRed}{)}\textcolor{Red}{\{} \\
\mbox{}\ \ \ \ criba\textcolor{BrickRed}{[}i\textcolor{BrickRed}{]}\ \textcolor{BrickRed}{=}\ \textbf{\textcolor{Blue}{true}}\textcolor{BrickRed}{;} \\
\mbox{}\ \ \textcolor{Red}{\}} \\
\mbox{}\ \ \textbf{\textcolor{Blue}{for}}\ \textcolor{BrickRed}{(}\textcolor{ForestGreen}{int}\ i\textcolor{BrickRed}{=}\textcolor{Purple}{3}\textcolor{BrickRed}{;}\ i\textcolor{BrickRed}{*}i\textcolor{BrickRed}{$<$=}SIZE\textcolor{BrickRed}{;}\ i\ \textcolor{BrickRed}{+=}\ \textcolor{Purple}{2}\textcolor{BrickRed}{)}\textcolor{Red}{\{} \\
\mbox{}\ \ \ \ \textbf{\textcolor{Blue}{if}}\ \textcolor{BrickRed}{(!}criba\textcolor{BrickRed}{[}i\textcolor{BrickRed}{])}\textcolor{Red}{\{} \\
\mbox{}\ \ \ \ \ \ \textbf{\textcolor{Blue}{for}}\ \textcolor{BrickRed}{(}\textcolor{ForestGreen}{int}\ j\textcolor{BrickRed}{=}i\textcolor{BrickRed}{*}i\textcolor{BrickRed}{;}\ j\textcolor{BrickRed}{$<$=}SIZE\textcolor{BrickRed}{;}\ j\ \textcolor{BrickRed}{+=}\ i\textcolor{BrickRed}{)}\textcolor{Red}{\{} \\
\mbox{}\ \ \ \ \ \ \ \ criba\textcolor{BrickRed}{[}j\textcolor{BrickRed}{]}\ \textcolor{BrickRed}{=}\ \textbf{\textcolor{Blue}{true}}\textcolor{BrickRed}{;} \\
\mbox{}\ \ \ \ \ \ \textcolor{Red}{\}} \\
\mbox{}\ \ \ \ \textcolor{Red}{\}} \\
\mbox{}\ \ \textcolor{Red}{\}} \\
\mbox{}\textcolor{Red}{\}} \\

} \normalfont\normalsize
%.tex

\subsection{Divisores de un número}
Este algoritmo imprime todos los divisores de un número (en desorden) en O($\sqrt{n}$).
Hasta 4294967295 (máximo \textit{unsigned long}) responde instantaneamente. Se puede
forzar un poco más usando \textit{unsigned long long} pero más allá de $10^{12}$ empieza a
responder muy lento.

\bigskip

% Generator: GNU source-highlight, by Lorenzo Bettini, http://www.gnu.org/software/src-highlite

{\ttfamily \raggedright {
\noindent
\mbox{}\textbf{\textcolor{Blue}{for}}\ \textcolor{BrickRed}{(}\textcolor{ForestGreen}{int}\ i\textcolor{BrickRed}{=}\textcolor{Purple}{1}\textcolor{BrickRed}{;}\ i\textcolor{BrickRed}{*}i\textcolor{BrickRed}{$<$=}n\textcolor{BrickRed}{;}\ i\textcolor{BrickRed}{++)}\ \textcolor{Red}{\{} \\
\mbox{}\ \ \textbf{\textcolor{Blue}{if}}\ \textcolor{BrickRed}{(}n\textcolor{BrickRed}{\%}i\ \textcolor{BrickRed}{==}\ \textcolor{Purple}{0}\textcolor{BrickRed}{)}\ \textcolor{Red}{\{} \\
\mbox{}\ \ \ \ cout\ \textcolor{BrickRed}{$<$$<$}\ i\ \textcolor{BrickRed}{$<$$<$}\ endl\textcolor{BrickRed}{;} \\
\mbox{}\ \ \ \ \textbf{\textcolor{Blue}{if}}\ \textcolor{BrickRed}{(}i\textcolor{BrickRed}{*}i\textcolor{BrickRed}{$<$}n\textcolor{BrickRed}{)}\ cout\ \textcolor{BrickRed}{$<$$<$}\ \textcolor{BrickRed}{(}n\textcolor{BrickRed}{/}i\textcolor{BrickRed}{)}\ \textcolor{BrickRed}{$<$$<$}\ endl\textcolor{BrickRed}{;} \\
\mbox{}\ \ \textcolor{Red}{\}} \\
\mbox{}\textcolor{Red}{\}}\  \\

} \normalfont\normalsize
%.tex

\section{Combinatoria}
\subsection{Cuadro resumen}
Fórmulas para combinaciones y permutaciones:
\begin{center}
\renewcommand{\arraystretch}{2} %Multiplica la altura de cada fila de la tabla por 2
%Si quiero aumentar el tamaño de una fila en particular insertar \rule{0cm}{1cm} en esa fila.
\begin{tabular}{| c | c | c |}
\hline
\textit{Tipo} & \textit{¿Se permite la repetición?} & \textit{Fórmula} \\ [1.5ex]
\hline\hline

$r$-permutaciones & No & $ \displaystyle\frac{n!}{(n-r)!} $ \\ [1.5ex]
\hline
$r$-combinaciones & No & $ \displaystyle\frac{n!}{r!(n-r)!} $ \\  [1.5ex]
\hline
$r$-permutaciones & Sí & $ \displaystyle n^{r} $ \\
\hline
$r$-combinaciones & Sí & $ \displaystyle\frac{(n+r-1)!}{r!(n-1)!} $ \\ [1.5ex]
\hline
\end{tabular}
\renewcommand{\arraystretch}{1}
\end{center}
Tomado de \textit{Matemática discreta y sus aplicaciones}, Kenneth Rosen, 5${}^{\hbox{ta}}$ edición, McGraw-Hill, página 315.

\subsection{Combinaciones, coeficientes binomiales, triángulo de Pascal}
\emph{Complejidad:} $ O(n^2) $ \\
$$ {n \choose k} = \left\{
\begin{array}{c l}
 1 & k = 0\\
 1 & n = k\\
 \displaystyle {n - 1 \choose k - 1} + {n - 1 \choose k} & \mbox{en otro caso}\\
\end{array}
\right.
$$

% Generator: GNU source-highlight, by Lorenzo Bettini, http://www.gnu.org/software/src-highlite

{\ttfamily \raggedright {
\noindent
\mbox{}\textbf{\textcolor{Blue}{const}}\ \textcolor{ForestGreen}{int}\ N\ \textcolor{BrickRed}{=}\ \textcolor{Purple}{30}\textcolor{BrickRed}{;} \\
\mbox{}\textcolor{ForestGreen}{long}\ \textcolor{ForestGreen}{long}\ choose\textcolor{BrickRed}{[}N\textcolor{BrickRed}{+}\textcolor{Purple}{1}\textcolor{BrickRed}{][}N\textcolor{BrickRed}{+}\textcolor{Purple}{1}\textcolor{BrickRed}{];} \\
\mbox{}\ \ \textit{\textcolor{Brown}{/*\ Binomial\ coefficients\ */}} \\
\mbox{}\ \ \textbf{\textcolor{Blue}{for}}\ \textcolor{BrickRed}{(}\textcolor{ForestGreen}{int}\ i\textcolor{BrickRed}{=}\textcolor{Purple}{0}\textcolor{BrickRed}{;}\ i\textcolor{BrickRed}{$<$=}N\textcolor{BrickRed}{;}\ \textcolor{BrickRed}{++}i\textcolor{BrickRed}{)}\ choose\textcolor{BrickRed}{[}i\textcolor{BrickRed}{][}\textcolor{Purple}{0}\textcolor{BrickRed}{]}\ \textcolor{BrickRed}{=}\ choose\textcolor{BrickRed}{[}i\textcolor{BrickRed}{][}i\textcolor{BrickRed}{]}\ \textcolor{BrickRed}{=}\ \textcolor{Purple}{1}\textcolor{BrickRed}{;} \\
\mbox{}\ \ \textbf{\textcolor{Blue}{for}}\ \textcolor{BrickRed}{(}\textcolor{ForestGreen}{int}\ i\textcolor{BrickRed}{=}\textcolor{Purple}{1}\textcolor{BrickRed}{;}\ i\textcolor{BrickRed}{$<$=}N\textcolor{BrickRed}{;}\ \textcolor{BrickRed}{++}i\textcolor{BrickRed}{)} \\
\mbox{}\ \ \ \ \textbf{\textcolor{Blue}{for}}\ \textcolor{BrickRed}{(}\textcolor{ForestGreen}{int}\ j\textcolor{BrickRed}{=}\textcolor{Purple}{1}\textcolor{BrickRed}{;}\ j\textcolor{BrickRed}{$<$}i\textcolor{BrickRed}{;}\ \textcolor{BrickRed}{++}j\textcolor{BrickRed}{)} \\
\mbox{}\ \ \ \ \ \ choose\textcolor{BrickRed}{[}i\textcolor{BrickRed}{][}j\textcolor{BrickRed}{]}\ \textcolor{BrickRed}{=}\ choose\textcolor{BrickRed}{[}i\textcolor{BrickRed}{-}\textcolor{Purple}{1}\textcolor{BrickRed}{][}j\textcolor{BrickRed}{-}\textcolor{Purple}{1}\textcolor{BrickRed}{]}\ \textcolor{BrickRed}{+}\ choose\textcolor{BrickRed}{[}i\textcolor{BrickRed}{-}\textcolor{Purple}{1}\textcolor{BrickRed}{][}j\textcolor{BrickRed}{];} \\

} \normalfont\normalsize
%.tex

\bigskip 
\textbf{Nota:} $ \displaystyle {n \choose k }  $ está indefinido en el código anterior si $ n > k$. ¡La tabla puede estar llena con cualquier basura del compilador!

\subsection{Permutaciones con elementos indistinguibles}
El número de permutaciones \underline{diferentes} de $n$ objetos, donde hay $n_{1}$ objetos indistinguibles de tipo 1,
$n_{2}$ objetos indistinguibles de tipo 2, ..., y $n_{k}$ objetos indistinguibles de tipo $k$, es
$$
\frac{n!}{n_{1}!n_{2}! \cdots n_{k}!}
$$
\textbf{Ejemplo:} Con las letras de la palabra \texttt{PROGRAMAR} se pueden formar $ \displaystyle \frac{9!}{2! \cdot 3!} = 
30240 $ permutaciones \underline{diferentes}.
\subsection{Desordenes, desarreglos o permutaciones completas}

Un desarreglo es una permutación donde ningún elemento $i$ está en la
posición $i$-ésima. Por ejemplo, \textit{4213} es un desarreglo de 4 elementos pero
\textit{3241} no lo es porque el 2 aparece en la posición 2.

Sea $D_{n}$ el número de desarreglos de $n$ elementos, entonces:
$$ {D_{n}} = \left\{
\begin{array}{c l}
 1 & n = 0\\
 0 & n = 1\\
 (n-1)(D_{n-1} + D_{n-2}) & n \geq 2\\
\end{array}
\right.
$$
Usando el principio de inclusión-exclusión, también se puede encontrar la fórmula
$$
D_{n} = n!\left [ 1 - \frac{1}{1!} + \frac{1}{2!} - \frac{1}{3!} + \cdots + (-1)^{n}\frac{1}{n!} \right ]
= n! \sum_{i=0}^{n} \frac{(-1)^{i}}{i!}
$$

\section{Grafos}
\subsection{Algoritmo de Dijkstra}
El peso de todas las aristas debe ser no negativo.
\\
% Generator: GNU source-highlight, by Lorenzo Bettini, http://www.gnu.org/software/src-highlite

{\ttfamily \raggedright {
\noindent
\mbox{}\textbf{\textcolor{RoyalBlue}{\#include}}\ \texttt{\textcolor{Red}{$<$iostream$>$}} \\
\mbox{}\textbf{\textcolor{RoyalBlue}{\#include}}\ \texttt{\textcolor{Red}{$<$algorithm$>$}} \\
\mbox{}\textbf{\textcolor{RoyalBlue}{\#include}}\ \texttt{\textcolor{Red}{$<$queue$>$}} \\
\mbox{} \\
\mbox{}\textbf{\textcolor{Blue}{using}}\ \textbf{\textcolor{Blue}{namespace}}\ std\textcolor{BrickRed}{;} \\
\mbox{} \\
\mbox{}\textbf{\textcolor{Blue}{struct}}\ edge\textcolor{Red}{\{} \\
\mbox{}\ \ \textcolor{ForestGreen}{int}\ to\textcolor{BrickRed}{,}\ weight\textcolor{BrickRed}{;} \\
\mbox{}\ \ \textbf{\textcolor{Black}{edge}}\textcolor{BrickRed}{()}\ \textcolor{Red}{\{\}} \\
\mbox{}\ \ \textbf{\textcolor{Black}{edge}}\textcolor{BrickRed}{(}\textcolor{ForestGreen}{int}\ t\textcolor{BrickRed}{,}\ \textcolor{ForestGreen}{int}\ w\textcolor{BrickRed}{)}\ \textcolor{BrickRed}{:}\ \textbf{\textcolor{Black}{to}}\textcolor{BrickRed}{(}t\textcolor{BrickRed}{),}\ \textbf{\textcolor{Black}{weight}}\textcolor{BrickRed}{(}w\textcolor{BrickRed}{)}\ \textcolor{Red}{\{\}} \\
\mbox{}\ \ \textcolor{ForestGreen}{bool}\ \textbf{\textcolor{Blue}{operator}}\ \textcolor{BrickRed}{$<$}\ \textcolor{BrickRed}{(}\textbf{\textcolor{Blue}{const}}\ edge\ \textcolor{BrickRed}{\&}that\textcolor{BrickRed}{)}\ \textbf{\textcolor{Blue}{const}}\ \textcolor{Red}{\{} \\
\mbox{}\ \ \ \ \textbf{\textcolor{Blue}{return}}\ weight\ \textcolor{BrickRed}{$>$}\ that\textcolor{BrickRed}{.}weight\textcolor{BrickRed}{;} \\
\mbox{}\ \ \textcolor{Red}{\}} \\
\mbox{}\textcolor{Red}{\}}\textcolor{BrickRed}{;} \\
\mbox{} \\
\mbox{}\textcolor{ForestGreen}{int}\ \textbf{\textcolor{Black}{main}}\textcolor{BrickRed}{()}\textcolor{Red}{\{} \\
\mbox{}\ \ \textcolor{ForestGreen}{int}\ N\textcolor{BrickRed}{,}\ C\textcolor{BrickRed}{=}\textcolor{Purple}{0}\textcolor{BrickRed}{;} \\
\mbox{}\ \ \textbf{\textcolor{Black}{scanf}}\textcolor{BrickRed}{(}\texttt{\textcolor{Red}{"{}\%d"{}}}\textcolor{BrickRed}{,}\ \textcolor{BrickRed}{\&}N\textcolor{BrickRed}{);} \\
\mbox{}\ \ \textbf{\textcolor{Blue}{while}}\ \textcolor{BrickRed}{(}N\textcolor{BrickRed}{-\/-}\ \textcolor{BrickRed}{\&\&}\ \textcolor{BrickRed}{++}C\textcolor{BrickRed}{)}\textcolor{Red}{\{} \\
\mbox{}\ \ \ \ \textcolor{ForestGreen}{int}\ n\textcolor{BrickRed}{,}\ m\textcolor{BrickRed}{,}\ s\textcolor{BrickRed}{,}\ t\textcolor{BrickRed}{;} \\
\mbox{}\ \ \ \ \textbf{\textcolor{Black}{scanf}}\textcolor{BrickRed}{(}\texttt{\textcolor{Red}{"{}\%d\ \%d\ \%d\ \%d"{}}}\textcolor{BrickRed}{,}\ \textcolor{BrickRed}{\&}n\textcolor{BrickRed}{,}\ \textcolor{BrickRed}{\&}m\textcolor{BrickRed}{,}\ \textcolor{BrickRed}{\&}s\textcolor{BrickRed}{,}\ \textcolor{BrickRed}{\&}t\textcolor{BrickRed}{);} \\
\mbox{}\ \ \ \ vector\textcolor{BrickRed}{$<$}edge\textcolor{BrickRed}{$>$}\ g\textcolor{BrickRed}{[}n\textcolor{BrickRed}{];} \\
\mbox{}\ \ \ \ \textbf{\textcolor{Blue}{while}}\ \textcolor{BrickRed}{(}m\textcolor{BrickRed}{-\/-)}\textcolor{Red}{\{} \\
\mbox{}\ \ \ \ \ \ \textcolor{ForestGreen}{int}\ u\textcolor{BrickRed}{,}\ v\textcolor{BrickRed}{,}\ w\textcolor{BrickRed}{;} \\
\mbox{}\ \ \ \ \ \ \textbf{\textcolor{Black}{scanf}}\textcolor{BrickRed}{(}\texttt{\textcolor{Red}{"{}\%d\ \%d\ \%d"{}}}\textcolor{BrickRed}{,}\ \textcolor{BrickRed}{\&}u\textcolor{BrickRed}{,}\ \textcolor{BrickRed}{\&}v\textcolor{BrickRed}{,}\ \textcolor{BrickRed}{\&}w\textcolor{BrickRed}{);} \\
\mbox{}\ \ \ \ \ \ g\textcolor{BrickRed}{[}u\textcolor{BrickRed}{].}\textbf{\textcolor{Black}{push$\_$back}}\textcolor{BrickRed}{(}\textbf{\textcolor{Black}{edge}}\textcolor{BrickRed}{(}v\textcolor{BrickRed}{,}\ w\textcolor{BrickRed}{));} \\
\mbox{}\ \ \ \ \ \ g\textcolor{BrickRed}{[}v\textcolor{BrickRed}{].}\textbf{\textcolor{Black}{push$\_$back}}\textcolor{BrickRed}{(}\textbf{\textcolor{Black}{edge}}\textcolor{BrickRed}{(}u\textcolor{BrickRed}{,}\ w\textcolor{BrickRed}{));} \\
\mbox{}\ \ \ \ \textcolor{Red}{\}} \\
\mbox{} \\
\mbox{}\ \ \ \ \textcolor{ForestGreen}{int}\ d\textcolor{BrickRed}{[}n\textcolor{BrickRed}{];} \\
\mbox{}\ \ \ \ \textbf{\textcolor{Blue}{for}}\ \textcolor{BrickRed}{(}\textcolor{ForestGreen}{int}\ i\textcolor{BrickRed}{=}\textcolor{Purple}{0}\textcolor{BrickRed}{;}\ i\textcolor{BrickRed}{$<$}n\textcolor{BrickRed}{;}\ \textcolor{BrickRed}{++}i\textcolor{BrickRed}{)}\ d\textcolor{BrickRed}{[}i\textcolor{BrickRed}{]}\ \textcolor{BrickRed}{=}\ INT$\_$MAX\textcolor{BrickRed}{;} \\
\mbox{}\ \ \ \ d\textcolor{BrickRed}{[}s\textcolor{BrickRed}{]}\ \textcolor{BrickRed}{=}\ \textcolor{Purple}{0}\textcolor{BrickRed}{;} \\
\mbox{}\ \ \ \ priority$\_$queue\textcolor{BrickRed}{$<$}edge\textcolor{BrickRed}{$>$}\ q\textcolor{BrickRed}{;} \\
\mbox{}\ \ \ \ q\textcolor{BrickRed}{.}\textbf{\textcolor{Black}{push}}\textcolor{BrickRed}{(}\textbf{\textcolor{Black}{edge}}\textcolor{BrickRed}{(}s\textcolor{BrickRed}{,}\ \textcolor{Purple}{0}\textcolor{BrickRed}{));} \\
\mbox{}\ \ \ \ \textbf{\textcolor{Blue}{while}}\ \textcolor{BrickRed}{(}q\textcolor{BrickRed}{.}\textbf{\textcolor{Black}{empty}}\textcolor{BrickRed}{()}\ \textcolor{BrickRed}{==}\ \textbf{\textcolor{Blue}{false}}\textcolor{BrickRed}{)}\textcolor{Red}{\{} \\
\mbox{}\ \ \ \ \ \ \textcolor{ForestGreen}{int}\ node\ \textcolor{BrickRed}{=}\ q\textcolor{BrickRed}{.}\textbf{\textcolor{Black}{top}}\textcolor{BrickRed}{().}to\textcolor{BrickRed}{;} \\
\mbox{}\ \ \ \ \ \ \textcolor{ForestGreen}{int}\ dist\ \textcolor{BrickRed}{=}\ q\textcolor{BrickRed}{.}\textbf{\textcolor{Black}{top}}\textcolor{BrickRed}{().}weight\textcolor{BrickRed}{;} \\
\mbox{}\ \ \ \ \ \ q\textcolor{BrickRed}{.}\textbf{\textcolor{Black}{pop}}\textcolor{BrickRed}{();} \\
\mbox{} \\
\mbox{}\ \ \ \ \ \ \textbf{\textcolor{Blue}{if}}\ \textcolor{BrickRed}{(}dist\ \textcolor{BrickRed}{$>$}\ d\textcolor{BrickRed}{[}node\textcolor{BrickRed}{])}\ \textbf{\textcolor{Blue}{continue}}\textcolor{BrickRed}{;} \\
\mbox{}\ \ \ \ \ \ \textbf{\textcolor{Blue}{if}}\ \textcolor{BrickRed}{(}node\ \textcolor{BrickRed}{==}\ t\textcolor{BrickRed}{)}\ \textbf{\textcolor{Blue}{break}}\textcolor{BrickRed}{;} \\
\mbox{} \\
\mbox{}\ \ \ \ \ \ \textbf{\textcolor{Blue}{for}}\ \textcolor{BrickRed}{(}\textcolor{ForestGreen}{int}\ i\textcolor{BrickRed}{=}\textcolor{Purple}{0}\textcolor{BrickRed}{;}\ i\textcolor{BrickRed}{$<$}g\textcolor{BrickRed}{[}node\textcolor{BrickRed}{].}\textbf{\textcolor{Black}{size}}\textcolor{BrickRed}{();}\ \textcolor{BrickRed}{++}i\textcolor{BrickRed}{)}\textcolor{Red}{\{} \\
\mbox{}\ \ \ \ \ \ \ \ \textcolor{ForestGreen}{int}\ to\ \textcolor{BrickRed}{=}\ g\textcolor{BrickRed}{[}node\textcolor{BrickRed}{][}i\textcolor{BrickRed}{].}to\textcolor{BrickRed}{;} \\
\mbox{}\ \ \ \ \ \ \ \ \textcolor{ForestGreen}{int}\ w$\_$extra\ \textcolor{BrickRed}{=}\ g\textcolor{BrickRed}{[}node\textcolor{BrickRed}{][}i\textcolor{BrickRed}{].}weight\textcolor{BrickRed}{;} \\
\mbox{} \\
\mbox{}\ \ \ \ \ \ \ \ \textbf{\textcolor{Blue}{if}}\ \textcolor{BrickRed}{(}dist\ \textcolor{BrickRed}{+}\ w$\_$extra\ \textcolor{BrickRed}{$<$}\ d\textcolor{BrickRed}{[}to\textcolor{BrickRed}{])}\textcolor{Red}{\{} \\
\mbox{}\ \ \ \ \ \ \ \ \ \ d\textcolor{BrickRed}{[}to\textcolor{BrickRed}{]}\ \textcolor{BrickRed}{=}\ dist\ \textcolor{BrickRed}{+}\ w$\_$extra\textcolor{BrickRed}{;} \\
\mbox{}\ \ \ \ \ \ \ \ \ \ q\textcolor{BrickRed}{.}\textbf{\textcolor{Black}{push}}\textcolor{BrickRed}{(}\textbf{\textcolor{Black}{edge}}\textcolor{BrickRed}{(}to\textcolor{BrickRed}{,}\ d\textcolor{BrickRed}{[}to\textcolor{BrickRed}{]));} \\
\mbox{}\ \ \ \ \ \ \ \ \textcolor{Red}{\}} \\
\mbox{}\ \ \ \ \ \ \textcolor{Red}{\}} \\
\mbox{}\ \ \ \ \textcolor{Red}{\}} \\
\mbox{}\ \ \ \ \textbf{\textcolor{Black}{printf}}\textcolor{BrickRed}{(}\texttt{\textcolor{Red}{"{}Case\ \#\%d:\ "{}}}\textcolor{BrickRed}{,}\ C\textcolor{BrickRed}{);} \\
\mbox{}\ \ \ \ \textbf{\textcolor{Blue}{if}}\ \textcolor{BrickRed}{(}d\textcolor{BrickRed}{[}t\textcolor{BrickRed}{]}\ \textcolor{BrickRed}{$<$}\ INT$\_$MAX\textcolor{BrickRed}{)}\ \textbf{\textcolor{Black}{printf}}\textcolor{BrickRed}{(}\texttt{\textcolor{Red}{"{}\%d}}\texttt{\textcolor{CarnationPink}{\textbackslash{}n}}\texttt{\textcolor{Red}{"{}}}\textcolor{BrickRed}{,}\ d\textcolor{BrickRed}{[}t\textcolor{BrickRed}{]);} \\
\mbox{}\ \ \ \ \textbf{\textcolor{Blue}{else}}\ \textbf{\textcolor{Black}{printf}}\textcolor{BrickRed}{(}\texttt{\textcolor{Red}{"{}unreachable}}\texttt{\textcolor{CarnationPink}{\textbackslash{}n}}\texttt{\textcolor{Red}{"{}}}\textcolor{BrickRed}{);} \\
\mbox{}\ \ \textcolor{Red}{\}} \\
\mbox{}\ \ \textbf{\textcolor{Blue}{return}}\ \textcolor{Purple}{0}\textcolor{BrickRed}{;} \\
\mbox{}\textcolor{Red}{\}} \\

} \normalfont\normalsize
%.tex

\subsection{Minimum spanning tree: Algoritmo de Prim}

% Generator: GNU source-highlight, by Lorenzo Bettini, http://www.gnu.org/software/src-highlite

{\ttfamily \raggedright {
\noindent
\mbox{}\textbf{\textcolor{RoyalBlue}{\#include}}\ \texttt{\textcolor{Red}{$<$stdio.h$>$}} \\
\mbox{}\textbf{\textcolor{RoyalBlue}{\#include}}\ \texttt{\textcolor{Red}{$<$string$>$}} \\
\mbox{}\textbf{\textcolor{RoyalBlue}{\#include}}\ \texttt{\textcolor{Red}{$<$set$>$}} \\
\mbox{}\textbf{\textcolor{RoyalBlue}{\#include}}\ \texttt{\textcolor{Red}{$<$vector$>$}} \\
\mbox{}\textbf{\textcolor{RoyalBlue}{\#include}}\ \texttt{\textcolor{Red}{$<$queue$>$}} \\
\mbox{}\textbf{\textcolor{RoyalBlue}{\#include}}\ \texttt{\textcolor{Red}{$<$iostream$>$}} \\
\mbox{}\textbf{\textcolor{RoyalBlue}{\#include}}\ \texttt{\textcolor{Red}{$<$map$>$}} \\
\mbox{} \\
\mbox{}\textbf{\textcolor{Blue}{using}}\ \textbf{\textcolor{Blue}{namespace}}\ std\textcolor{BrickRed}{;} \\
\mbox{} \\
\mbox{}\textbf{\textcolor{Blue}{typedef}}\ string\ node\textcolor{BrickRed}{;} \\
\mbox{}\textbf{\textcolor{Blue}{typedef}}\ pair\textcolor{BrickRed}{$<$}\textcolor{ForestGreen}{double}\textcolor{BrickRed}{,}\ node\textcolor{BrickRed}{$>$}\ edge\textcolor{BrickRed}{;} \\
\mbox{}\textbf{\textcolor{Blue}{typedef}}\ map\textcolor{BrickRed}{$<$}node\textcolor{BrickRed}{,}\ vector\textcolor{BrickRed}{$<$}edge\textcolor{BrickRed}{$>$}\ \textcolor{BrickRed}{$>$}\ graph\textcolor{BrickRed}{;} \\
\mbox{} \\
\mbox{} \\
\mbox{}\textcolor{ForestGreen}{int}\ \textbf{\textcolor{Black}{main}}\textcolor{BrickRed}{()}\textcolor{Red}{\{} \\
\mbox{}\ \ \textcolor{ForestGreen}{double}\ length\textcolor{BrickRed}{;} \\
\mbox{}\ \ \textbf{\textcolor{Blue}{while}}\ \textcolor{BrickRed}{(}cin\ \textcolor{BrickRed}{$>$$>$}\ length\textcolor{BrickRed}{)}\textcolor{Red}{\{} \\
\mbox{}\ \ \ \ \textcolor{ForestGreen}{int}\ cities\textcolor{BrickRed}{;} \\
\mbox{}\ \ \ \ cin\ \textcolor{BrickRed}{$>$$>$}\ cities\textcolor{BrickRed}{;} \\
\mbox{}\ \ \ \ graph\ g\textcolor{BrickRed}{;} \\
\mbox{}\ \ \ \ \textbf{\textcolor{Blue}{for}}\ \textcolor{BrickRed}{(}\textcolor{ForestGreen}{int}\ i\textcolor{BrickRed}{=}\textcolor{Purple}{0}\textcolor{BrickRed}{;}\ i\textcolor{BrickRed}{$<$}cities\textcolor{BrickRed}{;}\ \textcolor{BrickRed}{++}i\textcolor{BrickRed}{)}\textcolor{Red}{\{} \\
\mbox{}\ \ \ \ \ \ string\ s\textcolor{BrickRed}{;} \\
\mbox{}\ \ \ \ \ \ cin\ \textcolor{BrickRed}{$>$$>$}\ s\textcolor{BrickRed}{;} \\
\mbox{}\ \ \ \ \ \ g\textcolor{BrickRed}{[}s\textcolor{BrickRed}{]}\ \textcolor{BrickRed}{=}\ vector\textcolor{BrickRed}{$<$}edge\textcolor{BrickRed}{$>$();} \\
\mbox{}\ \ \ \ \textcolor{Red}{\}} \\
\mbox{}\ \ \ \ \textcolor{ForestGreen}{int}\ edges\textcolor{BrickRed}{;} \\
\mbox{}\ \ \ \ cin\ \textcolor{BrickRed}{$>$$>$}\ edges\textcolor{BrickRed}{;} \\
\mbox{}\ \ \ \ \textbf{\textcolor{Blue}{for}}\ \textcolor{BrickRed}{(}\textcolor{ForestGreen}{int}\ i\textcolor{BrickRed}{=}\textcolor{Purple}{0}\textcolor{BrickRed}{;}\ i\textcolor{BrickRed}{$<$}edges\textcolor{BrickRed}{;}\ \textcolor{BrickRed}{++}i\textcolor{BrickRed}{)}\textcolor{Red}{\{} \\
\mbox{}\ \ \ \ \ \ string\ u\textcolor{BrickRed}{,}\ v\textcolor{BrickRed}{;} \\
\mbox{}\ \ \ \ \ \ \textcolor{ForestGreen}{double}\ w\textcolor{BrickRed}{;} \\
\mbox{}\ \ \ \ \ \ cin\ \textcolor{BrickRed}{$>$$>$}\ u\ \textcolor{BrickRed}{$>$$>$}\ v\ \textcolor{BrickRed}{$>$$>$}\ w\textcolor{BrickRed}{;} \\
\mbox{}\ \ \ \ \ \ g\textcolor{BrickRed}{[}u\textcolor{BrickRed}{].}\textbf{\textcolor{Black}{push$\_$back}}\textcolor{BrickRed}{(}\textbf{\textcolor{Black}{edge}}\textcolor{BrickRed}{(}w\textcolor{BrickRed}{,}\ v\textcolor{BrickRed}{));} \\
\mbox{}\ \ \ \ \ \ g\textcolor{BrickRed}{[}v\textcolor{BrickRed}{].}\textbf{\textcolor{Black}{push$\_$back}}\textcolor{BrickRed}{(}\textbf{\textcolor{Black}{edge}}\textcolor{BrickRed}{(}w\textcolor{BrickRed}{,}\ u\textcolor{BrickRed}{));} \\
\mbox{}\ \ \ \ \textcolor{Red}{\}} \\
\mbox{} \\
\mbox{}\ \ \ \ \textcolor{ForestGreen}{double}\ total\ \textcolor{BrickRed}{=}\ \textcolor{Purple}{0.0}\textcolor{BrickRed}{;} \\
\mbox{}\ \ \ \ priority$\_$queue\textcolor{BrickRed}{$<$}edge\textcolor{BrickRed}{,}\ vector\textcolor{BrickRed}{$<$}edge\textcolor{BrickRed}{$>$,}\ greater\textcolor{BrickRed}{$<$}edge\textcolor{BrickRed}{$>$}\ \textcolor{BrickRed}{$>$}\ q\textcolor{BrickRed}{;} \\
\mbox{}\ \ \ \ q\textcolor{BrickRed}{.}\textbf{\textcolor{Black}{push}}\textcolor{BrickRed}{(}\textbf{\textcolor{Black}{edge}}\textcolor{BrickRed}{(}\textcolor{Purple}{0.0}\textcolor{BrickRed}{,}\ g\textcolor{BrickRed}{.}\textbf{\textcolor{Black}{begin}}\textcolor{BrickRed}{()-$>$}first\textcolor{BrickRed}{));} \\
\mbox{}\ \ \ \ set\textcolor{BrickRed}{$<$}node\textcolor{BrickRed}{$>$}\ visited\textcolor{BrickRed}{;} \\
\mbox{}\ \ \ \ \textbf{\textcolor{Blue}{while}}\ \textcolor{BrickRed}{(}q\textcolor{BrickRed}{.}\textbf{\textcolor{Black}{size}}\textcolor{BrickRed}{())}\textcolor{Red}{\{} \\
\mbox{}\ \ \ \ \ \ node\ u\ \textcolor{BrickRed}{=}\ q\textcolor{BrickRed}{.}\textbf{\textcolor{Black}{top}}\textcolor{BrickRed}{().}second\textcolor{BrickRed}{;} \\
\mbox{}\ \ \ \ \ \ \textcolor{ForestGreen}{double}\ w\ \textcolor{BrickRed}{=}\ q\textcolor{BrickRed}{.}\textbf{\textcolor{Black}{top}}\textcolor{BrickRed}{().}first\textcolor{BrickRed}{;} \\
\mbox{}\ \ \ \ \ \ q\textcolor{BrickRed}{.}\textbf{\textcolor{Black}{pop}}\textcolor{BrickRed}{();}\ \textit{\textcolor{Brown}{//!!}} \\
\mbox{} \\
\mbox{}\ \ \ \ \ \ \textbf{\textcolor{Blue}{if}}\ \textcolor{BrickRed}{(}visited\textcolor{BrickRed}{.}\textbf{\textcolor{Black}{count}}\textcolor{BrickRed}{(}u\textcolor{BrickRed}{))}\ \textbf{\textcolor{Blue}{continue}}\textcolor{BrickRed}{;} \\
\mbox{} \\
\mbox{}\ \ \ \ \ \ visited\textcolor{BrickRed}{.}\textbf{\textcolor{Black}{insert}}\textcolor{BrickRed}{(}u\textcolor{BrickRed}{);} \\
\mbox{}\ \ \ \ \ \ total\ \textcolor{BrickRed}{+=}\ w\textcolor{BrickRed}{;} \\
\mbox{}\ \ \ \ \ \ vector\textcolor{BrickRed}{$<$}edge\textcolor{BrickRed}{$>$}\ \textcolor{BrickRed}{\&}vecinos\ \textcolor{BrickRed}{=}\ g\textcolor{BrickRed}{[}u\textcolor{BrickRed}{];} \\
\mbox{}\ \ \ \ \ \ \textbf{\textcolor{Blue}{for}}\ \textcolor{BrickRed}{(}\textcolor{ForestGreen}{int}\ i\textcolor{BrickRed}{=}\textcolor{Purple}{0}\textcolor{BrickRed}{;}\ i\textcolor{BrickRed}{$<$}vecinos\textcolor{BrickRed}{.}\textbf{\textcolor{Black}{size}}\textcolor{BrickRed}{();}\ \textcolor{BrickRed}{++}i\textcolor{BrickRed}{)}\textcolor{Red}{\{} \\
\mbox{}\ \ \ \ \ \ \ \ node\ v\ \textcolor{BrickRed}{=}\ vecinos\textcolor{BrickRed}{[}i\textcolor{BrickRed}{].}second\textcolor{BrickRed}{;} \\
\mbox{}\ \ \ \ \ \ \ \ \textcolor{ForestGreen}{double}\ w$\_$extra\ \textcolor{BrickRed}{=}\ vecinos\textcolor{BrickRed}{[}i\textcolor{BrickRed}{].}first\textcolor{BrickRed}{;} \\
\mbox{}\ \ \ \ \ \ \ \ \textbf{\textcolor{Blue}{if}}\ \textcolor{BrickRed}{(}visited\textcolor{BrickRed}{.}\textbf{\textcolor{Black}{count}}\textcolor{BrickRed}{(}v\textcolor{BrickRed}{)}\ \textcolor{BrickRed}{==}\ \textcolor{Purple}{0}\textcolor{BrickRed}{)}\textcolor{Red}{\{} \\
\mbox{}\ \ \ \ \ \ \ \ \ \ q\textcolor{BrickRed}{.}\textbf{\textcolor{Black}{push}}\textcolor{BrickRed}{(}\textbf{\textcolor{Black}{edge}}\textcolor{BrickRed}{(}w$\_$extra\textcolor{BrickRed}{,}\ v\textcolor{BrickRed}{));} \\
\mbox{}\ \ \ \ \ \ \ \ \textcolor{Red}{\}} \\
\mbox{}\ \ \ \ \ \ \textcolor{Red}{\}} \\
\mbox{}\ \ \ \ \textcolor{Red}{\}} \\
\mbox{} \\
\mbox{}\ \ \ \ \textbf{\textcolor{Blue}{if}}\ \textcolor{BrickRed}{(}total\ \textcolor{BrickRed}{$>$}\ length\textcolor{BrickRed}{)}\textcolor{Red}{\{} \\
\mbox{}\ \ \ \ \ \ cout\ \textcolor{BrickRed}{$<$$<$}\ \texttt{\textcolor{Red}{"{}Not\ enough\ cable"{}}}\ \textcolor{BrickRed}{$<$$<$}\ endl\textcolor{BrickRed}{;} \\
\mbox{}\ \ \ \ \textcolor{Red}{\}}\textbf{\textcolor{Blue}{else}}\textcolor{Red}{\{} \\
\mbox{}\ \ \ \ \ \ \textbf{\textcolor{Black}{printf}}\textcolor{BrickRed}{(}\texttt{\textcolor{Red}{"{}Need\ \%.1lf\ miles\ of\ cable}}\texttt{\textcolor{CarnationPink}{\textbackslash{}n}}\texttt{\textcolor{Red}{"{}}}\textcolor{BrickRed}{,}\ total\textcolor{BrickRed}{);} \\
\mbox{}\ \ \ \ \textcolor{Red}{\}} \\
\mbox{} \\
\mbox{}\ \ \textcolor{Red}{\}} \\
\mbox{}\ \ \textbf{\textcolor{Blue}{return}}\ \textcolor{Purple}{0}\textcolor{BrickRed}{;} \\
\mbox{}\textcolor{Red}{\}} \\

} \normalfont\normalsize
%.tex

\subsection{Minimum spanning tree: Algoritmo de Kruskal + Union-Find}
% Generator: GNU source-highlight, by Lorenzo Bettini, http://www.gnu.org/software/src-highlite

{\ttfamily \raggedright {
\noindent
\mbox{}\textbf{\textcolor{RoyalBlue}{\#include}}\ \texttt{\textcolor{Red}{$<$iostream$>$}} \\
\mbox{}\textbf{\textcolor{RoyalBlue}{\#include}}\ \texttt{\textcolor{Red}{$<$vector$>$}} \\
\mbox{}\textbf{\textcolor{RoyalBlue}{\#include}}\ \texttt{\textcolor{Red}{$<$algorithm$>$}} \\
\mbox{} \\
\mbox{}\textbf{\textcolor{Blue}{using}}\ \textbf{\textcolor{Blue}{namespace}}\ std\textcolor{BrickRed}{;} \\
\mbox{} \\
\mbox{}\textit{\textcolor{Brown}{/*}} \\
\mbox{}\textit{\textcolor{Brown}{Algoritmo\ de\ Kruskal,\ para\ encontrar\ el\ árbol\ de\ recubrimiento\ de\ mínima\ suma.}} \\
\mbox{}\textit{\textcolor{Brown}{*/}} \\
\mbox{} \\
\mbox{}\textbf{\textcolor{Blue}{struct}}\ edge\textcolor{Red}{\{} \\
\mbox{}\ \ \textcolor{ForestGreen}{int}\ start\textcolor{BrickRed}{,}\ end\textcolor{BrickRed}{,}\ weight\textcolor{BrickRed}{;} \\
\mbox{}\ \ \textcolor{ForestGreen}{bool}\ \textbf{\textcolor{Blue}{operator}}\ \textcolor{BrickRed}{$<$}\ \textcolor{BrickRed}{(}\textbf{\textcolor{Blue}{const}}\ edge\ \textcolor{BrickRed}{\&}that\textcolor{BrickRed}{)}\ \textbf{\textcolor{Blue}{const}}\ \textcolor{Red}{\{} \\
\mbox{}\ \ \ \ \textit{\textcolor{Brown}{//Si\ se\ desea\ encontrar\ el\ árbol\ de\ recubrimiento\ de\ máxima\ suma,\ cambiar\ el\ $<$\ por\ un\ $>$}} \\
\mbox{}\ \ \ \ \textbf{\textcolor{Blue}{return}}\ weight\ \textcolor{BrickRed}{$<$}\ that\textcolor{BrickRed}{.}weight\textcolor{BrickRed}{;} \\
\mbox{}\ \ \textcolor{Red}{\}} \\
\mbox{}\textcolor{Red}{\}}\textcolor{BrickRed}{;} \\
\mbox{} \\
\mbox{} \\
\mbox{}\textit{\textcolor{Brown}{/*\ Union\ find\ */}} \\
\mbox{}\textcolor{ForestGreen}{int}\ p\textcolor{BrickRed}{[}\textcolor{Purple}{10001}\textcolor{BrickRed}{],}\ rank\textcolor{BrickRed}{[}\textcolor{Purple}{10001}\textcolor{BrickRed}{];} \\
\mbox{}\textcolor{ForestGreen}{void}\ \textbf{\textcolor{Black}{make$\_$set}}\textcolor{BrickRed}{(}\textcolor{ForestGreen}{int}\ x\textcolor{BrickRed}{)}\textcolor{Red}{\{}\ p\textcolor{BrickRed}{[}x\textcolor{BrickRed}{]}\ \textcolor{BrickRed}{=}\ x\textcolor{BrickRed}{,}\ rank\textcolor{BrickRed}{[}x\textcolor{BrickRed}{]}\ \textcolor{BrickRed}{=}\ \textcolor{Purple}{0}\textcolor{BrickRed}{;}\ \textcolor{Red}{\}} \\
\mbox{}\textcolor{ForestGreen}{void}\ \textbf{\textcolor{Black}{link}}\textcolor{BrickRed}{(}\textcolor{ForestGreen}{int}\ x\textcolor{BrickRed}{,}\ \textcolor{ForestGreen}{int}\ y\textcolor{BrickRed}{)}\textcolor{Red}{\{}\ rank\textcolor{BrickRed}{[}x\textcolor{BrickRed}{]}\ \textcolor{BrickRed}{$>$}\ rank\textcolor{BrickRed}{[}y\textcolor{BrickRed}{]}\ \textcolor{BrickRed}{?}\ p\textcolor{BrickRed}{[}y\textcolor{BrickRed}{]}\ \textcolor{BrickRed}{=}\ x\ \textcolor{BrickRed}{:}\ p\textcolor{BrickRed}{[}x\textcolor{BrickRed}{]}\ \textcolor{BrickRed}{=}\ y\textcolor{BrickRed}{,}\ rank\textcolor{BrickRed}{[}x\textcolor{BrickRed}{]}\ \textcolor{BrickRed}{==}\ rank\textcolor{BrickRed}{[}y\textcolor{BrickRed}{]}\ \textcolor{BrickRed}{?}\ rank\textcolor{BrickRed}{[}y\textcolor{BrickRed}{]++:}\ \textcolor{Purple}{0}\textcolor{BrickRed}{;}\ \textcolor{Red}{\}} \\
\mbox{}\textcolor{ForestGreen}{int}\ \textbf{\textcolor{Black}{find$\_$set}}\textcolor{BrickRed}{(}\textcolor{ForestGreen}{int}\ x\textcolor{BrickRed}{)}\textcolor{Red}{\{}\ \textbf{\textcolor{Blue}{return}}\ x\ \textcolor{BrickRed}{!=}\ p\textcolor{BrickRed}{[}x\textcolor{BrickRed}{]}\ \textcolor{BrickRed}{?}\ p\textcolor{BrickRed}{[}x\textcolor{BrickRed}{]}\ \textcolor{BrickRed}{=}\ \textbf{\textcolor{Black}{find$\_$set}}\textcolor{BrickRed}{(}p\textcolor{BrickRed}{[}x\textcolor{BrickRed}{])}\ \textcolor{BrickRed}{:}\ p\textcolor{BrickRed}{[}x\textcolor{BrickRed}{];}\ \textcolor{Red}{\}} \\
\mbox{}\textcolor{ForestGreen}{void}\ \textbf{\textcolor{Black}{merge}}\textcolor{BrickRed}{(}\textcolor{ForestGreen}{int}\ x\textcolor{BrickRed}{,}\ \textcolor{ForestGreen}{int}\ y\textcolor{BrickRed}{)}\textcolor{Red}{\{}\ \textbf{\textcolor{Black}{link}}\textcolor{BrickRed}{(}\textbf{\textcolor{Black}{find$\_$set}}\textcolor{BrickRed}{(}x\textcolor{BrickRed}{),}\ \textbf{\textcolor{Black}{find$\_$set}}\textcolor{BrickRed}{(}y\textcolor{BrickRed}{));}\ \textcolor{Red}{\}} \\
\mbox{}\textit{\textcolor{Brown}{/*\ End\ union\ find\ */}} \\
\mbox{} \\
\mbox{} \\
\mbox{}\textcolor{ForestGreen}{int}\ \textbf{\textcolor{Black}{main}}\textcolor{BrickRed}{()}\textcolor{Red}{\{} \\
\mbox{}\ \ \textcolor{ForestGreen}{int}\ c\textcolor{BrickRed}{;} \\
\mbox{}\ \ cin\ \textcolor{BrickRed}{$>$$>$}\ c\textcolor{BrickRed}{;} \\
\mbox{}\ \ \textbf{\textcolor{Blue}{while}}\ \textcolor{BrickRed}{(}c\textcolor{BrickRed}{-\/-)}\textcolor{Red}{\{} \\
\mbox{}\ \ \ \ \textcolor{ForestGreen}{int}\ n\textcolor{BrickRed}{,}\ m\textcolor{BrickRed}{;} \\
\mbox{}\ \ \ \ cin\ \textcolor{BrickRed}{$>$$>$}\ n\ \textcolor{BrickRed}{$>$$>$}\ m\textcolor{BrickRed}{;} \\
\mbox{}\ \ \ \ vector\textcolor{BrickRed}{$<$}edge\textcolor{BrickRed}{$>$}\ e\textcolor{BrickRed}{;} \\
\mbox{}\ \ \ \ \textcolor{ForestGreen}{long}\ \textcolor{ForestGreen}{long}\ total\ \textcolor{BrickRed}{=}\ \textcolor{Purple}{0}\textcolor{BrickRed}{;} \\
\mbox{}\ \ \ \ \textbf{\textcolor{Blue}{while}}\ \textcolor{BrickRed}{(}m\textcolor{BrickRed}{-\/-)}\textcolor{Red}{\{} \\
\mbox{}\ \ \ \ \ \ edge\ t\textcolor{BrickRed}{;} \\
\mbox{}\ \ \ \ \ \ cin\ \textcolor{BrickRed}{$>$$>$}\ t\textcolor{BrickRed}{.}start\ \textcolor{BrickRed}{$>$$>$}\ t\textcolor{BrickRed}{.}end\ \textcolor{BrickRed}{$>$$>$}\ t\textcolor{BrickRed}{.}weight\textcolor{BrickRed}{;} \\
\mbox{}\ \ \ \ \ \ e\textcolor{BrickRed}{.}\textbf{\textcolor{Black}{push$\_$back}}\textcolor{BrickRed}{(}t\textcolor{BrickRed}{);} \\
\mbox{}\ \ \ \ \ \ total\ \textcolor{BrickRed}{+=}\ t\textcolor{BrickRed}{.}weight\textcolor{BrickRed}{;} \\
\mbox{}\ \ \ \ \textcolor{Red}{\}} \\
\mbox{}\ \ \ \ \textbf{\textcolor{Black}{sort}}\textcolor{BrickRed}{(}e\textcolor{BrickRed}{.}\textbf{\textcolor{Black}{begin}}\textcolor{BrickRed}{(),}\ e\textcolor{BrickRed}{.}\textbf{\textcolor{Black}{end}}\textcolor{BrickRed}{());} \\
\mbox{}\ \ \ \ \textbf{\textcolor{Blue}{for}}\ \textcolor{BrickRed}{(}\textcolor{ForestGreen}{int}\ i\textcolor{BrickRed}{=}\textcolor{Purple}{0}\textcolor{BrickRed}{;}\ i\textcolor{BrickRed}{$<$=}n\textcolor{BrickRed}{;}\ \textcolor{BrickRed}{++}i\textcolor{BrickRed}{)}\textcolor{Red}{\{} \\
\mbox{}\ \ \ \ \ \ \textbf{\textcolor{Black}{make$\_$set}}\textcolor{BrickRed}{(}i\textcolor{BrickRed}{);} \\
\mbox{}\ \ \ \ \textcolor{Red}{\}} \\
\mbox{}\ \ \ \ \textbf{\textcolor{Blue}{for}}\ \textcolor{BrickRed}{(}\textcolor{ForestGreen}{int}\ i\textcolor{BrickRed}{=}\textcolor{Purple}{0}\textcolor{BrickRed}{;}\ i\textcolor{BrickRed}{$<$}e\textcolor{BrickRed}{.}\textbf{\textcolor{Black}{size}}\textcolor{BrickRed}{();}\ \textcolor{BrickRed}{++}i\textcolor{BrickRed}{)}\textcolor{Red}{\{} \\
\mbox{}\ \ \ \ \ \ \textcolor{ForestGreen}{int}\ u\ \textcolor{BrickRed}{=}\ e\textcolor{BrickRed}{[}i\textcolor{BrickRed}{].}start\textcolor{BrickRed}{,}\ v\ \textcolor{BrickRed}{=}\ e\textcolor{BrickRed}{[}i\textcolor{BrickRed}{].}end\textcolor{BrickRed}{,}\ w\ \textcolor{BrickRed}{=}\ e\textcolor{BrickRed}{[}i\textcolor{BrickRed}{].}weight\textcolor{BrickRed}{;} \\
\mbox{}\ \ \ \ \ \ \textbf{\textcolor{Blue}{if}}\ \textcolor{BrickRed}{(}\textbf{\textcolor{Black}{find$\_$set}}\textcolor{BrickRed}{(}u\textcolor{BrickRed}{)}\ \textcolor{BrickRed}{!=}\ \textbf{\textcolor{Black}{find$\_$set}}\textcolor{BrickRed}{(}v\textcolor{BrickRed}{))}\textcolor{Red}{\{} \\
\mbox{}\ \ \ \ \ \ \ \ \textit{\textcolor{Brown}{//printf("{}Joining\ \%d\ with\ \%d\ using\ weight\ \%d\textbackslash{}n"{},\ u,\ v,\ w);}} \\
\mbox{}\ \ \ \ \ \ \ \ total\ \textcolor{BrickRed}{-=}\ w\textcolor{BrickRed}{;} \\
\mbox{}\ \ \ \ \ \ \ \ \textbf{\textcolor{Black}{merge}}\textcolor{BrickRed}{(}u\textcolor{BrickRed}{,}\ v\textcolor{BrickRed}{);} \\
\mbox{}\ \ \ \ \ \ \textcolor{Red}{\}} \\
\mbox{}\ \ \ \ \textcolor{Red}{\}} \\
\mbox{}\ \ \ \ cout\ \textcolor{BrickRed}{$<$$<$}\ total\ \textcolor{BrickRed}{$<$$<$}\ endl\textcolor{BrickRed}{;} \\
\mbox{} \\
\mbox{}\ \ \textcolor{Red}{\}} \\
\mbox{}\ \ \textbf{\textcolor{Blue}{return}}\ \textcolor{Purple}{0}\textcolor{BrickRed}{;} \\
\mbox{}\textcolor{Red}{\}} \\

} \normalfont\normalsize
%.tex

\subsection{Algoritmo de Floyd-Warshall}
\emph{Complejidad:} $ O(n^3) $ \\
Se asume que no hay ciclos de costo negativo.
% Generator: GNU source-highlight, by Lorenzo Bettini, http://www.gnu.org/software/src-highlite

{\ttfamily \raggedright {
\noindent
\mbox{}\textit{\textcolor{Brown}{/*}} \\
\mbox{}\textit{\textcolor{Brown}{\ \ g[i][j]\ =\ Distancia\ entre\ el\ nodo\ i\ y\ el\ j.}} \\
\mbox{}\textit{\textcolor{Brown}{\ */}} \\
\mbox{}\textcolor{ForestGreen}{unsigned}\ \textcolor{ForestGreen}{long}\ \textcolor{ForestGreen}{long}\ g\textcolor{BrickRed}{[}\textcolor{Purple}{101}\textcolor{BrickRed}{][}\textcolor{Purple}{101}\textcolor{BrickRed}{];} \\
\mbox{} \\
\mbox{}\textcolor{ForestGreen}{void}\ \textbf{\textcolor{Black}{floyd}}\textcolor{BrickRed}{()}\textcolor{Red}{\{} \\
\mbox{}\ \ \textit{\textcolor{Brown}{//Llenar\ g}} \\
\mbox{}\ \ \textit{\textcolor{Brown}{//...}} \\
\mbox{} \\
\mbox{}\ \ \textbf{\textcolor{Blue}{for}}\ \textcolor{BrickRed}{(}\textcolor{ForestGreen}{int}\ k\textcolor{BrickRed}{=}\textcolor{Purple}{0}\textcolor{BrickRed}{;}\ k\textcolor{BrickRed}{$<$}n\textcolor{BrickRed}{;}\ \textcolor{BrickRed}{++}k\textcolor{BrickRed}{)}\textcolor{Red}{\{} \\
\mbox{}\ \ \ \ \textbf{\textcolor{Blue}{for}}\ \textcolor{BrickRed}{(}\textcolor{ForestGreen}{int}\ i\textcolor{BrickRed}{=}\textcolor{Purple}{0}\textcolor{BrickRed}{;}\ i\textcolor{BrickRed}{$<$}n\textcolor{BrickRed}{;}\ \textcolor{BrickRed}{++}i\textcolor{BrickRed}{)}\textcolor{Red}{\{} \\
\mbox{}\ \ \ \ \ \ \textbf{\textcolor{Blue}{for}}\ \textcolor{BrickRed}{(}\textcolor{ForestGreen}{int}\ j\textcolor{BrickRed}{=}\textcolor{Purple}{0}\textcolor{BrickRed}{;}\ j\textcolor{BrickRed}{$<$}n\textcolor{BrickRed}{;}\ \textcolor{BrickRed}{++}j\textcolor{BrickRed}{)}\textcolor{Red}{\{} \\
\mbox{}\ \ \ \ \ \ \ \ g\textcolor{BrickRed}{[}i\textcolor{BrickRed}{][}j\textcolor{BrickRed}{]}\ \textcolor{BrickRed}{=}\ \textbf{\textcolor{Black}{min}}\textcolor{BrickRed}{(}g\textcolor{BrickRed}{[}i\textcolor{BrickRed}{][}j\textcolor{BrickRed}{],}\ g\textcolor{BrickRed}{[}i\textcolor{BrickRed}{][}k\textcolor{BrickRed}{]}\ \textcolor{BrickRed}{+}\ g\textcolor{BrickRed}{[}k\textcolor{BrickRed}{][}j\textcolor{BrickRed}{]);} \\
\mbox{}\ \ \ \ \ \ \textcolor{Red}{\}} \\
\mbox{}\ \ \ \ \textcolor{Red}{\}} \\
\mbox{}\ \ \textcolor{Red}{\}} \\
\mbox{}\ \ \textit{\textcolor{Brown}{/*}} \\
\mbox{}\textit{\textcolor{Brown}{\ \ \ \ Acá\ se\ cumple\ que\ g[i][j]\ =\ Longitud\ de\ la\ ruta\ más\ corta\ de\ i\ a\ j.}} \\
\mbox{}\textit{\textcolor{Brown}{\ \ \ */}} \\
\mbox{}\textcolor{Red}{\}} \\

} \normalfont\normalsize
%.tex

\subsection{Algoritmo de Bellman-Ford}
Si no hay ciclos de coste negativo, encuentra la distancia más corta entre un nodo
y todos los demás. Si sí hay, permite saberlo. \\
El coste de las aristas \underline{sí} puede ser negativo.
% Generator: GNU source-highlight, by Lorenzo Bettini, http://www.gnu.org/software/src-highlite

{\ttfamily \raggedright {
\noindent
\mbox{}\textbf{\textcolor{Blue}{struct}}\ edge\textcolor{Red}{\{} \\
\mbox{}\ \ \textcolor{ForestGreen}{int}\ u\textcolor{BrickRed}{,}\ v\textcolor{BrickRed}{,}\ w\textcolor{BrickRed}{;} \\
\mbox{}\textcolor{Red}{\}}\textcolor{BrickRed}{;} \\
\mbox{} \\
\mbox{}edge\ \textcolor{BrickRed}{*}\ e\textcolor{BrickRed}{;}\ \textit{\textcolor{Brown}{//e\ =\ Arreglo\ de\ todas\ las\ aristas}} \\
\mbox{}\textcolor{ForestGreen}{long}\ \textcolor{ForestGreen}{long}\ d\textcolor{BrickRed}{[}\textcolor{Purple}{300}\textcolor{BrickRed}{];}\ \textit{\textcolor{Brown}{//Distancias}} \\
\mbox{}\textcolor{ForestGreen}{int}\ n\textcolor{BrickRed}{;}\ \textit{\textcolor{Brown}{//Cantidad\ de\ nodos}} \\
\mbox{}\textcolor{ForestGreen}{int}\ m\textcolor{BrickRed}{;}\ \textit{\textcolor{Brown}{//Cantidad\ de\ aristas}} \\
\mbox{} \\
\mbox{}\textit{\textcolor{Brown}{/*}} \\
\mbox{}\textit{\textcolor{Brown}{\ \ Retorna\ falso\ si\ hay\ un\ ciclo\ de\ costo\ negativo.}} \\
\mbox{} \\
\mbox{}\textit{\textcolor{Brown}{\ \ Si\ retorna\ verdadero,\ entonces\ d[i]\ contiene\ la\ distancia\ más\ corta\ entre\ el\ s\ y\ el\ nodo\ i.}} \\
\mbox{}\textit{\textcolor{Brown}{\ */}} \\
\mbox{}\textcolor{ForestGreen}{bool}\ \textbf{\textcolor{Black}{bellman}}\textcolor{BrickRed}{(}\textcolor{ForestGreen}{int}\ \textcolor{BrickRed}{\&}s\textcolor{BrickRed}{)}\textcolor{Red}{\{} \\
\mbox{}\ \ \textit{\textcolor{Brown}{//Llenar\ e}} \\
\mbox{}\ \ e\ \textcolor{BrickRed}{=}\ \textbf{\textcolor{Blue}{new}}\ edge\textcolor{BrickRed}{[}n\textcolor{BrickRed}{];} \\
\mbox{}\ \ \textit{\textcolor{Brown}{//...}} \\
\mbox{} \\
\mbox{}\ \ \textbf{\textcolor{Blue}{for}}\ \textcolor{BrickRed}{(}\textcolor{ForestGreen}{int}\ i\textcolor{BrickRed}{=}\textcolor{Purple}{0}\textcolor{BrickRed}{;}\ i\textcolor{BrickRed}{$<$}n\textcolor{BrickRed}{;}\ \textcolor{BrickRed}{++}i\textcolor{BrickRed}{)}\ d\textcolor{BrickRed}{[}i\textcolor{BrickRed}{]}\ \textcolor{BrickRed}{=}\ INT$\_$MAX\textcolor{BrickRed}{;} \\
\mbox{}\ \ d\textcolor{BrickRed}{[}s\textcolor{BrickRed}{]}\ \textcolor{BrickRed}{=}\ 0LL\textcolor{BrickRed}{;} \\
\mbox{} \\
\mbox{}\ \ \textbf{\textcolor{Blue}{for}}\ \textcolor{BrickRed}{(}\textcolor{ForestGreen}{int}\ i\textcolor{BrickRed}{=}\textcolor{Purple}{0}\textcolor{BrickRed}{;}\ i\textcolor{BrickRed}{$<$}n\textcolor{BrickRed}{-}\textcolor{Purple}{1}\textcolor{BrickRed}{;}\ \textcolor{BrickRed}{++}i\textcolor{BrickRed}{)}\textcolor{Red}{\{} \\
\mbox{}\ \ \ \ \textcolor{ForestGreen}{bool}\ cambio\ \textcolor{BrickRed}{=}\ \textbf{\textcolor{Blue}{false}}\textcolor{BrickRed}{;} \\
\mbox{}\ \ \ \ \textbf{\textcolor{Blue}{for}}\ \textcolor{BrickRed}{(}\textcolor{ForestGreen}{int}\ j\textcolor{BrickRed}{=}\textcolor{Purple}{0}\textcolor{BrickRed}{;}\ j\textcolor{BrickRed}{$<$}m\textcolor{BrickRed}{;}\ \textcolor{BrickRed}{++}j\textcolor{BrickRed}{)}\textcolor{Red}{\{} \\
\mbox{}\ \ \ \ \ \ \textcolor{ForestGreen}{int}\ u\ \textcolor{BrickRed}{=}\ e\textcolor{BrickRed}{[}j\textcolor{BrickRed}{].}u\textcolor{BrickRed}{,}\ v\ \textcolor{BrickRed}{=}\ e\textcolor{BrickRed}{[}j\textcolor{BrickRed}{].}v\textcolor{BrickRed}{;} \\
\mbox{}\ \ \ \ \ \ \textcolor{ForestGreen}{long}\ \textcolor{ForestGreen}{long}\ w\ \textcolor{BrickRed}{=}\ e\textcolor{BrickRed}{[}j\textcolor{BrickRed}{].}w\textcolor{BrickRed}{;} \\
\mbox{}\ \ \ \ \ \ \textbf{\textcolor{Blue}{if}}\ \textcolor{BrickRed}{(}d\textcolor{BrickRed}{[}u\textcolor{BrickRed}{]}\ \textcolor{BrickRed}{+}\ w\ \textcolor{BrickRed}{$<$}\ d\textcolor{BrickRed}{[}v\textcolor{BrickRed}{])}\textcolor{Red}{\{} \\
\mbox{}\ \ \ \ \ \ \ \ d\textcolor{BrickRed}{[}v\textcolor{BrickRed}{]}\ \textcolor{BrickRed}{=}\ d\textcolor{BrickRed}{[}u\textcolor{BrickRed}{]}\ \textcolor{BrickRed}{+}\ w\textcolor{BrickRed}{;} \\
\mbox{}\ \ \ \ \ \ \ \ cambio\ \textcolor{BrickRed}{=}\ \textbf{\textcolor{Blue}{true}}\textcolor{BrickRed}{;} \\
\mbox{}\ \ \ \ \ \ \textcolor{Red}{\}} \\
\mbox{}\ \ \ \ \textcolor{Red}{\}} \\
\mbox{}\ \ \ \ \textbf{\textcolor{Blue}{if}}\ \textcolor{BrickRed}{(!}cambio\textcolor{BrickRed}{)}\ \textbf{\textcolor{Blue}{break}}\textcolor{BrickRed}{;}\ \textit{\textcolor{Brown}{//Early-exit}} \\
\mbox{}\ \ \textcolor{Red}{\}} \\
\mbox{} \\
\mbox{}\ \ \textbf{\textcolor{Blue}{for}}\ \textcolor{BrickRed}{(}\textcolor{ForestGreen}{int}\ j\textcolor{BrickRed}{=}\textcolor{Purple}{0}\textcolor{BrickRed}{;}\ j\textcolor{BrickRed}{$<$}m\textcolor{BrickRed}{;}\ \textcolor{BrickRed}{++}j\textcolor{BrickRed}{)}\textcolor{Red}{\{} \\
\mbox{}\ \ \ \ \textcolor{ForestGreen}{int}\ u\ \textcolor{BrickRed}{=}\ e\textcolor{BrickRed}{[}j\textcolor{BrickRed}{].}u\textcolor{BrickRed}{,}\ v\ \textcolor{BrickRed}{=}\ e\textcolor{BrickRed}{[}j\textcolor{BrickRed}{].}v\textcolor{BrickRed}{;} \\
\mbox{}\ \ \ \ \textcolor{ForestGreen}{long}\ \textcolor{ForestGreen}{long}\ w\ \textcolor{BrickRed}{=}\ e\textcolor{BrickRed}{[}j\textcolor{BrickRed}{].}w\textcolor{BrickRed}{;} \\
\mbox{}\ \ \ \ \textbf{\textcolor{Blue}{if}}\ \textcolor{BrickRed}{(}d\textcolor{BrickRed}{[}u\textcolor{BrickRed}{]}\ \textcolor{BrickRed}{+}\ w\ \textcolor{BrickRed}{$<$}\ d\textcolor{BrickRed}{[}v\textcolor{BrickRed}{])}\ \textbf{\textcolor{Blue}{return}}\ \textbf{\textcolor{Blue}{false}}\textcolor{BrickRed}{;} \\
\mbox{}\ \ \textcolor{Red}{\}} \\
\mbox{} \\
\mbox{}\ \ \textbf{\textcolor{Blue}{delete}}\ \textcolor{BrickRed}{[]}\ e\textcolor{BrickRed}{;} \\
\mbox{}\ \ \textbf{\textcolor{Blue}{return}}\ \textbf{\textcolor{Blue}{true}}\textcolor{BrickRed}{;} \\
\mbox{}\textcolor{Red}{\}} \\

} \normalfont\normalsize
%.tex

\subsection{Puntos de articulación}
% Generator: GNU source-highlight, by Lorenzo Bettini, http://www.gnu.org/software/src-highlite

{\ttfamily \raggedright {
\noindent
\mbox{}\textbf{\textcolor{RoyalBlue}{\#include}}\ \texttt{\textcolor{Red}{$<$vector$>$}} \\
\mbox{}\textbf{\textcolor{RoyalBlue}{\#include}}\ \texttt{\textcolor{Red}{$<$set$>$}} \\
\mbox{}\textbf{\textcolor{RoyalBlue}{\#include}}\ \texttt{\textcolor{Red}{$<$map$>$}} \\
\mbox{}\textbf{\textcolor{RoyalBlue}{\#include}}\ \texttt{\textcolor{Red}{$<$algorithm$>$}} \\
\mbox{}\textbf{\textcolor{RoyalBlue}{\#include}}\ \texttt{\textcolor{Red}{$<$iostream$>$}} \\
\mbox{}\textbf{\textcolor{RoyalBlue}{\#include}}\ \texttt{\textcolor{Red}{$<$iterator$>$}} \\
\mbox{} \\
\mbox{}\textbf{\textcolor{Blue}{using}}\ \textbf{\textcolor{Blue}{namespace}}\ std\textcolor{BrickRed}{;} \\
\mbox{} \\
\mbox{}\textbf{\textcolor{Blue}{typedef}}\ string\ node\textcolor{BrickRed}{;} \\
\mbox{}\textbf{\textcolor{Blue}{typedef}}\ map\textcolor{BrickRed}{$<$}node\textcolor{BrickRed}{,}\ vector\textcolor{BrickRed}{$<$}node\textcolor{BrickRed}{$>$}\ \textcolor{BrickRed}{$>$}\ graph\textcolor{BrickRed}{;} \\
\mbox{}\textbf{\textcolor{Blue}{typedef}}\ \textcolor{ForestGreen}{char}\ color\textcolor{BrickRed}{;} \\
\mbox{} \\
\mbox{}\textbf{\textcolor{Blue}{const}}\ color\ WHITE\ \textcolor{BrickRed}{=}\ \textcolor{Purple}{0}\textcolor{BrickRed}{,}\ GRAY\ \textcolor{BrickRed}{=}\ \textcolor{Purple}{1}\textcolor{BrickRed}{,}\ BLACK\ \textcolor{BrickRed}{=}\ \textcolor{Purple}{2}\textcolor{BrickRed}{;} \\
\mbox{} \\
\mbox{}graph\ g\textcolor{BrickRed}{;} \\
\mbox{}map\textcolor{BrickRed}{$<$}node\textcolor{BrickRed}{,}\ color\textcolor{BrickRed}{$>$}\ colors\textcolor{BrickRed}{;} \\
\mbox{}map\textcolor{BrickRed}{$<$}node\textcolor{BrickRed}{,}\ \textcolor{ForestGreen}{int}\textcolor{BrickRed}{$>$}\ d\textcolor{BrickRed}{,}\ low\textcolor{BrickRed}{;} \\
\mbox{} \\
\mbox{}set\textcolor{BrickRed}{$<$}node\textcolor{BrickRed}{$>$}\ cameras\textcolor{BrickRed}{;} \\
\mbox{} \\
\mbox{}\textcolor{ForestGreen}{int}\ timeCount\textcolor{BrickRed}{;} \\
\mbox{} \\
\mbox{}\textcolor{ForestGreen}{void}\ \textbf{\textcolor{Black}{dfs}}\textcolor{BrickRed}{(}node\ v\textcolor{BrickRed}{,}\ \textcolor{ForestGreen}{bool}\ isRoot\ \textcolor{BrickRed}{=}\ \textbf{\textcolor{Blue}{true}}\textcolor{BrickRed}{)}\textcolor{Red}{\{} \\
\mbox{}\ \ colors\textcolor{BrickRed}{[}v\textcolor{BrickRed}{]}\ \textcolor{BrickRed}{=}\ GRAY\textcolor{BrickRed}{;} \\
\mbox{}\ \ d\textcolor{BrickRed}{[}v\textcolor{BrickRed}{]}\ \textcolor{BrickRed}{=}\ low\textcolor{BrickRed}{[}v\textcolor{BrickRed}{]}\ \textcolor{BrickRed}{=}\ \textcolor{BrickRed}{++}timeCount\textcolor{BrickRed}{;} \\
\mbox{}\ \ vector\textcolor{BrickRed}{$<$}node\textcolor{BrickRed}{$>$}\ neighbors\ \textcolor{BrickRed}{=}\ g\textcolor{BrickRed}{[}v\textcolor{BrickRed}{];} \\
\mbox{}\ \ \textcolor{ForestGreen}{int}\ count\ \textcolor{BrickRed}{=}\ \textcolor{Purple}{0}\textcolor{BrickRed}{;} \\
\mbox{}\ \ \textbf{\textcolor{Blue}{for}}\ \textcolor{BrickRed}{(}\textcolor{ForestGreen}{int}\ i\textcolor{BrickRed}{=}\textcolor{Purple}{0}\textcolor{BrickRed}{;}\ i\textcolor{BrickRed}{$<$}neighbors\textcolor{BrickRed}{.}\textbf{\textcolor{Black}{size}}\textcolor{BrickRed}{();}\ \textcolor{BrickRed}{++}i\textcolor{BrickRed}{)}\textcolor{Red}{\{} \\
\mbox{}\ \ \ \ \textbf{\textcolor{Blue}{if}}\ \textcolor{BrickRed}{(}colors\textcolor{BrickRed}{[}neighbors\textcolor{BrickRed}{[}i\textcolor{BrickRed}{]]}\ \textcolor{BrickRed}{==}\ WHITE\textcolor{BrickRed}{)}\textcolor{Red}{\{}\ \textit{\textcolor{Brown}{//\ \ (v,\ neighbors[i])\ is\ a\ tree\ edge}} \\
\mbox{}\ \ \ \ \ \ \textbf{\textcolor{Black}{dfs}}\textcolor{BrickRed}{(}neighbors\textcolor{BrickRed}{[}i\textcolor{BrickRed}{],}\ \textbf{\textcolor{Blue}{false}}\textcolor{BrickRed}{);} \\
\mbox{}\ \ \ \ \ \ \textbf{\textcolor{Blue}{if}}\ \textcolor{BrickRed}{(!}isRoot\ \textcolor{BrickRed}{\&\&}\ low\textcolor{BrickRed}{[}neighbors\textcolor{BrickRed}{[}i\textcolor{BrickRed}{]]}\ \textcolor{BrickRed}{$>$=}\ d\textcolor{BrickRed}{[}v\textcolor{BrickRed}{])}\textcolor{Red}{\{} \\
\mbox{}\ \ \ \ \ \ \ \ cameras\textcolor{BrickRed}{.}\textbf{\textcolor{Black}{insert}}\textcolor{BrickRed}{(}v\textcolor{BrickRed}{);} \\
\mbox{}\ \ \ \ \ \ \textcolor{Red}{\}} \\
\mbox{}\ \ \ \ \ \ low\textcolor{BrickRed}{[}v\textcolor{BrickRed}{]}\ \textcolor{BrickRed}{=}\ \textbf{\textcolor{Black}{min}}\textcolor{BrickRed}{(}low\textcolor{BrickRed}{[}v\textcolor{BrickRed}{],}\ low\textcolor{BrickRed}{[}neighbors\textcolor{BrickRed}{[}i\textcolor{BrickRed}{]]);} \\
\mbox{}\ \ \ \ \ \ \textcolor{BrickRed}{++}count\textcolor{BrickRed}{;} \\
\mbox{}\ \ \ \ \textcolor{Red}{\}}\textbf{\textcolor{Blue}{else}}\textcolor{Red}{\{}\ \textit{\textcolor{Brown}{//\ (v,\ neighbors[i])\ is\ a\ back\ edge}} \\
\mbox{}\ \ \ \ \ \ low\textcolor{BrickRed}{[}v\textcolor{BrickRed}{]}\ \textcolor{BrickRed}{=}\ \textbf{\textcolor{Black}{min}}\textcolor{BrickRed}{(}low\textcolor{BrickRed}{[}v\textcolor{BrickRed}{],}\ d\textcolor{BrickRed}{[}neighbors\textcolor{BrickRed}{[}i\textcolor{BrickRed}{]]);} \\
\mbox{}\ \ \ \ \textcolor{Red}{\}} \\
\mbox{}\ \ \textcolor{Red}{\}} \\
\mbox{}\ \ \textbf{\textcolor{Blue}{if}}\ \textcolor{BrickRed}{(}isRoot\ \textcolor{BrickRed}{\&\&}\ count\ \textcolor{BrickRed}{$>$}\ \textcolor{Purple}{1}\textcolor{BrickRed}{)}\textcolor{Red}{\{}\ \textit{\textcolor{Brown}{//Is\ root\ and\ has\ two\ neighbors\ in\ the\ DFS-tree}} \\
\mbox{}\ \ \ \ cameras\textcolor{BrickRed}{.}\textbf{\textcolor{Black}{insert}}\textcolor{BrickRed}{(}v\textcolor{BrickRed}{);} \\
\mbox{}\ \ \textcolor{Red}{\}} \\
\mbox{}\ \ colors\textcolor{BrickRed}{[}v\textcolor{BrickRed}{]}\ \textcolor{BrickRed}{=}\ BLACK\textcolor{BrickRed}{;} \\
\mbox{}\textcolor{Red}{\}} \\
\mbox{} \\
\mbox{}\textcolor{ForestGreen}{int}\ \textbf{\textcolor{Black}{main}}\textcolor{BrickRed}{()}\textcolor{Red}{\{} \\
\mbox{}\ \ \textcolor{ForestGreen}{int}\ n\textcolor{BrickRed}{;} \\
\mbox{}\ \ \textcolor{ForestGreen}{int}\ map\ \textcolor{BrickRed}{=}\ \textcolor{Purple}{1}\textcolor{BrickRed}{;} \\
\mbox{}\ \ \textbf{\textcolor{Blue}{while}}\ \textcolor{BrickRed}{(}cin\ \textcolor{BrickRed}{$>$$>$}\ n\ \textcolor{BrickRed}{\&\&}\ n\ \textcolor{BrickRed}{$>$}\ \textcolor{Purple}{0}\textcolor{BrickRed}{)}\textcolor{Red}{\{} \\
\mbox{}\ \ \ \ \textbf{\textcolor{Blue}{if}}\ \textcolor{BrickRed}{(}map\ \textcolor{BrickRed}{$>$}\ \textcolor{Purple}{1}\textcolor{BrickRed}{)}\ cout\ \textcolor{BrickRed}{$<$$<$}\ endl\textcolor{BrickRed}{;} \\
\mbox{}\ \ \ \ g\textcolor{BrickRed}{.}\textbf{\textcolor{Black}{clear}}\textcolor{BrickRed}{();} \\
\mbox{}\ \ \ \ colors\textcolor{BrickRed}{.}\textbf{\textcolor{Black}{clear}}\textcolor{BrickRed}{();} \\
\mbox{}\ \ \ \ d\textcolor{BrickRed}{.}\textbf{\textcolor{Black}{clear}}\textcolor{BrickRed}{();} \\
\mbox{}\ \ \ \ low\textcolor{BrickRed}{.}\textbf{\textcolor{Black}{clear}}\textcolor{BrickRed}{();} \\
\mbox{}\ \ \ \ timeCount\ \textcolor{BrickRed}{=}\ \textcolor{Purple}{0}\textcolor{BrickRed}{;} \\
\mbox{}\ \ \ \ \textbf{\textcolor{Blue}{while}}\ \textcolor{BrickRed}{(}n\textcolor{BrickRed}{-\/-)}\textcolor{Red}{\{} \\
\mbox{}\ \ \ \ \ \ node\ v\textcolor{BrickRed}{;} \\
\mbox{}\ \ \ \ \ \ cin\ \textcolor{BrickRed}{$>$$>$}\ v\textcolor{BrickRed}{;} \\
\mbox{}\ \ \ \ \ \ colors\textcolor{BrickRed}{[}v\textcolor{BrickRed}{]}\ \textcolor{BrickRed}{=}\ WHITE\textcolor{BrickRed}{;} \\
\mbox{}\ \ \ \ \ \ g\textcolor{BrickRed}{[}v\textcolor{BrickRed}{]}\ \textcolor{BrickRed}{=}\ vector\textcolor{BrickRed}{$<$}node\textcolor{BrickRed}{$>$();} \\
\mbox{}\ \ \ \ \textcolor{Red}{\}} \\
\mbox{}\ \ \ \  \\
\mbox{}\ \ \ \ cin\ \textcolor{BrickRed}{$>$$>$}\ n\textcolor{BrickRed}{;} \\
\mbox{}\ \ \ \ \textbf{\textcolor{Blue}{while}}\ \textcolor{BrickRed}{(}n\textcolor{BrickRed}{-\/-)}\textcolor{Red}{\{} \\
\mbox{}\ \ \ \ \ \ node\ v\textcolor{BrickRed}{,}u\textcolor{BrickRed}{;} \\
\mbox{}\ \ \ \ \ \ cin\ \textcolor{BrickRed}{$>$$>$}\ v\ \textcolor{BrickRed}{$>$$>$}\ u\textcolor{BrickRed}{;} \\
\mbox{}\ \ \ \ \ \ g\textcolor{BrickRed}{[}v\textcolor{BrickRed}{].}\textbf{\textcolor{Black}{push$\_$back}}\textcolor{BrickRed}{(}u\textcolor{BrickRed}{);} \\
\mbox{}\ \ \ \ \ \ g\textcolor{BrickRed}{[}u\textcolor{BrickRed}{].}\textbf{\textcolor{Black}{push$\_$back}}\textcolor{BrickRed}{(}v\textcolor{BrickRed}{);} \\
\mbox{}\ \ \ \ \textcolor{Red}{\}} \\
\mbox{}\ \ \ \  \\
\mbox{}\ \ \ \ cameras\textcolor{BrickRed}{.}\textbf{\textcolor{Black}{clear}}\textcolor{BrickRed}{();} \\
\mbox{}\ \ \ \  \\
\mbox{}\ \ \ \ \textbf{\textcolor{Blue}{for}}\ \textcolor{BrickRed}{(}graph\textcolor{BrickRed}{::}iterator\ i\ \textcolor{BrickRed}{=}\ g\textcolor{BrickRed}{.}\textbf{\textcolor{Black}{begin}}\textcolor{BrickRed}{();}\ i\ \textcolor{BrickRed}{!=}\ g\textcolor{BrickRed}{.}\textbf{\textcolor{Black}{end}}\textcolor{BrickRed}{();}\ \textcolor{BrickRed}{++}i\textcolor{BrickRed}{)}\textcolor{Red}{\{} \\
\mbox{}\ \ \ \ \ \ \textbf{\textcolor{Blue}{if}}\ \textcolor{BrickRed}{(}colors\textcolor{BrickRed}{[(*}i\textcolor{BrickRed}{).}first\textcolor{BrickRed}{]}\ \textcolor{BrickRed}{==}\ WHITE\textcolor{BrickRed}{)}\textcolor{Red}{\{} \\
\mbox{}\ \ \ \ \ \ \ \ \textbf{\textcolor{Black}{dfs}}\textcolor{BrickRed}{((*}i\textcolor{BrickRed}{).}first\textcolor{BrickRed}{);} \\
\mbox{}\ \ \ \ \ \ \textcolor{Red}{\}} \\
\mbox{}\ \ \ \ \textcolor{Red}{\}} \\
\mbox{}\ \ \ \ \ \  \\
\mbox{}\ \ \ \ cout\ \textcolor{BrickRed}{$<$$<$}\ \texttt{\textcolor{Red}{"{}City\ map\ \#"{}}}\textcolor{BrickRed}{$<$$<$}map\textcolor{BrickRed}{$<$$<$}\texttt{\textcolor{Red}{"{}:\ "{}}}\ \textcolor{BrickRed}{$<$$<$}\ cameras\textcolor{BrickRed}{.}\textbf{\textcolor{Black}{size}}\textcolor{BrickRed}{()}\ \textcolor{BrickRed}{$<$$<$}\ \texttt{\textcolor{Red}{"{}\ camera(s)\ found"{}}}\ \textcolor{BrickRed}{$<$$<$}\ endl\textcolor{BrickRed}{;} \\
\mbox{}\ \ \ \ \textbf{\textcolor{Black}{copy}}\textcolor{BrickRed}{(}cameras\textcolor{BrickRed}{.}\textbf{\textcolor{Black}{begin}}\textcolor{BrickRed}{(),}\ cameras\textcolor{BrickRed}{.}\textbf{\textcolor{Black}{end}}\textcolor{BrickRed}{(),}\ ostream$\_$iterator\textcolor{BrickRed}{$<$}node\textcolor{BrickRed}{$>$(}cout\textcolor{BrickRed}{,}\texttt{\textcolor{Red}{"{}}}\texttt{\textcolor{CarnationPink}{\textbackslash{}n}}\texttt{\textcolor{Red}{"{}}}\textcolor{BrickRed}{));} \\
\mbox{}\ \ \ \ \textcolor{BrickRed}{++}map\textcolor{BrickRed}{;} \\
\mbox{}\ \ \textcolor{Red}{\}} \\
\mbox{}\ \ \textbf{\textcolor{Blue}{return}}\ \textcolor{Purple}{0}\textcolor{BrickRed}{;} \\
\mbox{}\textcolor{Red}{\}} \\

} \normalfont\normalsize
%.tex

\subsection{Máximo flujo: Método de Ford-Fulkerson, algoritmo de Edmonds-Karp}
El algoritmo de Edmonds-Karp es una modificación al método de Ford-Fulkerson. Este último
utiliza DFS para hallar un camino de aumentación, pero la sugerencia de Edmonds-Karp
es utilizar BFS que lo hace más eficiente en algunos grafos.
\medskip

% Generator: GNU source-highlight, by Lorenzo Bettini, http://www.gnu.org/software/src-highlite

{\ttfamily \raggedright {
\noindent
\mbox{}\textit{\textcolor{Brown}{/*}} \\
\mbox{}\textit{\textcolor{Brown}{\ \ cap[i][j]\ =\ Capacidad\ de\ la\ arista\ (i,\ j).}} \\
\mbox{}\textit{\textcolor{Brown}{\ \ prev[i]\ =\ Predecesor\ del\ nodo\ i\ en\ un\ camino\ de\ aumentación.}} \\
\mbox{}\textit{\textcolor{Brown}{\ */}} \\
\mbox{}\textcolor{ForestGreen}{int}\ cap\textcolor{BrickRed}{[}MAXN\textcolor{BrickRed}{+}\textcolor{Purple}{1}\textcolor{BrickRed}{][}MAXN\textcolor{BrickRed}{+}\textcolor{Purple}{1}\textcolor{BrickRed}{],}\ prev\textcolor{BrickRed}{[}MAXN\textcolor{BrickRed}{+}\textcolor{Purple}{1}\textcolor{BrickRed}{];} \\
\mbox{} \\
\mbox{}vector\textcolor{BrickRed}{$<$}\textcolor{ForestGreen}{int}\textcolor{BrickRed}{$>$}\ g\textcolor{BrickRed}{[}MAXN\textcolor{BrickRed}{+}\textcolor{Purple}{1}\textcolor{BrickRed}{];}\ \textit{\textcolor{Brown}{//Vecinos\ de\ cada\ nodo.}} \\
\mbox{}\textbf{\textcolor{Blue}{inline}}\ \textcolor{ForestGreen}{void}\ \textbf{\textcolor{Black}{link}}\textcolor{BrickRed}{(}\textcolor{ForestGreen}{int}\ u\textcolor{BrickRed}{,}\ \textcolor{ForestGreen}{int}\ v\textcolor{BrickRed}{,}\ \textcolor{ForestGreen}{int}\ c\textcolor{BrickRed}{)}\textcolor{Red}{\{}\ cap\textcolor{BrickRed}{[}u\textcolor{BrickRed}{][}v\textcolor{BrickRed}{]}\ \textcolor{BrickRed}{=}\ c\textcolor{BrickRed}{;}\ g\textcolor{BrickRed}{[}u\textcolor{BrickRed}{].}\textbf{\textcolor{Black}{push$\_$back}}\textcolor{BrickRed}{(}v\textcolor{BrickRed}{),}\ g\textcolor{BrickRed}{[}v\textcolor{BrickRed}{].}\textbf{\textcolor{Black}{push$\_$back}}\textcolor{BrickRed}{(}u\textcolor{BrickRed}{);}\ \textcolor{Red}{\}} \\
\mbox{}\textit{\textcolor{Brown}{/*}} \\
\mbox{}\textit{\textcolor{Brown}{\ \ Notar\ que\ link\ crea\ las\ aristas\ (u,\ v)\ \&\&\ (v,\ u)\ en\ el\ grafo\ g.\ Esto\ es\ necesario}} \\
\mbox{}\textit{\textcolor{Brown}{\ \ porque\ el\ algoritmo\ de\ Edmonds-Karp\ necesita\ mirar\ el\ "{}back-edge"{}\ (j,\ i)\ que\ se\ crea}} \\
\mbox{}\textit{\textcolor{Brown}{\ \ al\ bombear\ flujo\ a\ través\ de\ (i,\ j).\ Sin\ embargo,\ no\ modifica\ cap[v][u],\ porque\ se}} \\
\mbox{}\textit{\textcolor{Brown}{\ \ asume\ que\ el\ grafo\ es\ dirigido.\ Si\ es\ no-dirigido,\ hacer\ cap[u][v]\ =\ cap[v][u]\ =\ c.}} \\
\mbox{}\textit{\textcolor{Brown}{\ */}} \\
\mbox{} \\
\mbox{} \\
\mbox{}\textit{\textcolor{Brown}{/*}} \\
\mbox{}\textit{\textcolor{Brown}{\ \ Método\ 1:}} \\
\mbox{} \\
\mbox{}\textit{\textcolor{Brown}{\ \ Mantener\ la\ red\ residual,\ donde\ residual[i][j]\ =\ cuánto\ flujo\ extra\ puedo\ inyectar}} \\
\mbox{}\textit{\textcolor{Brown}{\ \ a\ través\ de\ la\ arista\ (i,\ j).}} \\
\mbox{} \\
\mbox{}\textit{\textcolor{Brown}{\ \ Si\ empujo\ k\ unidades\ de\ i\ a\ j,\ entonces\ residual[i][j]\ -=\ k\ y\ residual[j][i]\ +=\ k}} \\
\mbox{}\textit{\textcolor{Brown}{\ \ (Puedo\ "{}desempujar"{}\ las\ k\ unidades\ de\ j\ a\ i).}} \\
\mbox{} \\
\mbox{}\textit{\textcolor{Brown}{\ \ Se\ puede\ modificar\ para\ que\ no\ utilice\ extra\ memoria\ en\ la\ tabla\ residual,\ sino}} \\
\mbox{}\textit{\textcolor{Brown}{\ \ que\ modifique\ directamente\ la\ tabla\ cap.}} \\
\mbox{}\textit{\textcolor{Brown}{*/}} \\
\mbox{} \\
\mbox{}\textcolor{ForestGreen}{int}\ residual\textcolor{BrickRed}{[}MAXN\textcolor{BrickRed}{+}\textcolor{Purple}{1}\textcolor{BrickRed}{][}MAXN\textcolor{BrickRed}{+}\textcolor{Purple}{1}\textcolor{BrickRed}{];} \\
\mbox{}\textcolor{ForestGreen}{int}\ \textbf{\textcolor{Black}{fordFulkerson}}\textcolor{BrickRed}{(}\textcolor{ForestGreen}{int}\ n\textcolor{BrickRed}{,}\ \textcolor{ForestGreen}{int}\ s\textcolor{BrickRed}{,}\ \textcolor{ForestGreen}{int}\ t\textcolor{BrickRed}{)}\textcolor{Red}{\{} \\
\mbox{}\ \ \textbf{\textcolor{Black}{memcpy}}\textcolor{BrickRed}{(}residual\textcolor{BrickRed}{,}\ cap\textcolor{BrickRed}{,}\ \textbf{\textcolor{Blue}{sizeof}}\ cap\textcolor{BrickRed}{);} \\
\mbox{} \\
\mbox{}\ \ \textcolor{ForestGreen}{int}\ ans\ \textcolor{BrickRed}{=}\ \textcolor{Purple}{0}\textcolor{BrickRed}{;} \\
\mbox{}\ \ \textbf{\textcolor{Blue}{while}}\ \textcolor{BrickRed}{(}\textbf{\textcolor{Blue}{true}}\textcolor{BrickRed}{)}\textcolor{Red}{\{} \\
\mbox{}\ \ \ \ \textbf{\textcolor{Black}{fill}}\textcolor{BrickRed}{(}prev\textcolor{BrickRed}{,}\ prev\textcolor{BrickRed}{+}n\textcolor{BrickRed}{,}\ \textcolor{BrickRed}{-}\textcolor{Purple}{1}\textcolor{BrickRed}{);} \\
\mbox{}\ \ \ \ queue\textcolor{BrickRed}{$<$}\textcolor{ForestGreen}{int}\textcolor{BrickRed}{$>$}\ q\textcolor{BrickRed}{;} \\
\mbox{}\ \ \ \ q\textcolor{BrickRed}{.}\textbf{\textcolor{Black}{push}}\textcolor{BrickRed}{(}s\textcolor{BrickRed}{);} \\
\mbox{}\ \ \ \ \textbf{\textcolor{Blue}{while}}\ \textcolor{BrickRed}{(}q\textcolor{BrickRed}{.}\textbf{\textcolor{Black}{size}}\textcolor{BrickRed}{()}\ \textcolor{BrickRed}{\&\&}\ prev\textcolor{BrickRed}{[}t\textcolor{BrickRed}{]}\ \textcolor{BrickRed}{==}\ \textcolor{BrickRed}{-}\textcolor{Purple}{1}\textcolor{BrickRed}{)}\textcolor{Red}{\{} \\
\mbox{}\ \ \ \ \ \ \textcolor{ForestGreen}{int}\ u\ \textcolor{BrickRed}{=}\ q\textcolor{BrickRed}{.}\textbf{\textcolor{Black}{front}}\textcolor{BrickRed}{();} \\
\mbox{}\ \ \ \ \ \ q\textcolor{BrickRed}{.}\textbf{\textcolor{Black}{pop}}\textcolor{BrickRed}{();} \\
\mbox{}\ \ \ \ \ \ vector\textcolor{BrickRed}{$<$}\textcolor{ForestGreen}{int}\textcolor{BrickRed}{$>$}\ \textcolor{BrickRed}{\&}out\ \textcolor{BrickRed}{=}\ g\textcolor{BrickRed}{[}u\textcolor{BrickRed}{];} \\
\mbox{}\ \ \ \ \ \ \textbf{\textcolor{Blue}{for}}\ \textcolor{BrickRed}{(}\textcolor{ForestGreen}{int}\ k\ \textcolor{BrickRed}{=}\ \textcolor{Purple}{0}\textcolor{BrickRed}{,}\ m\ \textcolor{BrickRed}{=}\ out\textcolor{BrickRed}{.}\textbf{\textcolor{Black}{size}}\textcolor{BrickRed}{();}\ k\textcolor{BrickRed}{$<$}m\textcolor{BrickRed}{;}\ \textcolor{BrickRed}{++}k\textcolor{BrickRed}{)}\textcolor{Red}{\{} \\
\mbox{}\ \ \ \ \ \ \ \ \textcolor{ForestGreen}{int}\ v\ \textcolor{BrickRed}{=}\ out\textcolor{BrickRed}{[}k\textcolor{BrickRed}{];} \\
\mbox{}\ \ \ \ \ \ \ \ \textbf{\textcolor{Blue}{if}}\ \textcolor{BrickRed}{(}v\ \textcolor{BrickRed}{!=}\ s\ \textcolor{BrickRed}{\&\&}\ prev\textcolor{BrickRed}{[}v\textcolor{BrickRed}{]}\ \textcolor{BrickRed}{==}\ \textcolor{BrickRed}{-}\textcolor{Purple}{1}\ \textcolor{BrickRed}{\&\&}\ residual\textcolor{BrickRed}{[}u\textcolor{BrickRed}{][}v\textcolor{BrickRed}{]}\ \textcolor{BrickRed}{$>$}\ \textcolor{Purple}{0}\textcolor{BrickRed}{)} \\
\mbox{}\ \ \ \ \ \ \ \ \ \ prev\textcolor{BrickRed}{[}v\textcolor{BrickRed}{]}\ \textcolor{BrickRed}{=}\ u\textcolor{BrickRed}{,}\ q\textcolor{BrickRed}{.}\textbf{\textcolor{Black}{push}}\textcolor{BrickRed}{(}v\textcolor{BrickRed}{);} \\
\mbox{}\ \ \ \ \ \ \textcolor{Red}{\}} \\
\mbox{}\ \ \ \ \textcolor{Red}{\}} \\
\mbox{} \\
\mbox{}\ \ \ \ \textbf{\textcolor{Blue}{if}}\ \textcolor{BrickRed}{(}prev\textcolor{BrickRed}{[}t\textcolor{BrickRed}{]}\ \textcolor{BrickRed}{==}\ \textcolor{BrickRed}{-}\textcolor{Purple}{1}\textcolor{BrickRed}{)}\ \textbf{\textcolor{Blue}{break}}\textcolor{BrickRed}{;} \\
\mbox{} \\
\mbox{}\ \ \ \ \textcolor{ForestGreen}{int}\ bottleneck\ \textcolor{BrickRed}{=}\ INT$\_$MAX\textcolor{BrickRed}{;} \\
\mbox{}\ \ \ \ \textbf{\textcolor{Blue}{for}}\ \textcolor{BrickRed}{(}\textcolor{ForestGreen}{int}\ v\ \textcolor{BrickRed}{=}\ t\textcolor{BrickRed}{,}\ u\ \textcolor{BrickRed}{=}\ prev\textcolor{BrickRed}{[}v\textcolor{BrickRed}{];}\ u\ \textcolor{BrickRed}{!=}\ \textcolor{BrickRed}{-}\textcolor{Purple}{1}\textcolor{BrickRed}{;}\ v\ \textcolor{BrickRed}{=}\ u\textcolor{BrickRed}{,}\ u\ \textcolor{BrickRed}{=}\ prev\textcolor{BrickRed}{[}v\textcolor{BrickRed}{])}\textcolor{Red}{\{} \\
\mbox{}\ \ \ \ \ \ bottleneck\ \textcolor{BrickRed}{=}\ \textbf{\textcolor{Black}{min}}\textcolor{BrickRed}{(}bottleneck\textcolor{BrickRed}{,}\ residual\textcolor{BrickRed}{[}u\textcolor{BrickRed}{][}v\textcolor{BrickRed}{]);} \\
\mbox{}\ \ \ \ \textcolor{Red}{\}} \\
\mbox{}\ \ \ \ \textbf{\textcolor{Blue}{for}}\ \textcolor{BrickRed}{(}\textcolor{ForestGreen}{int}\ v\ \textcolor{BrickRed}{=}\ t\textcolor{BrickRed}{,}\ u\ \textcolor{BrickRed}{=}\ prev\textcolor{BrickRed}{[}v\textcolor{BrickRed}{];}\ u\ \textcolor{BrickRed}{!=}\ \textcolor{BrickRed}{-}\textcolor{Purple}{1}\textcolor{BrickRed}{;}\ v\ \textcolor{BrickRed}{=}\ u\textcolor{BrickRed}{,}\ u\ \textcolor{BrickRed}{=}\ prev\textcolor{BrickRed}{[}v\textcolor{BrickRed}{])}\textcolor{Red}{\{} \\
\mbox{}\ \ \ \ \ \ residual\textcolor{BrickRed}{[}u\textcolor{BrickRed}{][}v\textcolor{BrickRed}{]}\ \textcolor{BrickRed}{-=}\ bottleneck\textcolor{BrickRed}{;} \\
\mbox{}\ \ \ \ \ \ residual\textcolor{BrickRed}{[}v\textcolor{BrickRed}{][}u\textcolor{BrickRed}{]}\ \textcolor{BrickRed}{+=}\ bottleneck\textcolor{BrickRed}{;} \\
\mbox{}\ \ \ \ \textcolor{Red}{\}} \\
\mbox{}\ \ \ \ ans\ \textcolor{BrickRed}{+=}\ bottleneck\textcolor{BrickRed}{;} \\
\mbox{}\ \ \textcolor{Red}{\}} \\
\mbox{}\ \ \textbf{\textcolor{Blue}{return}}\ ans\textcolor{BrickRed}{;} \\
\mbox{}\textcolor{Red}{\}} \\
\mbox{} \\
\mbox{} \\
\mbox{} \\
\mbox{}\textit{\textcolor{Brown}{/*}} \\
\mbox{}\textit{\textcolor{Brown}{\ \ Método\ 2:}} \\
\mbox{} \\
\mbox{}\textit{\textcolor{Brown}{\ \ Mantener\ la\ red\ de\ flujos,\ donde\ flow[i][j]\ =\ Flujo\ que,\ err,\ fluye}} \\
\mbox{}\textit{\textcolor{Brown}{\ \ de\ i\ a\ j.\ Notar\ que\ flow[i][j]\ puede\ ser\ negativo.\ Si\ esto\ pasa,}} \\
\mbox{}\textit{\textcolor{Brown}{\ \ es\ lo\ equivalente\ a\ decir\ que\ i\ "{}absorve"{}\ flujo\ de\ j,\ o\ lo\ que\ es}} \\
\mbox{}\textit{\textcolor{Brown}{\ \ lo\ mismo,\ que\ hay\ flujo\ positivo\ de\ j\ a\ i.}} \\
\mbox{} \\
\mbox{}\textit{\textcolor{Brown}{\ \ En\ cualquier\ momento\ se\ cumple\ la\ propiedad\ de\ skew\ symmetry,\ es\ decir,}} \\
\mbox{}\textit{\textcolor{Brown}{\ \ flow[i][j]\ =\ -flow[j][i].\ El\ flujo\ neto\ de\ i\ a\ j\ es\ entonces\ flow[i][j].}} \\
\mbox{} \\
\mbox{}\textit{\textcolor{Brown}{*/}} \\
\mbox{} \\
\mbox{}\textcolor{ForestGreen}{int}\ flow\textcolor{BrickRed}{[}MAXN\textcolor{BrickRed}{+}\textcolor{Purple}{1}\textcolor{BrickRed}{][}MAXN\textcolor{BrickRed}{+}\textcolor{Purple}{1}\textcolor{BrickRed}{];} \\
\mbox{}\textcolor{ForestGreen}{int}\ \textbf{\textcolor{Black}{fordFulkerson}}\textcolor{BrickRed}{(}\textcolor{ForestGreen}{int}\ n\textcolor{BrickRed}{,}\ \textcolor{ForestGreen}{int}\ s\textcolor{BrickRed}{,}\ \textcolor{ForestGreen}{int}\ t\textcolor{BrickRed}{)}\textcolor{Red}{\{} \\
\mbox{}\ \ \textit{\textcolor{Brown}{//memset(flow,\ 0,\ sizeof\ flow);}} \\
\mbox{}\ \ \textbf{\textcolor{Blue}{for}}\ \textcolor{BrickRed}{(}\textcolor{ForestGreen}{int}\ i\textcolor{BrickRed}{=}\textcolor{Purple}{0}\textcolor{BrickRed}{;}\ i\textcolor{BrickRed}{$<$}n\textcolor{BrickRed}{;}\ \textcolor{BrickRed}{++}i\textcolor{BrickRed}{)}\ \textbf{\textcolor{Black}{fill}}\textcolor{BrickRed}{(}flow\textcolor{BrickRed}{[}i\textcolor{BrickRed}{],}\ flow\textcolor{BrickRed}{[}i\textcolor{BrickRed}{]+}n\textcolor{BrickRed}{,}\ \textcolor{Purple}{0}\textcolor{BrickRed}{);} \\
\mbox{}\ \ \textcolor{ForestGreen}{int}\ ans\ \textcolor{BrickRed}{=}\ \textcolor{Purple}{0}\textcolor{BrickRed}{;} \\
\mbox{}\ \ \textbf{\textcolor{Blue}{while}}\ \textcolor{BrickRed}{(}\textbf{\textcolor{Blue}{true}}\textcolor{BrickRed}{)}\textcolor{Red}{\{} \\
\mbox{}\ \ \ \ \textbf{\textcolor{Black}{fill}}\textcolor{BrickRed}{(}prev\textcolor{BrickRed}{,}\ prev\textcolor{BrickRed}{+}n\textcolor{BrickRed}{,}\ \textcolor{BrickRed}{-}\textcolor{Purple}{1}\textcolor{BrickRed}{);} \\
\mbox{}\ \ \ \ queue\textcolor{BrickRed}{$<$}\textcolor{ForestGreen}{int}\textcolor{BrickRed}{$>$}\ q\textcolor{BrickRed}{;} \\
\mbox{}\ \ \ \ q\textcolor{BrickRed}{.}\textbf{\textcolor{Black}{push}}\textcolor{BrickRed}{(}s\textcolor{BrickRed}{);} \\
\mbox{}\ \ \ \ \textbf{\textcolor{Blue}{while}}\ \textcolor{BrickRed}{(}q\textcolor{BrickRed}{.}\textbf{\textcolor{Black}{size}}\textcolor{BrickRed}{()}\ \textcolor{BrickRed}{\&\&}\ prev\textcolor{BrickRed}{[}t\textcolor{BrickRed}{]}\ \textcolor{BrickRed}{==}\ \textcolor{BrickRed}{-}\textcolor{Purple}{1}\textcolor{BrickRed}{)}\textcolor{Red}{\{} \\
\mbox{}\ \ \ \ \ \ \textcolor{ForestGreen}{int}\ u\ \textcolor{BrickRed}{=}\ q\textcolor{BrickRed}{.}\textbf{\textcolor{Black}{front}}\textcolor{BrickRed}{();} \\
\mbox{}\ \ \ \ \ \ q\textcolor{BrickRed}{.}\textbf{\textcolor{Black}{pop}}\textcolor{BrickRed}{();} \\
\mbox{}\ \ \ \ \ \ vector\textcolor{BrickRed}{$<$}\textcolor{ForestGreen}{int}\textcolor{BrickRed}{$>$}\ \textcolor{BrickRed}{\&}out\ \textcolor{BrickRed}{=}\ g\textcolor{BrickRed}{[}u\textcolor{BrickRed}{];} \\
\mbox{}\ \ \ \ \ \ \textbf{\textcolor{Blue}{for}}\ \textcolor{BrickRed}{(}\textcolor{ForestGreen}{int}\ k\ \textcolor{BrickRed}{=}\ \textcolor{Purple}{0}\textcolor{BrickRed}{,}\ m\ \textcolor{BrickRed}{=}\ out\textcolor{BrickRed}{.}\textbf{\textcolor{Black}{size}}\textcolor{BrickRed}{();}\ k\textcolor{BrickRed}{$<$}m\textcolor{BrickRed}{;}\ \textcolor{BrickRed}{++}k\textcolor{BrickRed}{)}\textcolor{Red}{\{} \\
\mbox{}\ \ \ \ \ \ \ \ \textcolor{ForestGreen}{int}\ v\ \textcolor{BrickRed}{=}\ out\textcolor{BrickRed}{[}k\textcolor{BrickRed}{];} \\
\mbox{}\ \ \ \ \ \ \ \ \textbf{\textcolor{Blue}{if}}\ \textcolor{BrickRed}{(}v\ \textcolor{BrickRed}{!=}\ s\ \textcolor{BrickRed}{\&\&}\ prev\textcolor{BrickRed}{[}v\textcolor{BrickRed}{]}\ \textcolor{BrickRed}{==}\ \textcolor{BrickRed}{-}\textcolor{Purple}{1}\ \textcolor{BrickRed}{\&\&}\ cap\textcolor{BrickRed}{[}u\textcolor{BrickRed}{][}v\textcolor{BrickRed}{]}\ \textcolor{BrickRed}{$>$}\ flow\textcolor{BrickRed}{[}u\textcolor{BrickRed}{][}v\textcolor{BrickRed}{])} \\
\mbox{}\ \ \ \ \ \ \ \ \ \ prev\textcolor{BrickRed}{[}v\textcolor{BrickRed}{]}\ \textcolor{BrickRed}{=}\ u\textcolor{BrickRed}{,}\ q\textcolor{BrickRed}{.}\textbf{\textcolor{Black}{push}}\textcolor{BrickRed}{(}v\textcolor{BrickRed}{);} \\
\mbox{}\ \ \ \ \ \ \textcolor{Red}{\}} \\
\mbox{}\ \ \ \ \textcolor{Red}{\}} \\
\mbox{} \\
\mbox{}\ \ \ \ \textbf{\textcolor{Blue}{if}}\ \textcolor{BrickRed}{(}prev\textcolor{BrickRed}{[}t\textcolor{BrickRed}{]}\ \textcolor{BrickRed}{==}\ \textcolor{BrickRed}{-}\textcolor{Purple}{1}\textcolor{BrickRed}{)}\ \textbf{\textcolor{Blue}{break}}\textcolor{BrickRed}{;} \\
\mbox{} \\
\mbox{}\ \ \ \ \textcolor{ForestGreen}{int}\ bottleneck\ \textcolor{BrickRed}{=}\ INT$\_$MAX\textcolor{BrickRed}{;} \\
\mbox{}\ \ \ \ \textbf{\textcolor{Blue}{for}}\ \textcolor{BrickRed}{(}\textcolor{ForestGreen}{int}\ v\ \textcolor{BrickRed}{=}\ t\textcolor{BrickRed}{,}\ u\ \textcolor{BrickRed}{=}\ prev\textcolor{BrickRed}{[}v\textcolor{BrickRed}{];}\ u\ \textcolor{BrickRed}{!=}\ \textcolor{BrickRed}{-}\textcolor{Purple}{1}\textcolor{BrickRed}{;}\ v\ \textcolor{BrickRed}{=}\ u\textcolor{BrickRed}{,}\ u\ \textcolor{BrickRed}{=}\ prev\textcolor{BrickRed}{[}v\textcolor{BrickRed}{])}\textcolor{Red}{\{} \\
\mbox{}\ \ \ \ \ \ bottleneck\ \textcolor{BrickRed}{=}\ \textbf{\textcolor{Black}{min}}\textcolor{BrickRed}{(}bottleneck\textcolor{BrickRed}{,}\ cap\textcolor{BrickRed}{[}u\textcolor{BrickRed}{][}v\textcolor{BrickRed}{]}\ \textcolor{BrickRed}{-}\ flow\textcolor{BrickRed}{[}u\textcolor{BrickRed}{][}v\textcolor{BrickRed}{]);} \\
\mbox{}\ \ \ \ \textcolor{Red}{\}} \\
\mbox{}\ \ \ \ \textbf{\textcolor{Blue}{for}}\ \textcolor{BrickRed}{(}\textcolor{ForestGreen}{int}\ v\ \textcolor{BrickRed}{=}\ t\textcolor{BrickRed}{,}\ u\ \textcolor{BrickRed}{=}\ prev\textcolor{BrickRed}{[}v\textcolor{BrickRed}{];}\ u\ \textcolor{BrickRed}{!=}\ \textcolor{BrickRed}{-}\textcolor{Purple}{1}\textcolor{BrickRed}{;}\ v\ \textcolor{BrickRed}{=}\ u\textcolor{BrickRed}{,}\ u\ \textcolor{BrickRed}{=}\ prev\textcolor{BrickRed}{[}v\textcolor{BrickRed}{])}\textcolor{Red}{\{} \\
\mbox{}\ \ \ \ \ \ flow\textcolor{BrickRed}{[}u\textcolor{BrickRed}{][}v\textcolor{BrickRed}{]}\ \textcolor{BrickRed}{+=}\ bottleneck\textcolor{BrickRed}{;} \\
\mbox{}\ \ \ \ \ \ flow\textcolor{BrickRed}{[}v\textcolor{BrickRed}{][}u\textcolor{BrickRed}{]}\ \textcolor{BrickRed}{=}\ \textcolor{BrickRed}{-}flow\textcolor{BrickRed}{[}u\textcolor{BrickRed}{][}v\textcolor{BrickRed}{];} \\
\mbox{}\ \ \ \ \textcolor{Red}{\}} \\
\mbox{}\ \ \ \ ans\ \textcolor{BrickRed}{+=}\ bottleneck\textcolor{BrickRed}{;} \\
\mbox{}\ \ \textcolor{Red}{\}} \\
\mbox{}\ \ \textbf{\textcolor{Blue}{return}}\ ans\textcolor{BrickRed}{;} \\
\mbox{}\textcolor{Red}{\}} \\
\mbox{} \\

} \normalfont\normalsize
%.tex

\section{Programación dinámica}
\subsection{Longest common subsequence}
% Generator: GNU source-highlight, by Lorenzo Bettini, http://www.gnu.org/software/src-highlite

{\ttfamily \raggedright {
\noindent
\mbox{}\textbf{\textcolor{RoyalBlue}{\#define}}\ \textbf{\textcolor{Black}{MAX}}\textcolor{BrickRed}{(}a\textcolor{BrickRed}{,}b\textcolor{BrickRed}{)}\ \textcolor{BrickRed}{((}a\textcolor{BrickRed}{$>$}b\textcolor{BrickRed}{)?(}a\textcolor{BrickRed}{):(}b\textcolor{BrickRed}{))} \\
\mbox{}\textcolor{ForestGreen}{int}\ dp\textcolor{BrickRed}{[}\textcolor{Purple}{1001}\textcolor{BrickRed}{][}\textcolor{Purple}{1001}\textcolor{BrickRed}{];} \\
\mbox{} \\
\mbox{}\textcolor{ForestGreen}{int}\ \textbf{\textcolor{Black}{lcs}}\textcolor{BrickRed}{(}\textbf{\textcolor{Blue}{const}}\ string\ \textcolor{BrickRed}{\&}s\textcolor{BrickRed}{,}\ \textbf{\textcolor{Blue}{const}}\ string\ \textcolor{BrickRed}{\&}t\textcolor{BrickRed}{)}\textcolor{Red}{\{} \\
\mbox{}\ \ \textcolor{ForestGreen}{int}\ m\ \textcolor{BrickRed}{=}\ s\textcolor{BrickRed}{.}\textbf{\textcolor{Black}{size}}\textcolor{BrickRed}{(),}\ n\ \textcolor{BrickRed}{=}\ t\textcolor{BrickRed}{.}\textbf{\textcolor{Black}{size}}\textcolor{BrickRed}{();} \\
\mbox{}\ \ \textbf{\textcolor{Blue}{if}}\ \textcolor{BrickRed}{(}m\ \textcolor{BrickRed}{==}\ \textcolor{Purple}{0}\ \textcolor{BrickRed}{$|$$|$}\ n\ \textcolor{BrickRed}{==}\ \textcolor{Purple}{0}\textcolor{BrickRed}{)}\ \textbf{\textcolor{Blue}{return}}\ \textcolor{Purple}{0}\textcolor{BrickRed}{;} \\
\mbox{}\ \ \textbf{\textcolor{Blue}{for}}\ \textcolor{BrickRed}{(}\textcolor{ForestGreen}{int}\ i\textcolor{BrickRed}{=}\textcolor{Purple}{0}\textcolor{BrickRed}{;}\ i\textcolor{BrickRed}{$<$=}m\textcolor{BrickRed}{;}\ \textcolor{BrickRed}{++}i\textcolor{BrickRed}{)} \\
\mbox{}\ \ \ \ dp\textcolor{BrickRed}{[}i\textcolor{BrickRed}{][}\textcolor{Purple}{0}\textcolor{BrickRed}{]}\ \textcolor{BrickRed}{=}\ \textcolor{Purple}{0}\textcolor{BrickRed}{;} \\
\mbox{}\ \ \textbf{\textcolor{Blue}{for}}\ \textcolor{BrickRed}{(}\textcolor{ForestGreen}{int}\ j\textcolor{BrickRed}{=}\textcolor{Purple}{1}\textcolor{BrickRed}{;}\ j\textcolor{BrickRed}{$<$=}n\textcolor{BrickRed}{;}\ \textcolor{BrickRed}{++}j\textcolor{BrickRed}{)} \\
\mbox{}\ \ \ \ dp\textcolor{BrickRed}{[}\textcolor{Purple}{0}\textcolor{BrickRed}{][}j\textcolor{BrickRed}{]}\ \textcolor{BrickRed}{=}\ \textcolor{Purple}{0}\textcolor{BrickRed}{;} \\
\mbox{}\ \ \textbf{\textcolor{Blue}{for}}\ \textcolor{BrickRed}{(}\textcolor{ForestGreen}{int}\ i\textcolor{BrickRed}{=}\textcolor{Purple}{0}\textcolor{BrickRed}{;}\ i\textcolor{BrickRed}{$<$}m\textcolor{BrickRed}{;}\ \textcolor{BrickRed}{++}i\textcolor{BrickRed}{)} \\
\mbox{}\ \ \ \ \textbf{\textcolor{Blue}{for}}\ \textcolor{BrickRed}{(}\textcolor{ForestGreen}{int}\ j\textcolor{BrickRed}{=}\textcolor{Purple}{0}\textcolor{BrickRed}{;}\ j\textcolor{BrickRed}{$<$}n\textcolor{BrickRed}{;}\ \textcolor{BrickRed}{++}j\textcolor{BrickRed}{)} \\
\mbox{}\ \ \ \ \ \ \textbf{\textcolor{Blue}{if}}\ \textcolor{BrickRed}{(}s\textcolor{BrickRed}{[}i\textcolor{BrickRed}{]}\ \textcolor{BrickRed}{==}\ t\textcolor{BrickRed}{[}j\textcolor{BrickRed}{])} \\
\mbox{}\ \ \ \ \ \ \ \ dp\textcolor{BrickRed}{[}i\textcolor{BrickRed}{+}\textcolor{Purple}{1}\textcolor{BrickRed}{][}j\textcolor{BrickRed}{+}\textcolor{Purple}{1}\textcolor{BrickRed}{]}\ \textcolor{BrickRed}{=}\ dp\textcolor{BrickRed}{[}i\textcolor{BrickRed}{][}j\textcolor{BrickRed}{]+}\textcolor{Purple}{1}\textcolor{BrickRed}{;} \\
\mbox{}\ \ \ \ \ \ \textbf{\textcolor{Blue}{else}} \\
\mbox{}\ \ \ \ \ \ \ \ dp\textcolor{BrickRed}{[}i\textcolor{BrickRed}{+}\textcolor{Purple}{1}\textcolor{BrickRed}{][}j\textcolor{BrickRed}{+}\textcolor{Purple}{1}\textcolor{BrickRed}{]}\ \textcolor{BrickRed}{=}\ \textbf{\textcolor{Black}{MAX}}\textcolor{BrickRed}{(}dp\textcolor{BrickRed}{[}i\textcolor{BrickRed}{+}\textcolor{Purple}{1}\textcolor{BrickRed}{][}j\textcolor{BrickRed}{],}\ dp\textcolor{BrickRed}{[}i\textcolor{BrickRed}{][}j\textcolor{BrickRed}{+}\textcolor{Purple}{1}\textcolor{BrickRed}{]);} \\
\mbox{}\ \ \textbf{\textcolor{Blue}{return}}\ dp\textcolor{BrickRed}{[}m\textcolor{BrickRed}{][}n\textcolor{BrickRed}{];} \\
\mbox{}\textcolor{Red}{\}} \\

} \normalfont\normalsize
%.tex

\subsection{Partición de troncos}
Este problema es similar al problema de \textit{Matrix Chain Multiplication}. Se tiene
un tronco de longitud $n$, y $m$ puntos de corte en el tronco. Se puede hacer un corte a la vez,
cuyo costo es igual a la longitud del tronco. ¿Cuál es el mínimo costo para partir todo el tronco
en pedacitos individuales?
\\
\medskip
\textbf{Ejemplo:} Se tiene un tronco de longitud 10. Los puntos de corte son 2, 4, y 7. El mínimo
costo para partirlo es 20, y se obtiene así:
\begin{itemize}
\item Partir el tronco (0, 10) por 4. Vale 10 y quedan los troncos (0, 4) y (4, 10).
\item Partir el tronco (0, 4) por 2. Vale 4 y quedan los troncos (0, 2), (2, 4) y (4, 10).
\item No hay que partir el tronco (0, 2).
\item No hay que partir el tronco (2, 4).
\item Partir el tronco (4, 10) por 7. Vale 6 y quedan los troncos (4, 7) y (7, 10).
\item No hay que partir el tronco (4, 7).
\item No hay que partir el tronco (7, 10).
\item El costo total es $10+4+6 = 20$.
\end{itemize}

\medskip
El algoritmo es $O(n^3)$, pero optimizable a $O(n^2)$ con una tabla adicional:
% Generator: GNU source-highlight, by Lorenzo Bettini, http://www.gnu.org/software/src-highlite

{\ttfamily \raggedright {
\noindent
\mbox{}\textit{\textcolor{Brown}{/*}} \\
\mbox{}\textit{\textcolor{Brown}{\ \ O(n\textasciicircum{}3)}} \\
\mbox{} \\
\mbox{}\textit{\textcolor{Brown}{\ dp[i][j]\ =\ Mínimo\ costo\ de\ partir\ la\ cadena\ entre\ las\ particiones\ i\ e\ j,\ ambas\ incluídas.}} \\
\mbox{}\textit{\textcolor{Brown}{\ */}} \\
\mbox{}\textcolor{ForestGreen}{int}\ dp\textcolor{BrickRed}{[}\textcolor{Purple}{1005}\textcolor{BrickRed}{][}\textcolor{Purple}{1005}\textcolor{BrickRed}{];} \\
\mbox{}\textcolor{ForestGreen}{int}\ p\textcolor{BrickRed}{[}\textcolor{Purple}{1005}\textcolor{BrickRed}{];} \\
\mbox{} \\
\mbox{}\textcolor{ForestGreen}{int}\ \textbf{\textcolor{Black}{cubic}}\textcolor{BrickRed}{()}\textcolor{Red}{\{} \\
\mbox{}\ \ \textcolor{ForestGreen}{int}\ n\textcolor{BrickRed}{,}\ m\textcolor{BrickRed}{;} \\
\mbox{}\ \ \textbf{\textcolor{Blue}{while}}\ \textcolor{BrickRed}{(}\textbf{\textcolor{Black}{scanf}}\textcolor{BrickRed}{(}\texttt{\textcolor{Red}{"{}\%d\ \%d"{}}}\textcolor{BrickRed}{,}\ \textcolor{BrickRed}{\&}n\textcolor{BrickRed}{,}\ \textcolor{BrickRed}{\&}m\textcolor{BrickRed}{)==}\textcolor{Purple}{2}\textcolor{BrickRed}{)}\textcolor{Red}{\{} \\
\mbox{}\ \ \ \ p\textcolor{BrickRed}{[}\textcolor{Purple}{0}\textcolor{BrickRed}{]}\ \textcolor{BrickRed}{=}\ \textcolor{Purple}{0}\textcolor{BrickRed}{;} \\
\mbox{}\ \ \ \ \textbf{\textcolor{Blue}{for}}\ \textcolor{BrickRed}{(}\textcolor{ForestGreen}{int}\ i\textcolor{BrickRed}{=}\textcolor{Purple}{1}\textcolor{BrickRed}{;}\ i\textcolor{BrickRed}{$<$=}m\textcolor{BrickRed}{;}\ \textcolor{BrickRed}{++}i\textcolor{BrickRed}{)}\textcolor{Red}{\{} \\
\mbox{}\ \ \ \ \ \ \textbf{\textcolor{Black}{scanf}}\textcolor{BrickRed}{(}\texttt{\textcolor{Red}{"{}\%d"{}}}\textcolor{BrickRed}{,}\ \textcolor{BrickRed}{\&}p\textcolor{BrickRed}{[}i\textcolor{BrickRed}{]);} \\
\mbox{}\ \ \ \ \textcolor{Red}{\}} \\
\mbox{}\ \ \ \ p\textcolor{BrickRed}{[}m\textcolor{BrickRed}{+}\textcolor{Purple}{1}\textcolor{BrickRed}{]}\ \textcolor{BrickRed}{=}\ n\textcolor{BrickRed}{;} \\
\mbox{}\ \ \ \ m\ \textcolor{BrickRed}{+=}\ \textcolor{Purple}{2}\textcolor{BrickRed}{;} \\
\mbox{} \\
\mbox{}\ \ \ \ \textbf{\textcolor{Blue}{for}}\ \textcolor{BrickRed}{(}\textcolor{ForestGreen}{int}\ i\textcolor{BrickRed}{=}\textcolor{Purple}{0}\textcolor{BrickRed}{;}\ i\textcolor{BrickRed}{$<$}m\textcolor{BrickRed}{;}\ \textcolor{BrickRed}{++}i\textcolor{BrickRed}{)}\textcolor{Red}{\{} \\
\mbox{}\ \ \ \ \ \ dp\textcolor{BrickRed}{[}i\textcolor{BrickRed}{][}i\textcolor{BrickRed}{+}\textcolor{Purple}{1}\textcolor{BrickRed}{]}\ \textcolor{BrickRed}{=}\ \textcolor{Purple}{0}\textcolor{BrickRed}{;} \\
\mbox{}\ \ \ \ \textcolor{Red}{\}} \\
\mbox{}\ \ \ \  \\
\mbox{}\ \ \ \ \textbf{\textcolor{Blue}{for}}\ \textcolor{BrickRed}{(}\textcolor{ForestGreen}{int}\ i\textcolor{BrickRed}{=}m\textcolor{BrickRed}{-}\textcolor{Purple}{2}\textcolor{BrickRed}{;}\ i\textcolor{BrickRed}{$>$=}\textcolor{Purple}{0}\textcolor{BrickRed}{;}\ \textcolor{BrickRed}{-\/-}i\textcolor{BrickRed}{)}\textcolor{Red}{\{} \\
\mbox{}\ \ \ \ \ \ \textbf{\textcolor{Blue}{for}}\ \textcolor{BrickRed}{(}\textcolor{ForestGreen}{int}\ j\textcolor{BrickRed}{=}i\textcolor{BrickRed}{+}\textcolor{Purple}{2}\textcolor{BrickRed}{;}\ j\textcolor{BrickRed}{$<$}m\textcolor{BrickRed}{;}\ \textcolor{BrickRed}{++}j\textcolor{BrickRed}{)}\textcolor{Red}{\{} \\
\mbox{}\ \ \ \ \ \ \ \ dp\textcolor{BrickRed}{[}i\textcolor{BrickRed}{][}j\textcolor{BrickRed}{]}\ \textcolor{BrickRed}{=}\ p\textcolor{BrickRed}{[}j\textcolor{BrickRed}{]-}p\textcolor{BrickRed}{[}i\textcolor{BrickRed}{];} \\
\mbox{}\ \ \ \ \ \ \ \ \textcolor{ForestGreen}{int}\ t\ \textcolor{BrickRed}{=}\ INT$\_$MAX\textcolor{BrickRed}{;} \\
\mbox{}\ \ \ \ \ \ \ \ \textbf{\textcolor{Blue}{for}}\ \textcolor{BrickRed}{(}\textcolor{ForestGreen}{int}\ k\textcolor{BrickRed}{=}i\textcolor{BrickRed}{+}\textcolor{Purple}{1}\textcolor{BrickRed}{;}\ k\textcolor{BrickRed}{$<$}j\textcolor{BrickRed}{;}\ \textcolor{BrickRed}{++}k\textcolor{BrickRed}{)}\textcolor{Red}{\{} \\
\mbox{}\ \ \ \ \ \ \ \ \ \ t\ \textcolor{BrickRed}{=}\ \textbf{\textcolor{Black}{min}}\textcolor{BrickRed}{(}t\textcolor{BrickRed}{,}\ dp\textcolor{BrickRed}{[}i\textcolor{BrickRed}{][}k\textcolor{BrickRed}{]}\ \textcolor{BrickRed}{+}\ dp\textcolor{BrickRed}{[}k\textcolor{BrickRed}{][}j\textcolor{BrickRed}{]);} \\
\mbox{}\ \ \ \ \ \ \ \ \textcolor{Red}{\}} \\
\mbox{}\ \ \ \ \ \ \ \ dp\textcolor{BrickRed}{[}i\textcolor{BrickRed}{][}j\textcolor{BrickRed}{]}\ \textcolor{BrickRed}{+=}\ t\textcolor{BrickRed}{;} \\
\mbox{}\ \ \ \ \ \ \textcolor{Red}{\}} \\
\mbox{}\ \ \ \ \textcolor{Red}{\}} \\
\mbox{} \\
\mbox{}\ \ \ \ \textbf{\textcolor{Black}{printf}}\textcolor{BrickRed}{(}\texttt{\textcolor{Red}{"{}\%d}}\texttt{\textcolor{CarnationPink}{\textbackslash{}n}}\texttt{\textcolor{Red}{"{}}}\textcolor{BrickRed}{,}\ dp\textcolor{BrickRed}{[}\textcolor{Purple}{0}\textcolor{BrickRed}{][}m\textcolor{BrickRed}{-}\textcolor{Purple}{1}\textcolor{BrickRed}{]);} \\
\mbox{}\ \ \textcolor{Red}{\}} \\
\mbox{}\ \ \textbf{\textcolor{Blue}{return}}\ \textcolor{Purple}{0}\textcolor{BrickRed}{;} \\
\mbox{}\textcolor{Red}{\}} \\
\mbox{} \\
\mbox{} \\
\mbox{}\textit{\textcolor{Brown}{/*}} \\
\mbox{}\textit{\textcolor{Brown}{\ \ O(n\textasciicircum{}2)}} \\
\mbox{} \\
\mbox{}\textit{\textcolor{Brown}{\ \ dp[i][j]\ =\ Mínimo\ costo\ de\ partir\ la\ cadena\ entre\ las\ particiones\ i\ e\ j,\ ambas\ incluídas.}} \\
\mbox{}\textit{\textcolor{Brown}{\ \ pivot[i][j]\ =\ Índice\ de\ la\ partición\ que\ usé\ para\ lograr\ dp[i][j].}} \\
\mbox{}\textit{\textcolor{Brown}{\ */}} \\
\mbox{}\textcolor{ForestGreen}{int}\ dp\textcolor{BrickRed}{[}\textcolor{Purple}{1005}\textcolor{BrickRed}{][}\textcolor{Purple}{1005}\textcolor{BrickRed}{],}\ pivot\textcolor{BrickRed}{[}\textcolor{Purple}{1005}\textcolor{BrickRed}{][}\textcolor{Purple}{1005}\textcolor{BrickRed}{];} \\
\mbox{}\textcolor{ForestGreen}{int}\ p\textcolor{BrickRed}{[}\textcolor{Purple}{1005}\textcolor{BrickRed}{];} \\
\mbox{} \\
\mbox{}\textcolor{ForestGreen}{int}\ \textbf{\textcolor{Black}{quadratic}}\textcolor{BrickRed}{()}\textcolor{Red}{\{} \\
\mbox{}\ \ \textcolor{ForestGreen}{int}\ n\textcolor{BrickRed}{,}\ m\textcolor{BrickRed}{;} \\
\mbox{}\ \ \textbf{\textcolor{Blue}{while}}\ \textcolor{BrickRed}{(}\textbf{\textcolor{Black}{scanf}}\textcolor{BrickRed}{(}\texttt{\textcolor{Red}{"{}\%d\ \%d"{}}}\textcolor{BrickRed}{,}\ \textcolor{BrickRed}{\&}n\textcolor{BrickRed}{,}\ \textcolor{BrickRed}{\&}m\textcolor{BrickRed}{)==}\textcolor{Purple}{2}\textcolor{BrickRed}{)}\textcolor{Red}{\{} \\
\mbox{}\ \ \ \ p\textcolor{BrickRed}{[}\textcolor{Purple}{0}\textcolor{BrickRed}{]}\ \textcolor{BrickRed}{=}\ \textcolor{Purple}{0}\textcolor{BrickRed}{;} \\
\mbox{}\ \ \ \ \textbf{\textcolor{Blue}{for}}\ \textcolor{BrickRed}{(}\textcolor{ForestGreen}{int}\ i\textcolor{BrickRed}{=}\textcolor{Purple}{1}\textcolor{BrickRed}{;}\ i\textcolor{BrickRed}{$<$=}m\textcolor{BrickRed}{;}\ \textcolor{BrickRed}{++}i\textcolor{BrickRed}{)}\textcolor{Red}{\{} \\
\mbox{}\ \ \ \ \ \ \textbf{\textcolor{Black}{scanf}}\textcolor{BrickRed}{(}\texttt{\textcolor{Red}{"{}\%d"{}}}\textcolor{BrickRed}{,}\ \textcolor{BrickRed}{\&}p\textcolor{BrickRed}{[}i\textcolor{BrickRed}{]);} \\
\mbox{}\ \ \ \ \textcolor{Red}{\}} \\
\mbox{}\ \ \ \ p\textcolor{BrickRed}{[}m\textcolor{BrickRed}{+}\textcolor{Purple}{1}\textcolor{BrickRed}{]}\ \textcolor{BrickRed}{=}\ n\textcolor{BrickRed}{;} \\
\mbox{}\ \ \ \ m\ \textcolor{BrickRed}{+=}\ \textcolor{Purple}{2}\textcolor{BrickRed}{;} \\
\mbox{} \\
\mbox{}\ \ \ \ \textbf{\textcolor{Blue}{for}}\ \textcolor{BrickRed}{(}\textcolor{ForestGreen}{int}\ i\textcolor{BrickRed}{=}\textcolor{Purple}{0}\textcolor{BrickRed}{;}\ i\textcolor{BrickRed}{$<$}m\textcolor{BrickRed}{-}\textcolor{Purple}{1}\textcolor{BrickRed}{;}\ \textcolor{BrickRed}{++}i\textcolor{BrickRed}{)}\textcolor{Red}{\{} \\
\mbox{}\ \ \ \ \ \ dp\textcolor{BrickRed}{[}i\textcolor{BrickRed}{][}i\textcolor{BrickRed}{+}\textcolor{Purple}{1}\textcolor{BrickRed}{]}\ \textcolor{BrickRed}{=}\ \textcolor{Purple}{0}\textcolor{BrickRed}{;} \\
\mbox{}\ \ \ \ \textcolor{Red}{\}} \\
\mbox{}\ \ \ \ \textbf{\textcolor{Blue}{for}}\ \textcolor{BrickRed}{(}\textcolor{ForestGreen}{int}\ i\textcolor{BrickRed}{=}\textcolor{Purple}{0}\textcolor{BrickRed}{;}\ i\textcolor{BrickRed}{$<$}m\textcolor{BrickRed}{-}\textcolor{Purple}{2}\textcolor{BrickRed}{;}\ \textcolor{BrickRed}{++}i\textcolor{BrickRed}{)}\textcolor{Red}{\{} \\
\mbox{}\ \ \ \ \ \ dp\textcolor{BrickRed}{[}i\textcolor{BrickRed}{][}i\textcolor{BrickRed}{+}\textcolor{Purple}{2}\textcolor{BrickRed}{]}\ \textcolor{BrickRed}{=}\ p\textcolor{BrickRed}{[}i\textcolor{BrickRed}{+}\textcolor{Purple}{2}\textcolor{BrickRed}{]}\ \textcolor{BrickRed}{-}\ p\textcolor{BrickRed}{[}i\textcolor{BrickRed}{];} \\
\mbox{}\ \ \ \ \ \ pivot\textcolor{BrickRed}{[}i\textcolor{BrickRed}{][}i\textcolor{BrickRed}{+}\textcolor{Purple}{2}\textcolor{BrickRed}{]}\ \textcolor{BrickRed}{=}\ i\textcolor{BrickRed}{+}\textcolor{Purple}{1}\textcolor{BrickRed}{;} \\
\mbox{}\ \ \ \ \textcolor{Red}{\}} \\
\mbox{}\ \ \ \  \\
\mbox{}\ \ \ \ \textbf{\textcolor{Blue}{for}}\ \textcolor{BrickRed}{(}\textcolor{ForestGreen}{int}\ d\textcolor{BrickRed}{=}\textcolor{Purple}{3}\textcolor{BrickRed}{;}\ d\textcolor{BrickRed}{$<$}m\textcolor{BrickRed}{;}\ \textcolor{BrickRed}{++}d\textcolor{BrickRed}{)}\textcolor{Red}{\{}\ \textit{\textcolor{Brown}{//d\ =\ longitud}} \\
\mbox{}\ \ \ \ \ \ \textbf{\textcolor{Blue}{for}}\ \textcolor{BrickRed}{(}\textcolor{ForestGreen}{int}\ j\textcolor{BrickRed}{,}\ i\textcolor{BrickRed}{=}\textcolor{Purple}{0}\textcolor{BrickRed}{;}\ \textcolor{BrickRed}{(}j\ \textcolor{BrickRed}{=}\ i\ \textcolor{BrickRed}{+}\ d\textcolor{BrickRed}{)}\ \textcolor{BrickRed}{$<$}\ m\textcolor{BrickRed}{;}\ \textcolor{BrickRed}{++}i\textcolor{BrickRed}{)}\textcolor{Red}{\{} \\
\mbox{}\ \ \ \ \ \ \ \ dp\textcolor{BrickRed}{[}i\textcolor{BrickRed}{][}j\textcolor{BrickRed}{]}\ \textcolor{BrickRed}{=}\ p\textcolor{BrickRed}{[}j\textcolor{BrickRed}{]}\ \textcolor{BrickRed}{-}\ p\textcolor{BrickRed}{[}i\textcolor{BrickRed}{];} \\
\mbox{}\ \ \ \ \ \ \ \ \textcolor{ForestGreen}{int}\ t\ \textcolor{BrickRed}{=}\ INT$\_$MAX\textcolor{BrickRed}{,}\ s\textcolor{BrickRed}{;} \\
\mbox{}\ \ \ \ \ \ \ \ \textbf{\textcolor{Blue}{for}}\ \textcolor{BrickRed}{(}\textcolor{ForestGreen}{int}\ k\textcolor{BrickRed}{=}pivot\textcolor{BrickRed}{[}i\textcolor{BrickRed}{][}j\textcolor{BrickRed}{-}\textcolor{Purple}{1}\textcolor{BrickRed}{];}\ k\textcolor{BrickRed}{$<$=}pivot\textcolor{BrickRed}{[}i\textcolor{BrickRed}{+}\textcolor{Purple}{1}\textcolor{BrickRed}{][}j\textcolor{BrickRed}{];}\ \textcolor{BrickRed}{++}k\textcolor{BrickRed}{)}\textcolor{Red}{\{} \\
\mbox{}\ \ \ \ \ \ \ \ \ \ \textcolor{ForestGreen}{int}\ x\ \textcolor{BrickRed}{=}\ dp\textcolor{BrickRed}{[}i\textcolor{BrickRed}{][}k\textcolor{BrickRed}{]}\ \textcolor{BrickRed}{+}\ dp\textcolor{BrickRed}{[}k\textcolor{BrickRed}{][}j\textcolor{BrickRed}{];} \\
\mbox{}\ \ \ \ \ \ \ \ \ \ \textbf{\textcolor{Blue}{if}}\ \textcolor{BrickRed}{(}x\ \textcolor{BrickRed}{$<$}\ t\textcolor{BrickRed}{)}\ t\ \textcolor{BrickRed}{=}\ x\textcolor{BrickRed}{,}\ s\ \textcolor{BrickRed}{=}\ k\textcolor{BrickRed}{;} \\
\mbox{}\ \ \ \ \ \ \ \ \textcolor{Red}{\}} \\
\mbox{}\ \ \ \ \ \ \ \ dp\textcolor{BrickRed}{[}i\textcolor{BrickRed}{][}j\textcolor{BrickRed}{]}\ \textcolor{BrickRed}{+=}\ t\textcolor{BrickRed}{,}\ pivot\textcolor{BrickRed}{[}i\textcolor{BrickRed}{][}j\textcolor{BrickRed}{]}\ \textcolor{BrickRed}{=}\ s\textcolor{BrickRed}{;} \\
\mbox{}\ \ \ \ \ \ \textcolor{Red}{\}} \\
\mbox{}\ \ \ \ \textcolor{Red}{\}} \\
\mbox{} \\
\mbox{}\ \ \ \ \textbf{\textcolor{Black}{printf}}\textcolor{BrickRed}{(}\texttt{\textcolor{Red}{"{}\%d}}\texttt{\textcolor{CarnationPink}{\textbackslash{}n}}\texttt{\textcolor{Red}{"{}}}\textcolor{BrickRed}{,}\ dp\textcolor{BrickRed}{[}\textcolor{Purple}{0}\textcolor{BrickRed}{][}m\textcolor{BrickRed}{-}\textcolor{Purple}{1}\textcolor{BrickRed}{]);} \\
\mbox{}\ \ \textcolor{Red}{\}} \\
\mbox{}\ \ \textbf{\textcolor{Blue}{return}}\ \textcolor{Purple}{0}\textcolor{BrickRed}{;} \\
\mbox{}\textcolor{Red}{\}} \\

} \normalfont\normalsize
%.tex

\section{Geometría}
\subsection{Área de un polígono}
Si P es un polígono simple (no se intersecta a sí mismo) su área está dada por: \\

$ A(P) = \frac{1}{2} \displaystyle\sum_{i=0}^{n-1} (x_{i} \cdot y_{i+1} - x_{i+1} \cdot y_{i}) $ \\
\bigskip
% Generator: GNU source-highlight, by Lorenzo Bettini, http://www.gnu.org/software/src-highlite

{\ttfamily \raggedright {
\noindent
\mbox{}\textit{\textcolor{Brown}{//P\ es\ un\ polígono\ ordenado\ anticlockwise.}} \\
\mbox{}\textit{\textcolor{Brown}{//Si\ es\ clockwise,\ retorna\ el\ area\ negativa.}} \\
\mbox{}\textit{\textcolor{Brown}{//Si\ no\ esta\ ordenado\ retorna\ pura\ mierda.}} \\
\mbox{}\textit{\textcolor{Brown}{//P[0]\ !=\ P[n-1]}} \\
\mbox{}\textcolor{ForestGreen}{double}\ \textbf{\textcolor{Black}{PolygonArea}}\textcolor{BrickRed}{(}\textbf{\textcolor{Blue}{const}}\ vector\textcolor{BrickRed}{$<$}point\textcolor{BrickRed}{$>$}\ \textcolor{BrickRed}{\&}p\textcolor{BrickRed}{)}\textcolor{Red}{\{} \\
\mbox{}\ \ \textcolor{ForestGreen}{double}\ r\ \textcolor{BrickRed}{=}\ \textcolor{Purple}{0.0}\textcolor{BrickRed}{;} \\
\mbox{}\ \ \textbf{\textcolor{Blue}{for}}\ \textcolor{BrickRed}{(}\textcolor{ForestGreen}{int}\ i\textcolor{BrickRed}{=}\textcolor{Purple}{0}\textcolor{BrickRed}{;}\ i\textcolor{BrickRed}{$<$}p\textcolor{BrickRed}{.}\textbf{\textcolor{Black}{size}}\textcolor{BrickRed}{();}\ \textcolor{BrickRed}{++}i\textcolor{BrickRed}{)}\textcolor{Red}{\{} \\
\mbox{}\ \ \ \ \textcolor{ForestGreen}{int}\ j\ \textcolor{BrickRed}{=}\ \textcolor{BrickRed}{(}i\textcolor{BrickRed}{+}\textcolor{Purple}{1}\textcolor{BrickRed}{)}\ \textcolor{BrickRed}{\%}\ p\textcolor{BrickRed}{.}\textbf{\textcolor{Black}{size}}\textcolor{BrickRed}{();} \\
\mbox{}\ \ \ \ r\ \textcolor{BrickRed}{+=}\ p\textcolor{BrickRed}{[}i\textcolor{BrickRed}{].}x\textcolor{BrickRed}{*}p\textcolor{BrickRed}{[}j\textcolor{BrickRed}{].}y\ \textcolor{BrickRed}{-}\ p\textcolor{BrickRed}{[}j\textcolor{BrickRed}{].}x\textcolor{BrickRed}{*}p\textcolor{BrickRed}{[}i\textcolor{BrickRed}{].}y\textcolor{BrickRed}{;} \\
\mbox{}\ \ \textcolor{Red}{\}} \\
\mbox{}\ \ \textbf{\textcolor{Blue}{return}}\ r\textcolor{BrickRed}{/}\textcolor{Purple}{2.0}\textcolor{BrickRed}{;} \\
\mbox{}\textcolor{Red}{\}} \\

} \normalfont\normalsize
%.tex

\subsection{Centro de masa de un polígono}
Si P es un polígono simple (no se intersecta a sí mismo) su centro de masa está dado por: \\

$ \displaystyle\bar{C}_{x} = \frac{ \displaystyle\iint_{R} x \, dA }{M} = \frac{1}{6M}\sum_{i=1}^{n} (y_{i+1} - y_{i}) (x_{i+1}^2 + x_{i+1} \cdot x_{i} + x_{i}^2) $

\medskip

$\displaystyle\bar{C}_{y} = \frac{ \displaystyle\iint_{R} y \, dA }{M} = \frac{1}{6M} \sum_{i=1}^{n} (x_{i} - x_{i+1}) (y_{i+1}^2 + y_{i+1} \cdot y_{i} + y_{i}^2)$

\medskip

Donde $ M $ es el área del polígono. \\

Otra posible fórmula equivalente:

$ \displaystyle\bar{C}_{x} = \frac{1}{6A} \sum_{i=0}^{n-1} (x_{i} + x_{i+1}) (x_{i} \cdot y_{i+1} - x_{i+1} \cdot y_{i}) $

\medskip

$ \displaystyle\bar{C}_{y} = \frac{1}{6A} \sum_{i=0}^{n-1} (y_{i} + y_{i+1}) (x_{i} \cdot y_{i+1} - x_{i+1} \cdot y_{i}) $


\subsection{Convex hull: Graham Scan}
\emph{Complejidad:} $ O(n \log_{2}{n}) $
% Generator: GNU source-highlight, by Lorenzo Bettini, http://www.gnu.org/software/src-highlite

{\ttfamily \raggedright {
\noindent
\mbox{}\textit{\textcolor{Brown}{/*}} \\
\mbox{}\textit{\textcolor{Brown}{\ \ Graham\ Scan.}} \\
\mbox{}\textit{\textcolor{Brown}{\ */}} \\
\mbox{}\textbf{\textcolor{RoyalBlue}{\#include}}\ \texttt{\textcolor{Red}{$<$iostream$>$}} \\
\mbox{}\textbf{\textcolor{RoyalBlue}{\#include}}\ \texttt{\textcolor{Red}{$<$vector$>$}} \\
\mbox{}\textbf{\textcolor{RoyalBlue}{\#include}}\ \texttt{\textcolor{Red}{$<$algorithm$>$}} \\
\mbox{}\textbf{\textcolor{RoyalBlue}{\#include}}\ \texttt{\textcolor{Red}{$<$iterator$>$}} \\
\mbox{}\textbf{\textcolor{RoyalBlue}{\#include}}\ \texttt{\textcolor{Red}{$<$math.h$>$}} \\
\mbox{}\textbf{\textcolor{RoyalBlue}{\#include}}\ \texttt{\textcolor{Red}{$<$stdio.h$>$}} \\
\mbox{} \\
\mbox{}\textbf{\textcolor{Blue}{using}}\ \textbf{\textcolor{Blue}{namespace}}\ std\textcolor{BrickRed}{;} \\
\mbox{} \\
\mbox{}\textbf{\textcolor{Blue}{const}}\ \textcolor{ForestGreen}{double}\ pi\ \textcolor{BrickRed}{=}\ \textcolor{Purple}{2}\textcolor{BrickRed}{*}\textbf{\textcolor{Black}{acos}}\textcolor{BrickRed}{(}\textcolor{Purple}{0}\textcolor{BrickRed}{);} \\
\mbox{} \\
\mbox{}\textbf{\textcolor{Blue}{struct}}\ point\textcolor{Red}{\{} \\
\mbox{}\ \ \textcolor{ForestGreen}{int}\ x\textcolor{BrickRed}{,}y\textcolor{BrickRed}{;} \\
\mbox{}\ \ \textbf{\textcolor{Black}{point}}\textcolor{BrickRed}{()}\ \textcolor{Red}{\{\}} \\
\mbox{}\ \ \textbf{\textcolor{Black}{point}}\textcolor{BrickRed}{(}\textcolor{ForestGreen}{int}\ X\textcolor{BrickRed}{,}\ \textcolor{ForestGreen}{int}\ Y\textcolor{BrickRed}{)}\ \textcolor{BrickRed}{:}\ \textbf{\textcolor{Black}{x}}\textcolor{BrickRed}{(}X\textcolor{BrickRed}{),}\ \textbf{\textcolor{Black}{y}}\textcolor{BrickRed}{(}Y\textcolor{BrickRed}{)}\ \textcolor{Red}{\{\}} \\
\mbox{}\textcolor{Red}{\}}\textcolor{BrickRed}{;} \\
\mbox{} \\
\mbox{}point\ pivot\textcolor{BrickRed}{;} \\
\mbox{} \\
\mbox{}ostream\textcolor{BrickRed}{\&}\ \textbf{\textcolor{Blue}{operator}}\textcolor{BrickRed}{$<$$<$}\ \textcolor{BrickRed}{(}ostream\textcolor{BrickRed}{\&}\ out\textcolor{BrickRed}{,}\ \textbf{\textcolor{Blue}{const}}\ point\textcolor{BrickRed}{\&}\ c\textcolor{BrickRed}{)} \\
\mbox{}\textcolor{Red}{\{} \\
\mbox{}\ \ out\ \textcolor{BrickRed}{$<$$<$}\ \texttt{\textcolor{Red}{"{}("{}}}\ \textcolor{BrickRed}{$<$$<$}\ c\textcolor{BrickRed}{.}x\ \textcolor{BrickRed}{$<$$<$}\ \texttt{\textcolor{Red}{"{},"{}}}\ \textcolor{BrickRed}{$<$$<$}\ c\textcolor{BrickRed}{.}y\ \textcolor{BrickRed}{$<$$<$}\ \texttt{\textcolor{Red}{"{})"{}}}\textcolor{BrickRed}{;} \\
\mbox{}\ \ \textbf{\textcolor{Blue}{return}}\ out\textcolor{BrickRed}{;} \\
\mbox{}\textcolor{Red}{\}} \\
\mbox{} \\
\mbox{}\textbf{\textcolor{Blue}{inline}}\ \textcolor{ForestGreen}{int}\ \textbf{\textcolor{Black}{distsqr}}\textcolor{BrickRed}{(}\textbf{\textcolor{Blue}{const}}\ point\ \textcolor{BrickRed}{\&}a\textcolor{BrickRed}{,}\ \textbf{\textcolor{Blue}{const}}\ point\ \textcolor{BrickRed}{\&}b\textcolor{BrickRed}{)}\textcolor{Red}{\{} \\
\mbox{}\ \ \textbf{\textcolor{Blue}{return}}\ \textcolor{BrickRed}{(}a\textcolor{BrickRed}{.}x\ \textcolor{BrickRed}{-}\ b\textcolor{BrickRed}{.}x\textcolor{BrickRed}{)*(}a\textcolor{BrickRed}{.}x\ \textcolor{BrickRed}{-}\ b\textcolor{BrickRed}{.}x\textcolor{BrickRed}{)}\ \textcolor{BrickRed}{+}\ \textcolor{BrickRed}{(}a\textcolor{BrickRed}{.}y\ \textcolor{BrickRed}{-}\ b\textcolor{BrickRed}{.}y\textcolor{BrickRed}{)*(}a\textcolor{BrickRed}{.}y\ \textcolor{BrickRed}{-}\ b\textcolor{BrickRed}{.}y\textcolor{BrickRed}{);} \\
\mbox{}\textcolor{Red}{\}} \\
\mbox{} \\
\mbox{}\textbf{\textcolor{Blue}{inline}}\ \textcolor{ForestGreen}{double}\ \textbf{\textcolor{Black}{dist}}\textcolor{BrickRed}{(}\textbf{\textcolor{Blue}{const}}\ point\ \textcolor{BrickRed}{\&}a\textcolor{BrickRed}{,}\ \textbf{\textcolor{Blue}{const}}\ point\ \textcolor{BrickRed}{\&}b\textcolor{BrickRed}{)}\textcolor{Red}{\{} \\
\mbox{}\ \ \textbf{\textcolor{Blue}{return}}\ \textbf{\textcolor{Black}{sqrt}}\textcolor{BrickRed}{(}\textbf{\textcolor{Black}{distsqr}}\textcolor{BrickRed}{(}a\textcolor{BrickRed}{,}\ b\textcolor{BrickRed}{));} \\
\mbox{}\textcolor{Red}{\}} \\
\mbox{} \\
\mbox{}\textit{\textcolor{Brown}{//retorna\ $>$\ 0\ si\ c\ esta\ a\ la\ izquierda\ del\ segmento\ AB}} \\
\mbox{}\textit{\textcolor{Brown}{//retorna\ $<$\ 0\ si\ c\ esta\ a\ la\ derecha\ del\ segmento\ AB}} \\
\mbox{}\textit{\textcolor{Brown}{//retorna\ ==\ 0\ si\ c\ es\ colineal\ con\ el\ segmento\ AB}} \\
\mbox{}\textbf{\textcolor{Blue}{inline}}\ \textcolor{ForestGreen}{int}\ \textbf{\textcolor{Black}{cross}}\textcolor{BrickRed}{(}\textbf{\textcolor{Blue}{const}}\ point\ \textcolor{BrickRed}{\&}a\textcolor{BrickRed}{,}\ \textbf{\textcolor{Blue}{const}}\ point\ \textcolor{BrickRed}{\&}b\textcolor{BrickRed}{,}\ \textbf{\textcolor{Blue}{const}}\ point\ \textcolor{BrickRed}{\&}c\textcolor{BrickRed}{)}\textcolor{Red}{\{} \\
\mbox{}\ \ \textbf{\textcolor{Blue}{return}}\ \textcolor{BrickRed}{(}b\textcolor{BrickRed}{.}x\textcolor{BrickRed}{-}a\textcolor{BrickRed}{.}x\textcolor{BrickRed}{)*(}c\textcolor{BrickRed}{.}y\textcolor{BrickRed}{-}a\textcolor{BrickRed}{.}y\textcolor{BrickRed}{)}\ \textcolor{BrickRed}{-}\ \textcolor{BrickRed}{(}c\textcolor{BrickRed}{.}x\textcolor{BrickRed}{-}a\textcolor{BrickRed}{.}x\textcolor{BrickRed}{)*(}b\textcolor{BrickRed}{.}y\textcolor{BrickRed}{-}a\textcolor{BrickRed}{.}y\textcolor{BrickRed}{);} \\
\mbox{}\textcolor{Red}{\}} \\
\mbox{} \\
\mbox{}\textit{\textcolor{Brown}{//Self\ $<$\ that\ si\ esta\ a\ la\ derecha\ del\ segmento\ Pivot-That}} \\
\mbox{}\textcolor{ForestGreen}{bool}\ \textbf{\textcolor{Black}{angleCmp}}\textcolor{BrickRed}{(}\textbf{\textcolor{Blue}{const}}\ point\ \textcolor{BrickRed}{\&}self\textcolor{BrickRed}{,}\ \textbf{\textcolor{Blue}{const}}\ point\ \textcolor{BrickRed}{\&}that\textcolor{BrickRed}{)}\textcolor{Red}{\{} \\
\mbox{}\ \ \textcolor{ForestGreen}{int}\ t\ \textcolor{BrickRed}{=}\ \textbf{\textcolor{Black}{cross}}\textcolor{BrickRed}{(}pivot\textcolor{BrickRed}{,}\ that\textcolor{BrickRed}{,}\ self\textcolor{BrickRed}{);} \\
\mbox{}\ \ \textbf{\textcolor{Blue}{if}}\ \textcolor{BrickRed}{(}t\ \textcolor{BrickRed}{$<$}\ \textcolor{Purple}{0}\textcolor{BrickRed}{)}\ \textbf{\textcolor{Blue}{return}}\ \textbf{\textcolor{Blue}{true}}\textcolor{BrickRed}{;} \\
\mbox{}\ \ \textbf{\textcolor{Blue}{if}}\ \textcolor{BrickRed}{(}t\ \textcolor{BrickRed}{==}\ \textcolor{Purple}{0}\textcolor{BrickRed}{)}\textcolor{Red}{\{} \\
\mbox{}\ \ \ \ \textit{\textcolor{Brown}{//Self\ $<$\ that\ si\ está\ más\ cerquita}} \\
\mbox{}\ \ \ \ \textbf{\textcolor{Blue}{return}}\ \textcolor{BrickRed}{(}\textbf{\textcolor{Black}{distsqr}}\textcolor{BrickRed}{(}pivot\textcolor{BrickRed}{,}\ self\textcolor{BrickRed}{)}\ \textcolor{BrickRed}{$<$}\ \textbf{\textcolor{Black}{distsqr}}\textcolor{BrickRed}{(}pivot\textcolor{BrickRed}{,}\ that\textcolor{BrickRed}{));} \\
\mbox{}\ \ \textcolor{Red}{\}} \\
\mbox{}\ \ \textbf{\textcolor{Blue}{return}}\ \textbf{\textcolor{Blue}{false}}\textcolor{BrickRed}{;} \\
\mbox{}\textcolor{Red}{\}} \\
\mbox{} \\
\mbox{}vector\textcolor{BrickRed}{$<$}point\textcolor{BrickRed}{$>$}\ \textbf{\textcolor{Black}{graham}}\textcolor{BrickRed}{(}vector\textcolor{BrickRed}{$<$}point\textcolor{BrickRed}{$>$}\ p\textcolor{BrickRed}{)}\textcolor{Red}{\{} \\
\mbox{}\ \ \textit{\textcolor{Brown}{//Metemos\ el\ más\ abajo\ más\ a\ la\ izquierda\ en\ la\ posición\ 0}} \\
\mbox{}\ \ \textbf{\textcolor{Blue}{for}}\ \textcolor{BrickRed}{(}\textcolor{ForestGreen}{int}\ i\textcolor{BrickRed}{=}\textcolor{Purple}{1}\textcolor{BrickRed}{;}\ i\textcolor{BrickRed}{$<$}p\textcolor{BrickRed}{.}\textbf{\textcolor{Black}{size}}\textcolor{BrickRed}{();}\ \textcolor{BrickRed}{++}i\textcolor{BrickRed}{)}\textcolor{Red}{\{} \\
\mbox{}\ \ \ \ \textbf{\textcolor{Blue}{if}}\ \textcolor{BrickRed}{(}p\textcolor{BrickRed}{[}i\textcolor{BrickRed}{].}y\ \textcolor{BrickRed}{$<$}\ p\textcolor{BrickRed}{[}\textcolor{Purple}{0}\textcolor{BrickRed}{].}y\ \textcolor{BrickRed}{$|$$|$}\ \textcolor{BrickRed}{(}p\textcolor{BrickRed}{[}i\textcolor{BrickRed}{].}y\ \textcolor{BrickRed}{==}\ p\textcolor{BrickRed}{[}\textcolor{Purple}{0}\textcolor{BrickRed}{].}y\ \textcolor{BrickRed}{\&\&}\ p\textcolor{BrickRed}{[}i\textcolor{BrickRed}{].}x\ \textcolor{BrickRed}{$<$}\ p\textcolor{BrickRed}{[}\textcolor{Purple}{0}\textcolor{BrickRed}{].}x\ \textcolor{BrickRed}{))} \\
\mbox{}\ \ \ \ \ \ \textbf{\textcolor{Black}{swap}}\textcolor{BrickRed}{(}p\textcolor{BrickRed}{[}\textcolor{Purple}{0}\textcolor{BrickRed}{],}\ p\textcolor{BrickRed}{[}i\textcolor{BrickRed}{]);} \\
\mbox{}\ \ \textcolor{Red}{\}} \\
\mbox{}\ \  \\
\mbox{}\ \ pivot\ \textcolor{BrickRed}{=}\ p\textcolor{BrickRed}{[}\textcolor{Purple}{0}\textcolor{BrickRed}{];} \\
\mbox{}\ \ \textbf{\textcolor{Black}{sort}}\textcolor{BrickRed}{(}p\textcolor{BrickRed}{.}\textbf{\textcolor{Black}{begin}}\textcolor{BrickRed}{(),}\ p\textcolor{BrickRed}{.}\textbf{\textcolor{Black}{end}}\textcolor{BrickRed}{(),}\ angleCmp\textcolor{BrickRed}{);} \\
\mbox{} \\
\mbox{}\ \ \textit{\textcolor{Brown}{//Ordenar\ por\ ángulo\ y\ eliminar\ repetidos.}} \\
\mbox{}\ \ \textit{\textcolor{Brown}{//Si\ varios\ puntos\ tienen\ el\ mismo\ angulo\ el\ más\ lejano\ queda\ después\ en\ la\ lista}} \\
\mbox{}\ \ vector\textcolor{BrickRed}{$<$}point\textcolor{BrickRed}{$>$}\ \textbf{\textcolor{Black}{chull}}\textcolor{BrickRed}{(}p\textcolor{BrickRed}{.}\textbf{\textcolor{Black}{begin}}\textcolor{BrickRed}{(),}\ p\textcolor{BrickRed}{.}\textbf{\textcolor{Black}{begin}}\textcolor{BrickRed}{()+}\textcolor{Purple}{3}\textcolor{BrickRed}{);} \\
\mbox{} \\
\mbox{}\ \ \textit{\textcolor{Brown}{//Ahora\ sí!!!}} \\
\mbox{}\ \ \textbf{\textcolor{Blue}{for}}\ \textcolor{BrickRed}{(}\textcolor{ForestGreen}{int}\ i\textcolor{BrickRed}{=}\textcolor{Purple}{3}\textcolor{BrickRed}{;}\ i\textcolor{BrickRed}{$<$}p\textcolor{BrickRed}{.}\textbf{\textcolor{Black}{size}}\textcolor{BrickRed}{();}\ \textcolor{BrickRed}{++}i\textcolor{BrickRed}{)}\textcolor{Red}{\{} \\
\mbox{}\ \ \ \ \textbf{\textcolor{Blue}{while}}\ \textcolor{BrickRed}{(}\ chull\textcolor{BrickRed}{.}\textbf{\textcolor{Black}{size}}\textcolor{BrickRed}{()}\ \textcolor{BrickRed}{$>$=}\ \textcolor{Purple}{2}\ \textcolor{BrickRed}{\&\&}\ \textbf{\textcolor{Black}{cross}}\textcolor{BrickRed}{(}chull\textcolor{BrickRed}{[}chull\textcolor{BrickRed}{.}\textbf{\textcolor{Black}{size}}\textcolor{BrickRed}{()-}\textcolor{Purple}{2}\textcolor{BrickRed}{],}\ chull\textcolor{BrickRed}{[}chull\textcolor{BrickRed}{.}\textbf{\textcolor{Black}{size}}\textcolor{BrickRed}{()-}\textcolor{Purple}{1}\textcolor{BrickRed}{],}\ p\textcolor{BrickRed}{[}i\textcolor{BrickRed}{])}\ \textcolor{BrickRed}{$<$=}\ \textcolor{Purple}{0}\textcolor{BrickRed}{)}\textcolor{Red}{\{} \\
\mbox{}\ \ \ \ \ \ chull\textcolor{BrickRed}{.}\textbf{\textcolor{Black}{erase}}\textcolor{BrickRed}{(}chull\textcolor{BrickRed}{.}\textbf{\textcolor{Black}{end}}\textcolor{BrickRed}{()}\ \textcolor{BrickRed}{-}\ \textcolor{Purple}{1}\textcolor{BrickRed}{);} \\
\mbox{}\ \ \ \ \textcolor{Red}{\}} \\
\mbox{}\ \ \ \ chull\textcolor{BrickRed}{.}\textbf{\textcolor{Black}{push$\_$back}}\textcolor{BrickRed}{(}p\textcolor{BrickRed}{[}i\textcolor{BrickRed}{]);} \\
\mbox{}\ \ \textcolor{Red}{\}} \\
\mbox{}\ \ \textit{\textcolor{Brown}{//chull\ contiene\ los\ puntos\ del\ convex\ hull\ ordenados\ anti-clockwise.}} \\
\mbox{}\ \ \textit{\textcolor{Brown}{//No\ contiene\ ningún\ punto\ repetido.}} \\
\mbox{}\ \ \textit{\textcolor{Brown}{//El\ primer\ punto\ no\ es\ el\ mismo\ que\ el\ último,\ i.e,\ la\ última\ arista}} \\
\mbox{}\ \ \textit{\textcolor{Brown}{//va\ de\ chull[chull.size()-1]\ a\ chull[0]}} \\
\mbox{}\ \ \textbf{\textcolor{Blue}{return}}\ chull\textcolor{BrickRed}{;} \\
\mbox{}\textcolor{Red}{\}} \\

} \normalfont\normalsize
%.tex

\subsection{Convex hull: Andrew's monotone chain}
\emph{Complejidad:} $ O(n \log_{2}{n}) $
% Generator: GNU source-highlight, by Lorenzo Bettini, http://www.gnu.org/software/src-highlite

{\ttfamily \raggedright {
\noindent
\mbox{}\textit{\textcolor{Brown}{//\ Implementation\ of\ Monotone\ Chain\ Convex\ Hull\ algorithm.}} \\
\mbox{}\textbf{\textcolor{RoyalBlue}{\#include}}\ \texttt{\textcolor{Red}{$<$algorithm$>$}} \\
\mbox{}\textbf{\textcolor{RoyalBlue}{\#include}}\ \texttt{\textcolor{Red}{$<$vector$>$}} \\
\mbox{}\textbf{\textcolor{Blue}{using}}\ \textbf{\textcolor{Blue}{namespace}}\ std\textcolor{BrickRed}{;} \\
\mbox{}\  \\
\mbox{}\textbf{\textcolor{Blue}{typedef}}\ \textcolor{ForestGreen}{long}\ \textcolor{ForestGreen}{long}\ CoordType\textcolor{BrickRed}{;} \\
\mbox{}\  \\
\mbox{}\textbf{\textcolor{Blue}{struct}}\ Point\ \textcolor{Red}{\{} \\
\mbox{}\ \ \ \ \ \ \ \ CoordType\ x\textcolor{BrickRed}{,}\ y\textcolor{BrickRed}{;} \\
\mbox{}\  \\
\mbox{}\ \ \ \ \ \ \ \ \textcolor{ForestGreen}{bool}\ \textbf{\textcolor{Blue}{operator}}\ \textcolor{BrickRed}{$<$(}\textbf{\textcolor{Blue}{const}}\ Point\ \textcolor{BrickRed}{\&}p\textcolor{BrickRed}{)}\ \textbf{\textcolor{Blue}{const}}\ \textcolor{Red}{\{} \\
\mbox{}\ \ \ \ \ \ \ \ \ \ \ \ \ \ \ \ \textbf{\textcolor{Blue}{return}}\ x\ \textcolor{BrickRed}{$<$}\ p\textcolor{BrickRed}{.}x\ \textcolor{BrickRed}{$|$$|$}\ \textcolor{BrickRed}{(}x\ \textcolor{BrickRed}{==}\ p\textcolor{BrickRed}{.}x\ \textcolor{BrickRed}{\&\&}\ y\ \textcolor{BrickRed}{$<$}\ p\textcolor{BrickRed}{.}y\textcolor{BrickRed}{);} \\
\mbox{}\ \ \ \ \ \ \ \ \textcolor{Red}{\}} \\
\mbox{}\textcolor{Red}{\}}\textcolor{BrickRed}{;} \\
\mbox{}\  \\
\mbox{}\textit{\textcolor{Brown}{//\ 2D\ cross\ product.}} \\
\mbox{}\textit{\textcolor{Brown}{//\ Return\ a\ positive\ value,\ if\ OAB\ makes\ a\ counter-clockwise\ turn,}} \\
\mbox{}\textit{\textcolor{Brown}{//\ negative\ for\ clockwise\ turn,\ and\ zero\ if\ the\ points\ are\ collinear.}} \\
\mbox{}CoordType\ \textbf{\textcolor{Black}{cross}}\textcolor{BrickRed}{(}\textbf{\textcolor{Blue}{const}}\ Point\ \textcolor{BrickRed}{\&}O\textcolor{BrickRed}{,}\ \textbf{\textcolor{Blue}{const}}\ Point\ \textcolor{BrickRed}{\&}A\textcolor{BrickRed}{,}\ \textbf{\textcolor{Blue}{const}}\ Point\ \textcolor{BrickRed}{\&}B\textcolor{BrickRed}{)} \\
\mbox{}\textcolor{Red}{\{} \\
\mbox{}\ \ \ \ \ \ \ \ \textbf{\textcolor{Blue}{return}}\ \textcolor{BrickRed}{(}A\textcolor{BrickRed}{.}x\ \textcolor{BrickRed}{-}\ O\textcolor{BrickRed}{.}x\textcolor{BrickRed}{)}\ \textcolor{BrickRed}{*}\ \textcolor{BrickRed}{(}B\textcolor{BrickRed}{.}y\ \textcolor{BrickRed}{-}\ O\textcolor{BrickRed}{.}y\textcolor{BrickRed}{)}\ \textcolor{BrickRed}{-}\ \textcolor{BrickRed}{(}A\textcolor{BrickRed}{.}y\ \textcolor{BrickRed}{-}\ O\textcolor{BrickRed}{.}y\textcolor{BrickRed}{)}\ \textcolor{BrickRed}{*}\ \textcolor{BrickRed}{(}B\textcolor{BrickRed}{.}x\ \textcolor{BrickRed}{-}\ O\textcolor{BrickRed}{.}x\textcolor{BrickRed}{);} \\
\mbox{}\textcolor{Red}{\}} \\
\mbox{}\  \\
\mbox{}\textit{\textcolor{Brown}{//\ Returns\ a\ list\ of\ points\ on\ the\ convex\ hull\ in\ counter-clockwise\ order.}} \\
\mbox{}\textit{\textcolor{Brown}{//\ Note:\ the\ last\ point\ in\ the\ returned\ list\ is\ the\ same\ as\ the\ first\ one.}} \\
\mbox{}vector\textcolor{BrickRed}{$<$}Point\textcolor{BrickRed}{$>$}\ \textbf{\textcolor{Black}{convexHull}}\textcolor{BrickRed}{(}vector\textcolor{BrickRed}{$<$}Point\textcolor{BrickRed}{$>$}\ P\textcolor{BrickRed}{)} \\
\mbox{}\textcolor{Red}{\{} \\
\mbox{}\ \ \ \ \ \ \ \ \textcolor{ForestGreen}{int}\ n\ \textcolor{BrickRed}{=}\ P\textcolor{BrickRed}{.}\textbf{\textcolor{Black}{size}}\textcolor{BrickRed}{(),}\ k\ \textcolor{BrickRed}{=}\ \textcolor{Purple}{0}\textcolor{BrickRed}{;} \\
\mbox{}\ \ \ \ \ \ \ \ vector\textcolor{BrickRed}{$<$}Point\textcolor{BrickRed}{$>$}\ \textbf{\textcolor{Black}{H}}\textcolor{BrickRed}{(}\textcolor{Purple}{2}\textcolor{BrickRed}{*}n\textcolor{BrickRed}{);} \\
\mbox{}\  \\
\mbox{}\ \ \ \ \ \ \ \ \textit{\textcolor{Brown}{//\ Sort\ points\ lexicographically}} \\
\mbox{}\ \ \ \ \ \ \ \ \textbf{\textcolor{Black}{sort}}\textcolor{BrickRed}{(}P\textcolor{BrickRed}{.}\textbf{\textcolor{Black}{begin}}\textcolor{BrickRed}{(),}\ P\textcolor{BrickRed}{.}\textbf{\textcolor{Black}{end}}\textcolor{BrickRed}{());} \\
\mbox{}\  \\
\mbox{}\ \ \ \ \ \ \ \ \textit{\textcolor{Brown}{//\ Build\ lower\ hull}} \\
\mbox{}\ \ \ \ \ \ \ \ \textbf{\textcolor{Blue}{for}}\ \textcolor{BrickRed}{(}\textcolor{ForestGreen}{int}\ i\ \textcolor{BrickRed}{=}\ \textcolor{Purple}{0}\textcolor{BrickRed}{;}\ i\ \textcolor{BrickRed}{$<$}\ n\textcolor{BrickRed}{;}\ i\textcolor{BrickRed}{++)}\ \textcolor{Red}{\{} \\
\mbox{}\ \ \ \ \ \ \ \ \ \ \ \ \ \ \ \ \textbf{\textcolor{Blue}{while}}\ \textcolor{BrickRed}{(}k\ \textcolor{BrickRed}{$>$=}\ \textcolor{Purple}{2}\ \textcolor{BrickRed}{\&\&}\ \textbf{\textcolor{Black}{cross}}\textcolor{BrickRed}{(}H\textcolor{BrickRed}{[}k\textcolor{BrickRed}{-}\textcolor{Purple}{2}\textcolor{BrickRed}{],}\ H\textcolor{BrickRed}{[}k\textcolor{BrickRed}{-}\textcolor{Purple}{1}\textcolor{BrickRed}{],}\ P\textcolor{BrickRed}{[}i\textcolor{BrickRed}{])}\ \textcolor{BrickRed}{$<$=}\ \textcolor{Purple}{0}\textcolor{BrickRed}{)}\ k\textcolor{BrickRed}{-\/-;} \\
\mbox{}\ \ \ \ \ \ \ \ \ \ \ \ \ \ \ \ H\textcolor{BrickRed}{[}k\textcolor{BrickRed}{++]}\ \textcolor{BrickRed}{=}\ P\textcolor{BrickRed}{[}i\textcolor{BrickRed}{];} \\
\mbox{}\ \ \ \ \ \ \ \ \textcolor{Red}{\}} \\
\mbox{}\  \\
\mbox{}\ \ \ \ \ \ \ \ \textit{\textcolor{Brown}{//\ Build\ upper\ hull}} \\
\mbox{}\ \ \ \ \ \ \ \ \textbf{\textcolor{Blue}{for}}\ \textcolor{BrickRed}{(}\textcolor{ForestGreen}{int}\ i\ \textcolor{BrickRed}{=}\ n\textcolor{BrickRed}{-}\textcolor{Purple}{2}\textcolor{BrickRed}{,}\ t\ \textcolor{BrickRed}{=}\ k\textcolor{BrickRed}{+}\textcolor{Purple}{1}\textcolor{BrickRed}{;}\ i\ \textcolor{BrickRed}{$>$=}\ \textcolor{Purple}{0}\textcolor{BrickRed}{;}\ i\textcolor{BrickRed}{-\/-)}\ \textcolor{Red}{\{} \\
\mbox{}\ \ \ \ \ \ \ \ \ \ \ \ \ \ \ \ \textbf{\textcolor{Blue}{while}}\ \textcolor{BrickRed}{(}k\ \textcolor{BrickRed}{$>$=}\ t\ \textcolor{BrickRed}{\&\&}\ \textbf{\textcolor{Black}{cross}}\textcolor{BrickRed}{(}H\textcolor{BrickRed}{[}k\textcolor{BrickRed}{-}\textcolor{Purple}{2}\textcolor{BrickRed}{],}\ H\textcolor{BrickRed}{[}k\textcolor{BrickRed}{-}\textcolor{Purple}{1}\textcolor{BrickRed}{],}\ P\textcolor{BrickRed}{[}i\textcolor{BrickRed}{])}\ \textcolor{BrickRed}{$<$=}\ \textcolor{Purple}{0}\textcolor{BrickRed}{)}\ k\textcolor{BrickRed}{-\/-;} \\
\mbox{}\ \ \ \ \ \ \ \ \ \ \ \ \ \ \ \ H\textcolor{BrickRed}{[}k\textcolor{BrickRed}{++]}\ \textcolor{BrickRed}{=}\ P\textcolor{BrickRed}{[}i\textcolor{BrickRed}{];} \\
\mbox{}\ \ \ \ \ \ \ \ \textcolor{Red}{\}} \\
\mbox{}\  \\
\mbox{}\ \ \ \ \ \ \ \ H\textcolor{BrickRed}{.}\textbf{\textcolor{Black}{resize}}\textcolor{BrickRed}{(}k\textcolor{BrickRed}{);} \\
\mbox{}\ \ \ \ \ \ \ \ \textbf{\textcolor{Blue}{return}}\ H\textcolor{BrickRed}{;} \\
\mbox{}\textcolor{Red}{\}} \\

} \normalfont\normalsize
%.tex

\subsection{Mínima distancia entre un punto y un segmento}
% Generator: GNU source-highlight, by Lorenzo Bettini, http://www.gnu.org/software/src-highlite

{\ttfamily \raggedright {
\noindent
\mbox{}\textbf{\textcolor{Blue}{struct}}\ point\textcolor{Red}{\{} \\
\mbox{}\ \ \textcolor{ForestGreen}{double}\ x\textcolor{BrickRed}{,}y\textcolor{BrickRed}{;} \\
\mbox{}\textcolor{Red}{\}}\textcolor{BrickRed}{;} \\
\mbox{} \\
\mbox{}\textbf{\textcolor{Blue}{inline}}\ \textcolor{ForestGreen}{double}\ \textbf{\textcolor{Black}{dist}}\textcolor{BrickRed}{(}\textbf{\textcolor{Blue}{const}}\ point\ \textcolor{BrickRed}{\&}a\textcolor{BrickRed}{,}\ \textbf{\textcolor{Blue}{const}}\ point\ \textcolor{BrickRed}{\&}b\textcolor{BrickRed}{)}\textcolor{Red}{\{} \\
\mbox{}\ \ \textbf{\textcolor{Blue}{return}}\ \textbf{\textcolor{Black}{sqrt}}\textcolor{BrickRed}{((}a\textcolor{BrickRed}{.}x\textcolor{BrickRed}{-}b\textcolor{BrickRed}{.}x\textcolor{BrickRed}{)*(}a\textcolor{BrickRed}{.}x\textcolor{BrickRed}{-}b\textcolor{BrickRed}{.}x\textcolor{BrickRed}{)}\ \textcolor{BrickRed}{+}\ \textcolor{BrickRed}{(}a\textcolor{BrickRed}{.}y\textcolor{BrickRed}{-}b\textcolor{BrickRed}{.}y\textcolor{BrickRed}{)*(}a\textcolor{BrickRed}{.}y\textcolor{BrickRed}{-}b\textcolor{BrickRed}{.}y\textcolor{BrickRed}{));} \\
\mbox{}\textcolor{Red}{\}} \\
\mbox{} \\
\mbox{}\textbf{\textcolor{Blue}{inline}}\ \textcolor{ForestGreen}{double}\ \textbf{\textcolor{Black}{distsqr}}\textcolor{BrickRed}{(}\textbf{\textcolor{Blue}{const}}\ point\ \textcolor{BrickRed}{\&}a\textcolor{BrickRed}{,}\ \textbf{\textcolor{Blue}{const}}\ point\ \textcolor{BrickRed}{\&}b\textcolor{BrickRed}{)}\textcolor{Red}{\{} \\
\mbox{}\ \ \textbf{\textcolor{Blue}{return}}\ \textcolor{BrickRed}{(}a\textcolor{BrickRed}{.}x\textcolor{BrickRed}{-}b\textcolor{BrickRed}{.}x\textcolor{BrickRed}{)*(}a\textcolor{BrickRed}{.}x\textcolor{BrickRed}{-}b\textcolor{BrickRed}{.}x\textcolor{BrickRed}{)}\ \textcolor{BrickRed}{+}\ \textcolor{BrickRed}{(}a\textcolor{BrickRed}{.}y\textcolor{BrickRed}{-}b\textcolor{BrickRed}{.}y\textcolor{BrickRed}{)*(}a\textcolor{BrickRed}{.}y\textcolor{BrickRed}{-}b\textcolor{BrickRed}{.}y\textcolor{BrickRed}{);} \\
\mbox{}\textcolor{Red}{\}} \\
\mbox{} \\
\mbox{}\textit{\textcolor{Brown}{/*}} \\
\mbox{}\textit{\textcolor{Brown}{\ \ Returns\ the\ closest\ distance\ between\ point\ pnt\ and\ the\ segment\ that\ goes\ from\ point\ a\ to\ b}} \\
\mbox{}\textit{\textcolor{Brown}{\ \ Idea\ by:\ }}\underline{\texttt{\textcolor{Blue}{http://local.wasp.uwa.edu.au/}}}\textit{\textcolor{Brown}{\textasciitilde{}pbourke/geometry/pointline/}} \\
\mbox{}\textit{\textcolor{Brown}{\ */}} \\
\mbox{}\textcolor{ForestGreen}{double}\ \textbf{\textcolor{Black}{distance$\_$point$\_$to$\_$segment}}\textcolor{BrickRed}{(}\textbf{\textcolor{Blue}{const}}\ point\ \textcolor{BrickRed}{\&}a\textcolor{BrickRed}{,}\ \textbf{\textcolor{Blue}{const}}\ point\ \textcolor{BrickRed}{\&}b\textcolor{BrickRed}{,}\ \textbf{\textcolor{Blue}{const}}\ point\ \textcolor{BrickRed}{\&}pnt\textcolor{BrickRed}{)}\textcolor{Red}{\{} \\
\mbox{}\ \ \textcolor{ForestGreen}{double}\ u\ \textcolor{BrickRed}{=}\ \textcolor{BrickRed}{((}pnt\textcolor{BrickRed}{.}x\ \textcolor{BrickRed}{-}\ a\textcolor{BrickRed}{.}x\textcolor{BrickRed}{)*(}b\textcolor{BrickRed}{.}x\ \textcolor{BrickRed}{-}\ a\textcolor{BrickRed}{.}x\textcolor{BrickRed}{)}\ \textcolor{BrickRed}{+}\ \textcolor{BrickRed}{(}pnt\textcolor{BrickRed}{.}y\ \textcolor{BrickRed}{-}\ a\textcolor{BrickRed}{.}y\textcolor{BrickRed}{)*(}b\textcolor{BrickRed}{.}y\ \textcolor{BrickRed}{-}\ a\textcolor{BrickRed}{.}y\textcolor{BrickRed}{))}\ \textcolor{BrickRed}{/}\ \textbf{\textcolor{Black}{distsqr}}\textcolor{BrickRed}{(}a\textcolor{BrickRed}{,}\ b\textcolor{BrickRed}{);} \\
\mbox{}\ \ point\ intersection\textcolor{BrickRed}{;} \\
\mbox{}\ \ intersection\textcolor{BrickRed}{.}x\ \textcolor{BrickRed}{=}\ a\textcolor{BrickRed}{.}x\ \textcolor{BrickRed}{+}\ u\textcolor{BrickRed}{*(}b\textcolor{BrickRed}{.}x\ \textcolor{BrickRed}{-}\ a\textcolor{BrickRed}{.}x\textcolor{BrickRed}{);} \\
\mbox{}\ \ intersection\textcolor{BrickRed}{.}y\ \textcolor{BrickRed}{=}\ a\textcolor{BrickRed}{.}y\ \textcolor{BrickRed}{+}\ u\textcolor{BrickRed}{*(}b\textcolor{BrickRed}{.}y\ \textcolor{BrickRed}{-}\ a\textcolor{BrickRed}{.}y\textcolor{BrickRed}{);} \\
\mbox{}\ \ \textbf{\textcolor{Blue}{if}}\ \textcolor{BrickRed}{(}u\ \textcolor{BrickRed}{$<$}\ \textcolor{Purple}{0.0}\ \textcolor{BrickRed}{$|$$|$}\ u\ \textcolor{BrickRed}{$>$}\ \textcolor{Purple}{1.0}\textcolor{BrickRed}{)}\textcolor{Red}{\{} \\
\mbox{}\ \ \ \ \textbf{\textcolor{Blue}{return}}\ \textbf{\textcolor{Black}{min}}\textcolor{BrickRed}{(}\textbf{\textcolor{Black}{dist}}\textcolor{BrickRed}{(}a\textcolor{BrickRed}{,}\ pnt\textcolor{BrickRed}{),}\ \textbf{\textcolor{Black}{dist}}\textcolor{BrickRed}{(}b\textcolor{BrickRed}{,}\ pnt\textcolor{BrickRed}{));} \\
\mbox{}\ \ \textcolor{Red}{\}} \\
\mbox{}\ \ \textbf{\textcolor{Blue}{return}}\ \textbf{\textcolor{Black}{dist}}\textcolor{BrickRed}{(}pnt\textcolor{BrickRed}{,}\ intersection\textcolor{BrickRed}{);} \\
\mbox{}\textcolor{Red}{\}} \\

} \normalfont\normalsize
%.tex

\subsection{Mínima distancia entre un punto y una recta}
% Generator: GNU source-highlight, by Lorenzo Bettini, http://www.gnu.org/software/src-highlite

{\ttfamily \raggedright {
\noindent
\mbox{}\textit{\textcolor{Brown}{/*}} \\
\mbox{}\textit{\textcolor{Brown}{\ \ Returns\ the\ closest\ distance\ between\ point\ pnt\ and\ the\ line\ that\ passes\ through\ points\ a\ and\ b}} \\
\mbox{}\textit{\textcolor{Brown}{\ \ Idea\ by:\ }}\underline{\texttt{\textcolor{Blue}{http://local.wasp.uwa.edu.au/}}}\textit{\textcolor{Brown}{\textasciitilde{}pbourke/geometry/pointline/}} \\
\mbox{}\textit{\textcolor{Brown}{\ */}} \\
\mbox{}\textcolor{ForestGreen}{double}\ \textbf{\textcolor{Black}{distance$\_$point$\_$to$\_$line}}\textcolor{BrickRed}{(}\textbf{\textcolor{Blue}{const}}\ point\ \textcolor{BrickRed}{\&}a\textcolor{BrickRed}{,}\ \textbf{\textcolor{Blue}{const}}\ point\ \textcolor{BrickRed}{\&}b\textcolor{BrickRed}{,}\ \textbf{\textcolor{Blue}{const}}\ point\ \textcolor{BrickRed}{\&}pnt\textcolor{BrickRed}{)}\textcolor{Red}{\{} \\
\mbox{}\ \ \textcolor{ForestGreen}{double}\ u\ \textcolor{BrickRed}{=}\ \textcolor{BrickRed}{((}pnt\textcolor{BrickRed}{.}x\ \textcolor{BrickRed}{-}\ a\textcolor{BrickRed}{.}x\textcolor{BrickRed}{)*(}b\textcolor{BrickRed}{.}x\ \textcolor{BrickRed}{-}\ a\textcolor{BrickRed}{.}x\textcolor{BrickRed}{)}\ \textcolor{BrickRed}{+}\ \textcolor{BrickRed}{(}pnt\textcolor{BrickRed}{.}y\ \textcolor{BrickRed}{-}\ a\textcolor{BrickRed}{.}y\textcolor{BrickRed}{)*(}b\textcolor{BrickRed}{.}y\ \textcolor{BrickRed}{-}\ a\textcolor{BrickRed}{.}y\textcolor{BrickRed}{))}\ \textcolor{BrickRed}{/}\ \textbf{\textcolor{Black}{distsqr}}\textcolor{BrickRed}{(}a\textcolor{BrickRed}{,}\ b\textcolor{BrickRed}{);} \\
\mbox{}\ \ point\ intersection\textcolor{BrickRed}{;} \\
\mbox{}\ \ intersection\textcolor{BrickRed}{.}x\ \textcolor{BrickRed}{=}\ a\textcolor{BrickRed}{.}x\ \textcolor{BrickRed}{+}\ u\textcolor{BrickRed}{*(}b\textcolor{BrickRed}{.}x\ \textcolor{BrickRed}{-}\ a\textcolor{BrickRed}{.}x\textcolor{BrickRed}{);} \\
\mbox{}\ \ intersection\textcolor{BrickRed}{.}y\ \textcolor{BrickRed}{=}\ a\textcolor{BrickRed}{.}y\ \textcolor{BrickRed}{+}\ u\textcolor{BrickRed}{*(}b\textcolor{BrickRed}{.}y\ \textcolor{BrickRed}{-}\ a\textcolor{BrickRed}{.}y\textcolor{BrickRed}{);} \\
\mbox{}\ \ \textbf{\textcolor{Blue}{return}}\ \textbf{\textcolor{Black}{dist}}\textcolor{BrickRed}{(}pnt\textcolor{BrickRed}{,}\ intersection\textcolor{BrickRed}{);} \\
\mbox{}\textcolor{Red}{\}} \\

} \normalfont\normalsize
%.tex

\subsection{Determinar si un polígono es convexo}
% Generator: GNU source-highlight, by Lorenzo Bettini, http://www.gnu.org/software/src-highlite

{\ttfamily \raggedright {
\noindent
\mbox{}\textit{\textcolor{Brown}{/*}} \\
\mbox{}\textit{\textcolor{Brown}{\ \ Returns\ positive\ if\ a-b-c\ make\ a\ left\ turn.}} \\
\mbox{}\textit{\textcolor{Brown}{\ \ Returns\ negative\ if\ a-b-c\ make\ a\ right\ turn.}} \\
\mbox{}\textit{\textcolor{Brown}{\ \ Returns\ 0.0\ if\ a-b-c\ are\ colineal.}} \\
\mbox{}\textit{\textcolor{Brown}{\ */}} \\
\mbox{}\textcolor{ForestGreen}{double}\ \textbf{\textcolor{Black}{turn}}\textcolor{BrickRed}{(}\textbf{\textcolor{Blue}{const}}\ point\ \textcolor{BrickRed}{\&}a\textcolor{BrickRed}{,}\ \textbf{\textcolor{Blue}{const}}\ point\ \textcolor{BrickRed}{\&}b\textcolor{BrickRed}{,}\ \textbf{\textcolor{Blue}{const}}\ point\ \textcolor{BrickRed}{\&}c\textcolor{BrickRed}{)}\textcolor{Red}{\{} \\
\mbox{}\ \ \textcolor{ForestGreen}{double}\ z\ \textcolor{BrickRed}{=}\ \textcolor{BrickRed}{(}b\textcolor{BrickRed}{.}x\ \textcolor{BrickRed}{-}\ a\textcolor{BrickRed}{.}x\textcolor{BrickRed}{)*(}c\textcolor{BrickRed}{.}y\ \textcolor{BrickRed}{-}\ a\textcolor{BrickRed}{.}y\textcolor{BrickRed}{)}\ \textcolor{BrickRed}{-}\ \textcolor{BrickRed}{(}b\textcolor{BrickRed}{.}y\ \textcolor{BrickRed}{-}\ a\textcolor{BrickRed}{.}y\textcolor{BrickRed}{)*(}c\textcolor{BrickRed}{.}x\ \textcolor{BrickRed}{-}\ a\textcolor{BrickRed}{.}x\textcolor{BrickRed}{);} \\
\mbox{}\ \ \textbf{\textcolor{Blue}{if}}\ \textcolor{BrickRed}{(}\textbf{\textcolor{Black}{fabs}}\textcolor{BrickRed}{(}z\textcolor{BrickRed}{)}\ \textcolor{BrickRed}{$<$}\ \textcolor{Purple}{1e-9}\textcolor{BrickRed}{)}\ \textbf{\textcolor{Blue}{return}}\ \textcolor{Purple}{0.0}\textcolor{BrickRed}{;} \\
\mbox{}\ \ \textbf{\textcolor{Blue}{return}}\ z\textcolor{BrickRed}{;} \\
\mbox{}\textcolor{Red}{\}} \\
\mbox{} \\
\mbox{}\textit{\textcolor{Brown}{/*}} \\
\mbox{}\textit{\textcolor{Brown}{\ \ Returns\ true\ if\ polygon\ p\ is\ convex.}} \\
\mbox{}\textit{\textcolor{Brown}{\ \ False\ if\ it's\ concave\ or\ it\ can't\ be\ determined}} \\
\mbox{}\textit{\textcolor{Brown}{\ \ (For\ example,\ if\ all\ points\ are\ colineal\ we\ can't\ }} \\
\mbox{}\textit{\textcolor{Brown}{\ \ make\ a\ choice).}} \\
\mbox{}\textit{\textcolor{Brown}{\ */}} \\
\mbox{}\textcolor{ForestGreen}{bool}\ \textbf{\textcolor{Black}{isConvexPolygon}}\textcolor{BrickRed}{(}\textbf{\textcolor{Blue}{const}}\ vector\textcolor{BrickRed}{$<$}point\textcolor{BrickRed}{$>$}\ \textcolor{BrickRed}{\&}p\textcolor{BrickRed}{)}\textcolor{Red}{\{} \\
\mbox{}\ \ \textcolor{ForestGreen}{int}\ mask\ \textcolor{BrickRed}{=}\ \textcolor{Purple}{0}\textcolor{BrickRed}{;} \\
\mbox{}\ \ \textcolor{ForestGreen}{int}\ n\ \textcolor{BrickRed}{=}\ p\textcolor{BrickRed}{.}\textbf{\textcolor{Black}{size}}\textcolor{BrickRed}{();} \\
\mbox{}\ \ \textbf{\textcolor{Blue}{for}}\ \textcolor{BrickRed}{(}\textcolor{ForestGreen}{int}\ i\textcolor{BrickRed}{=}\textcolor{Purple}{0}\textcolor{BrickRed}{;}\ i\textcolor{BrickRed}{$<$}n\textcolor{BrickRed}{;}\ \textcolor{BrickRed}{++}i\textcolor{BrickRed}{)}\textcolor{Red}{\{} \\
\mbox{}\ \ \ \ \textcolor{ForestGreen}{int}\ j\textcolor{BrickRed}{=(}i\textcolor{BrickRed}{+}\textcolor{Purple}{1}\textcolor{BrickRed}{)\%}n\textcolor{BrickRed}{;} \\
\mbox{}\ \ \ \ \textcolor{ForestGreen}{int}\ k\textcolor{BrickRed}{=(}i\textcolor{BrickRed}{+}\textcolor{Purple}{2}\textcolor{BrickRed}{)\%}n\textcolor{BrickRed}{;} \\
\mbox{}\ \ \ \ \textcolor{ForestGreen}{double}\ z\ \textcolor{BrickRed}{=}\ \textbf{\textcolor{Black}{turn}}\textcolor{BrickRed}{(}p\textcolor{BrickRed}{[}i\textcolor{BrickRed}{],}\ p\textcolor{BrickRed}{[}j\textcolor{BrickRed}{],}\ p\textcolor{BrickRed}{[}k\textcolor{BrickRed}{]);} \\
\mbox{}\ \ \ \ \textbf{\textcolor{Blue}{if}}\ \textcolor{BrickRed}{(}z\ \textcolor{BrickRed}{$<$}\ \textcolor{Purple}{0.0}\textcolor{BrickRed}{)}\textcolor{Red}{\{} \\
\mbox{}\ \ \ \ \ \ mask\ \textcolor{BrickRed}{$|$=}\ \textcolor{Purple}{1}\textcolor{BrickRed}{;} \\
\mbox{}\ \ \ \ \textcolor{Red}{\}}\textbf{\textcolor{Blue}{else}}\ \textbf{\textcolor{Blue}{if}}\ \textcolor{BrickRed}{(}z\ \textcolor{BrickRed}{$>$}\ \textcolor{Purple}{0.0}\textcolor{BrickRed}{)}\textcolor{Red}{\{} \\
\mbox{}\ \ \ \ \ \ mask\ \textcolor{BrickRed}{$|$=}\ \textcolor{Purple}{2}\textcolor{BrickRed}{;} \\
\mbox{}\ \ \ \ \textcolor{Red}{\}} \\
\mbox{}\ \ \ \ \textbf{\textcolor{Blue}{if}}\ \textcolor{BrickRed}{(}mask\ \textcolor{BrickRed}{==}\ \textcolor{Purple}{3}\textcolor{BrickRed}{)}\ \textbf{\textcolor{Blue}{return}}\ \textbf{\textcolor{Blue}{false}}\textcolor{BrickRed}{;} \\
\mbox{}\ \ \textcolor{Red}{\}} \\
\mbox{}\ \ \textbf{\textcolor{Blue}{return}}\ mask\ \textcolor{BrickRed}{!=}\ \textcolor{Purple}{0}\textcolor{BrickRed}{;} \\
\mbox{}\textcolor{Red}{\}} \\

} \normalfont\normalsize
%.tex

\subsection{Determinar si un punto está dentro de un polígono convexo}
% Generator: GNU source-highlight, by Lorenzo Bettini, http://www.gnu.org/software/src-highlite

{\ttfamily \raggedright {
\noindent
\mbox{}\textit{\textcolor{Brown}{/*}} \\
\mbox{}\textit{\textcolor{Brown}{\ \ Returns\ true\ if\ point\ a\ is\ inside\ convex\ polygon\ p.}} \\
\mbox{}\textit{\textcolor{Brown}{\ \ Note\ that\ if\ point\ a\ lies\ on\ the\ border\ of\ p\ it}} \\
\mbox{}\textit{\textcolor{Brown}{\ \ is\ considered\ outside.}} \\
\mbox{} \\
\mbox{}\textit{\textcolor{Brown}{\ \ We\ assume\ p\ is\ convex!\ The\ result\ is\ useless\ if\ p}} \\
\mbox{}\textit{\textcolor{Brown}{\ \ is\ concave.}} \\
\mbox{}\textit{\textcolor{Brown}{\ */}} \\
\mbox{}\textcolor{ForestGreen}{bool}\ \textbf{\textcolor{Black}{insideConvexPolygon}}\textcolor{BrickRed}{(}\textbf{\textcolor{Blue}{const}}\ vector\textcolor{BrickRed}{$<$}point\textcolor{BrickRed}{$>$}\ \textcolor{BrickRed}{\&}p\textcolor{BrickRed}{,}\ \textbf{\textcolor{Blue}{const}}\ point\ \textcolor{BrickRed}{\&}a\textcolor{BrickRed}{)}\textcolor{Red}{\{} \\
\mbox{}\ \ \textcolor{ForestGreen}{int}\ mask\ \textcolor{BrickRed}{=}\ \textcolor{Purple}{0}\textcolor{BrickRed}{;} \\
\mbox{}\ \ \textbf{\textcolor{Blue}{for}}\ \textcolor{BrickRed}{(}\textcolor{ForestGreen}{int}\ i\textcolor{BrickRed}{=}\textcolor{Purple}{0}\textcolor{BrickRed}{;}\ i\textcolor{BrickRed}{$<$}n\textcolor{BrickRed}{;}\ \textcolor{BrickRed}{++}i\textcolor{BrickRed}{)}\textcolor{Red}{\{} \\
\mbox{}\ \ \ \ \textcolor{ForestGreen}{int}\ j\ \textcolor{BrickRed}{=}\ \textcolor{BrickRed}{(}i\textcolor{BrickRed}{+}\textcolor{Purple}{1}\textcolor{BrickRed}{)\%}n\textcolor{BrickRed}{;} \\
\mbox{}\ \ \ \ \textcolor{ForestGreen}{double}\ z\ \textcolor{BrickRed}{=}\ \textbf{\textcolor{Black}{turn}}\textcolor{BrickRed}{(}p\textcolor{BrickRed}{[}i\textcolor{BrickRed}{],}\ p\textcolor{BrickRed}{[}j\textcolor{BrickRed}{],}\ a\textcolor{BrickRed}{);} \\
\mbox{}\ \ \ \ \textbf{\textcolor{Blue}{if}}\ \textcolor{BrickRed}{(}z\ \textcolor{BrickRed}{$<$}\ \textcolor{Purple}{0.0}\textcolor{BrickRed}{)}\textcolor{Red}{\{} \\
\mbox{}\ \ \ \ \ \ mask\ \textcolor{BrickRed}{$|$=}\ \textcolor{Purple}{1}\textcolor{BrickRed}{;} \\
\mbox{}\ \ \ \ \textcolor{Red}{\}}\textbf{\textcolor{Blue}{else}}\ \textbf{\textcolor{Blue}{if}}\ \textcolor{BrickRed}{(}z\ \textcolor{BrickRed}{$>$}\ \textcolor{Purple}{0.0}\textcolor{BrickRed}{)}\textcolor{Red}{\{} \\
\mbox{}\ \ \ \ \ \ mask\ \textcolor{BrickRed}{$|$=}\ \textcolor{Purple}{2}\textcolor{BrickRed}{;} \\
\mbox{}\ \ \ \ \textcolor{Red}{\}}\textbf{\textcolor{Blue}{else}}\ \textbf{\textcolor{Blue}{if}}\ \textcolor{BrickRed}{(}z\ \textcolor{BrickRed}{==}\ \textcolor{Purple}{0.0}\textcolor{BrickRed}{)}\ \textbf{\textcolor{Blue}{return}}\ \textbf{\textcolor{Blue}{false}}\textcolor{BrickRed}{;} \\
\mbox{}\ \ \ \ \textbf{\textcolor{Blue}{if}}\ \textcolor{BrickRed}{(}mask\ \textcolor{BrickRed}{==}\ \textcolor{Purple}{3}\textcolor{BrickRed}{)}\ \textbf{\textcolor{Blue}{return}}\ \textbf{\textcolor{Blue}{false}}\textcolor{BrickRed}{;} \\
\mbox{}\ \ \textcolor{Red}{\}} \\
\mbox{}\ \ \textbf{\textcolor{Blue}{return}}\ mask\ \textcolor{BrickRed}{!=}\ \textcolor{Purple}{0}\textcolor{BrickRed}{;} \\
\mbox{} \\
\mbox{}\textcolor{Red}{\}} \\

} \normalfont\normalsize
%.tex

\subsection{Determinar si un punto está dentro de un polígono cualquiera}
\small
\textbf{Field-testing:} 
\begin{itemize}
\item \emph{TopCoder} -  SRM 187 - Division 2 Hard - PointInPolygon
\end{itemize}
% Generator: GNU source-highlight, by Lorenzo Bettini, http://www.gnu.org/software/src-highlite

{\ttfamily \raggedright {
\noindent
\mbox{}\textit{\textcolor{Brown}{/*********}} \\
\mbox{}\textit{\textcolor{Brown}{\ *\ Point\ *}} \\
\mbox{}\textit{\textcolor{Brown}{\ *********}} \\
\mbox{}\textit{\textcolor{Brown}{\ *\ A\ simple\ point\ class\ used\ by\ some\ of\ the\ routines\ below.}} \\
\mbox{}\textit{\textcolor{Brown}{\ *\ Anything\ else\ that\ supports\ .x\ and\ .y\ will\ work\ just\ as}} \\
\mbox{}\textit{\textcolor{Brown}{\ *\ well.\ There\ are\ 2\ variants\ -\ double\ and\ int.}} \\
\mbox{}\textit{\textcolor{Brown}{\ **/}} \\
\mbox{}\textbf{\textcolor{Blue}{struct}}\ P\ \textcolor{Red}{\{}\ \textcolor{ForestGreen}{double}\ x\textcolor{BrickRed}{,}\ y\textcolor{BrickRed}{;}\ \textbf{\textcolor{Black}{P}}\textcolor{BrickRed}{()}\ \textcolor{Red}{\{\}}\textcolor{BrickRed}{;}\ \textbf{\textcolor{Black}{P}}\textcolor{BrickRed}{(}\ \textcolor{ForestGreen}{double}\ q\textcolor{BrickRed}{,}\ \textcolor{ForestGreen}{double}\ w\ \textcolor{BrickRed}{)}\ \textcolor{BrickRed}{:}\ \textbf{\textcolor{Black}{x}}\textcolor{BrickRed}{(}\ q\ \textcolor{BrickRed}{),}\ \textbf{\textcolor{Black}{y}}\textcolor{BrickRed}{(}\ w\ \textcolor{BrickRed}{)}\ \textcolor{Red}{\{\}}\ \textcolor{Red}{\}}\textcolor{BrickRed}{;} \\
\mbox{}\textbf{\textcolor{Blue}{struct}}\ P\ \textcolor{Red}{\{}\ \textcolor{ForestGreen}{int}\ x\textcolor{BrickRed}{,}\ y\textcolor{BrickRed}{;}\ \textbf{\textcolor{Black}{P}}\textcolor{BrickRed}{()}\ \textcolor{Red}{\{\}}\textcolor{BrickRed}{;}\ \textbf{\textcolor{Black}{P}}\textcolor{BrickRed}{(}\ \textcolor{ForestGreen}{int}\ q\textcolor{BrickRed}{,}\ \textcolor{ForestGreen}{int}\ w\ \textcolor{BrickRed}{)}\ \textcolor{BrickRed}{:}\ \textbf{\textcolor{Black}{x}}\textcolor{BrickRed}{(}\ q\ \textcolor{BrickRed}{),}\ \textbf{\textcolor{Black}{y}}\textcolor{BrickRed}{(}\ w\ \textcolor{BrickRed}{)}\ \textcolor{Red}{\{\}}\ \textcolor{Red}{\}}\textcolor{BrickRed}{;} \\
\mbox{} \\
\mbox{}\textit{\textcolor{Brown}{/***************}} \\
\mbox{}\textit{\textcolor{Brown}{\ *\ Polar\ angle\ *}} \\
\mbox{}\textit{\textcolor{Brown}{\ ***************}} \\
\mbox{}\textit{\textcolor{Brown}{\ *\ Returns\ an\ angle\ in\ the\ range\ [0,\ 2*Pi)\ of\ a\ given\ Cartesian\ point.}} \\
\mbox{}\textit{\textcolor{Brown}{\ *\ If\ the\ point\ is\ (0,0),\ -1.0\ is\ returned.}} \\
\mbox{}\textit{\textcolor{Brown}{\ *\ REQUIRES:}} \\
\mbox{}\textit{\textcolor{Brown}{\ *\ include\ math.h}} \\
\mbox{}\textit{\textcolor{Brown}{\ *\ define\ EPS\ 0.000000001,\ or\ your\ choice}} \\
\mbox{}\textit{\textcolor{Brown}{\ *\ P\ has\ members\ x\ and\ y.}} \\
\mbox{}\textit{\textcolor{Brown}{\ **/}} \\
\mbox{}\textcolor{ForestGreen}{double}\ \textbf{\textcolor{Black}{polarAngle}}\textcolor{BrickRed}{(}\ P\ p\ \textcolor{BrickRed}{)} \\
\mbox{}\textcolor{Red}{\{} \\
\mbox{}\ \ \ \ \textbf{\textcolor{Blue}{if}}\textcolor{BrickRed}{(}\ \textbf{\textcolor{Black}{fabs}}\textcolor{BrickRed}{(}\ p\textcolor{BrickRed}{.}x\ \textcolor{BrickRed}{)}\ \textcolor{BrickRed}{$<$=}\ EPS\ \textcolor{BrickRed}{\&\&}\ \textbf{\textcolor{Black}{fabs}}\textcolor{BrickRed}{(}\ p\textcolor{BrickRed}{.}y\ \textcolor{BrickRed}{)}\ \textcolor{BrickRed}{$<$=}\ EPS\ \textcolor{BrickRed}{)}\ \textbf{\textcolor{Blue}{return}}\ \textcolor{BrickRed}{-}\textcolor{Purple}{1.0}\textcolor{BrickRed}{;} \\
\mbox{}\ \ \ \ \textbf{\textcolor{Blue}{if}}\textcolor{BrickRed}{(}\ \textbf{\textcolor{Black}{fabs}}\textcolor{BrickRed}{(}\ p\textcolor{BrickRed}{.}x\ \textcolor{BrickRed}{)}\ \textcolor{BrickRed}{$<$=}\ EPS\ \textcolor{BrickRed}{)}\ \textbf{\textcolor{Blue}{return}}\ \textcolor{BrickRed}{(}\ p\textcolor{BrickRed}{.}y\ \textcolor{BrickRed}{$>$}\ EPS\ \textcolor{BrickRed}{?}\ \textcolor{Purple}{1.0}\ \textcolor{BrickRed}{:}\ \textcolor{Purple}{3.0}\ \textcolor{BrickRed}{)}\ \textcolor{BrickRed}{*}\ \textbf{\textcolor{Black}{acos}}\textcolor{BrickRed}{(}\ \textcolor{Purple}{0}\ \textcolor{BrickRed}{);} \\
\mbox{}\ \ \ \ \textcolor{ForestGreen}{double}\ theta\ \textcolor{BrickRed}{=}\ \textbf{\textcolor{Black}{atan}}\textcolor{BrickRed}{(}\ \textcolor{Purple}{1.0}\ \textcolor{BrickRed}{*}\ p\textcolor{BrickRed}{.}y\ \textcolor{BrickRed}{/}\ p\textcolor{BrickRed}{.}x\ \textcolor{BrickRed}{);} \\
\mbox{}\ \ \ \ \textbf{\textcolor{Blue}{if}}\textcolor{BrickRed}{(}\ p\textcolor{BrickRed}{.}x\ \textcolor{BrickRed}{$>$}\ EPS\ \textcolor{BrickRed}{)}\ \textbf{\textcolor{Blue}{return}}\textcolor{BrickRed}{(}\ p\textcolor{BrickRed}{.}y\ \textcolor{BrickRed}{$>$=}\ \textcolor{BrickRed}{-}EPS\ \textcolor{BrickRed}{?}\ theta\ \textcolor{BrickRed}{:}\ \textcolor{BrickRed}{(}\ \textcolor{Purple}{4}\ \textcolor{BrickRed}{*}\ \textbf{\textcolor{Black}{acos}}\textcolor{BrickRed}{(}\ \textcolor{Purple}{0}\ \textcolor{BrickRed}{)}\ \textcolor{BrickRed}{+}\ theta\ \textcolor{BrickRed}{)}\ \textcolor{BrickRed}{);} \\
\mbox{}\ \ \ \ \textbf{\textcolor{Blue}{return}}\textcolor{BrickRed}{(}\ \textcolor{Purple}{2}\ \textcolor{BrickRed}{*}\ \textbf{\textcolor{Black}{acos}}\textcolor{BrickRed}{(}\ \textcolor{Purple}{0}\ \textcolor{BrickRed}{)}\ \textcolor{BrickRed}{+}\ theta\ \textcolor{BrickRed}{);} \\
\mbox{}\textcolor{Red}{\}} \\
\mbox{} \\
\mbox{}\textit{\textcolor{Brown}{/************************}} \\
\mbox{}\textit{\textcolor{Brown}{\ *\ Point\ inside\ polygon\ *}} \\
\mbox{}\textit{\textcolor{Brown}{\ ************************}} \\
\mbox{}\textit{\textcolor{Brown}{\ *\ Returns\ true\ iff\ p\ is\ inside\ poly.}} \\
\mbox{}\textit{\textcolor{Brown}{\ *\ PRE:\ The\ vertices\ of\ poly\ are\ ordered\ (either\ clockwise\ or}} \\
\mbox{}\textit{\textcolor{Brown}{\ *\ \ \ \ \ \ counter-clockwise.}} \\
\mbox{}\textit{\textcolor{Brown}{\ *\ POST:\ Modify\ code\ inside\ to\ handle\ the\ special\ case\ of\ "{}on\ an\ edge"{}.}} \\
\mbox{}\textit{\textcolor{Brown}{\ *\ REQUIRES:}} \\
\mbox{}\textit{\textcolor{Brown}{\ *\ polarAngle()}} \\
\mbox{}\textit{\textcolor{Brown}{\ *\ include\ math.h}} \\
\mbox{}\textit{\textcolor{Brown}{\ *\ include\ vector}} \\
\mbox{}\textit{\textcolor{Brown}{\ *\ define\ EPS\ 0.000000001,\ or\ your\ choice}} \\
\mbox{}\textit{\textcolor{Brown}{\ **/}} \\
\mbox{}\textcolor{ForestGreen}{bool}\ \textbf{\textcolor{Black}{pointInPoly}}\textcolor{BrickRed}{(}\ P\ p\textcolor{BrickRed}{,}\ vector\textcolor{BrickRed}{$<$}\ P\ \textcolor{BrickRed}{$>$}\ \textcolor{BrickRed}{\&}poly\ \textcolor{BrickRed}{)} \\
\mbox{}\textcolor{Red}{\{} \\
\mbox{}\ \ \ \ \textcolor{ForestGreen}{int}\ n\ \textcolor{BrickRed}{=}\ poly\textcolor{BrickRed}{.}\textbf{\textcolor{Black}{size}}\textcolor{BrickRed}{();} \\
\mbox{}\ \ \ \ \textcolor{ForestGreen}{double}\ ang\ \textcolor{BrickRed}{=}\ \textcolor{Purple}{0.0}\textcolor{BrickRed}{;} \\
\mbox{}\ \ \ \ \textbf{\textcolor{Blue}{for}}\textcolor{BrickRed}{(}\ \textcolor{ForestGreen}{int}\ i\ \textcolor{BrickRed}{=}\ n\ \textcolor{BrickRed}{-}\ \textcolor{Purple}{1}\textcolor{BrickRed}{,}\ j\ \textcolor{BrickRed}{=}\ \textcolor{Purple}{0}\textcolor{BrickRed}{;}\ j\ \textcolor{BrickRed}{$<$}\ n\textcolor{BrickRed}{;}\ i\ \textcolor{BrickRed}{=}\ j\textcolor{BrickRed}{++}\ \textcolor{BrickRed}{)} \\
\mbox{}\ \ \ \ \textcolor{Red}{\{} \\
\mbox{}\ \ \ \ \ \ \ \ P\ \textbf{\textcolor{Black}{v}}\textcolor{BrickRed}{(}\ poly\textcolor{BrickRed}{[}i\textcolor{BrickRed}{].}x\ \textcolor{BrickRed}{-}\ p\textcolor{BrickRed}{.}x\textcolor{BrickRed}{,}\ poly\textcolor{BrickRed}{[}i\textcolor{BrickRed}{].}y\ \textcolor{BrickRed}{-}\ p\textcolor{BrickRed}{.}y\ \textcolor{BrickRed}{);} \\
\mbox{}\ \ \ \ \ \ \ \ P\ \textbf{\textcolor{Black}{w}}\textcolor{BrickRed}{(}\ poly\textcolor{BrickRed}{[}j\textcolor{BrickRed}{].}x\ \textcolor{BrickRed}{-}\ p\textcolor{BrickRed}{.}x\textcolor{BrickRed}{,}\ poly\textcolor{BrickRed}{[}j\textcolor{BrickRed}{].}y\ \textcolor{BrickRed}{-}\ p\textcolor{BrickRed}{.}y\ \textcolor{BrickRed}{);} \\
\mbox{}\ \ \ \ \ \ \ \ \textcolor{ForestGreen}{double}\ va\ \textcolor{BrickRed}{=}\ \textbf{\textcolor{Black}{polarAngle}}\textcolor{BrickRed}{(}\ v\ \textcolor{BrickRed}{);} \\
\mbox{}\ \ \ \ \ \ \ \ \textcolor{ForestGreen}{double}\ wa\ \textcolor{BrickRed}{=}\ \textbf{\textcolor{Black}{polarAngle}}\textcolor{BrickRed}{(}\ w\ \textcolor{BrickRed}{);} \\
\mbox{}\ \ \ \ \ \ \ \ \textcolor{ForestGreen}{double}\ xx\ \textcolor{BrickRed}{=}\ wa\ \textcolor{BrickRed}{-}\ va\textcolor{BrickRed}{;} \\
\mbox{}\ \ \ \ \ \ \ \ \textbf{\textcolor{Blue}{if}}\textcolor{BrickRed}{(}\ va\ \textcolor{BrickRed}{$<$}\ \textcolor{BrickRed}{-}\textcolor{Purple}{0.5}\ \textcolor{BrickRed}{$|$$|$}\ wa\ \textcolor{BrickRed}{$<$}\ \textcolor{BrickRed}{-}\textcolor{Purple}{0.5}\ \textcolor{BrickRed}{$|$$|$}\ \textbf{\textcolor{Black}{fabs}}\textcolor{BrickRed}{(}\ \textbf{\textcolor{Black}{fabs}}\textcolor{BrickRed}{(}\ xx\ \textcolor{BrickRed}{)}\ \textcolor{BrickRed}{-}\ \textcolor{Purple}{2}\ \textcolor{BrickRed}{*}\ \textbf{\textcolor{Black}{acos}}\textcolor{BrickRed}{(}\ \textcolor{Purple}{0}\ \textcolor{BrickRed}{)}\ \textcolor{BrickRed}{)}\ \textcolor{BrickRed}{$<$}\ EPS\ \textcolor{BrickRed}{)} \\
\mbox{}\ \ \ \ \ \ \ \ \textcolor{Red}{\{} \\
\mbox{}\ \ \ \ \ \ \ \ \ \ \ \ \textit{\textcolor{Brown}{//\ POINT\ IS\ ON\ THE\ EDGE}} \\
\mbox{}\ \ \ \ \ \ \ \ \ \ \ \ \textbf{\textcolor{Black}{assert}}\textcolor{BrickRed}{(}\ \textbf{\textcolor{Blue}{false}}\ \textcolor{BrickRed}{);} \\
\mbox{}\ \ \ \ \ \ \ \ \ \ \ \ ang\ \textcolor{BrickRed}{+=}\ \textcolor{Purple}{2}\ \textcolor{BrickRed}{*}\ \textbf{\textcolor{Black}{acos}}\textcolor{BrickRed}{(}\ \textcolor{Purple}{0}\ \textcolor{BrickRed}{);} \\
\mbox{}\ \ \ \ \ \ \ \ \ \ \ \ \textbf{\textcolor{Blue}{continue}}\textcolor{BrickRed}{;} \\
\mbox{}\ \ \ \ \ \ \ \ \textcolor{Red}{\}} \\
\mbox{}\ \ \ \ \ \ \ \ \textbf{\textcolor{Blue}{if}}\textcolor{BrickRed}{(}\ xx\ \textcolor{BrickRed}{$<$}\ \textcolor{BrickRed}{-}\textcolor{Purple}{2}\ \textcolor{BrickRed}{*}\ \textbf{\textcolor{Black}{acos}}\textcolor{BrickRed}{(}\ \textcolor{Purple}{0}\ \textcolor{BrickRed}{)}\ \textcolor{BrickRed}{)}\ ang\ \textcolor{BrickRed}{+=}\ xx\ \textcolor{BrickRed}{+}\ \textcolor{Purple}{4}\ \textcolor{BrickRed}{*}\ \textbf{\textcolor{Black}{acos}}\textcolor{BrickRed}{(}\ \textcolor{Purple}{0}\ \textcolor{BrickRed}{);} \\
\mbox{}\ \ \ \ \ \ \ \ \textbf{\textcolor{Blue}{else}}\ \textbf{\textcolor{Blue}{if}}\textcolor{BrickRed}{(}\ xx\ \textcolor{BrickRed}{$>$}\ \textcolor{Purple}{2}\ \textcolor{BrickRed}{*}\ \textbf{\textcolor{Black}{acos}}\textcolor{BrickRed}{(}\ \textcolor{Purple}{0}\ \textcolor{BrickRed}{)}\ \textcolor{BrickRed}{)}\ ang\ \textcolor{BrickRed}{+=}\ xx\ \textcolor{BrickRed}{-}\ \textcolor{Purple}{4}\ \textcolor{BrickRed}{*}\ \textbf{\textcolor{Black}{acos}}\textcolor{BrickRed}{(}\ \textcolor{Purple}{0}\ \textcolor{BrickRed}{);} \\
\mbox{}\ \ \ \ \ \ \ \ \textbf{\textcolor{Blue}{else}}\ ang\ \textcolor{BrickRed}{+=}\ xx\textcolor{BrickRed}{;} \\
\mbox{}\ \ \ \ \textcolor{Red}{\}} \\
\mbox{}\ \ \ \ \textbf{\textcolor{Blue}{return}}\textcolor{BrickRed}{(}\ ang\ \textcolor{BrickRed}{*}\ ang\ \textcolor{BrickRed}{$>$}\ \textcolor{Purple}{1.0}\ \textcolor{BrickRed}{);} \\
\mbox{}\textcolor{Red}{\}} \\

} \normalfont\normalsize
%.tex

%---------------------------------------------------------------
\section{Estructuras de datos}
\subsection{Árboles de Fenwick ó Binary indexed trees}

Se tiene un arreglo $\{a_0, a_1, \cdots, a_{n-1}\}$. Los árboles
de Fenwick permiten encontrar $ \displaystyle \sum_{k=i}^{j} a_k $ en orden $O(\log_{2}{n})$ para parejas de $(i, j)$ con $i \leq j$. De la misma manera, permiten sumarle una cantidad a un $a_i$ también en tiempo $O(log_{2}{n})$.
\medskip
% Generator: GNU source-highlight, by Lorenzo Bettini, http://www.gnu.org/software/src-highlite

{\ttfamily \raggedright {
\noindent
\mbox{}\textbf{\textcolor{Blue}{class}}\ FenwickTree\textcolor{Red}{\{} \\
\mbox{}\ \ vector\textcolor{BrickRed}{$<$}\textcolor{ForestGreen}{long}\ \textcolor{ForestGreen}{long}\textcolor{BrickRed}{$>$}\ v\textcolor{BrickRed}{;} \\
\mbox{}\ \ \textcolor{ForestGreen}{int}\ maxSize\textcolor{BrickRed}{;} \\
\mbox{} \\
\mbox{}\textbf{\textcolor{Blue}{public}}\textcolor{BrickRed}{:} \\
\mbox{}\ \ \textbf{\textcolor{Black}{FenwickTree}}\textcolor{BrickRed}{(}\textcolor{ForestGreen}{int}\ $\_$maxSize\textcolor{BrickRed}{)}\ \textcolor{BrickRed}{:}\ \textbf{\textcolor{Black}{maxSize}}\textcolor{BrickRed}{(}$\_$maxSize\textcolor{BrickRed}{+}\textcolor{Purple}{1}\textcolor{BrickRed}{)}\ \textcolor{Red}{\{} \\
\mbox{}\ \ \ \ v\ \textcolor{BrickRed}{=}\ vector\textcolor{BrickRed}{$<$}\textcolor{ForestGreen}{long}\ \textcolor{ForestGreen}{long}\textcolor{BrickRed}{$>$(}maxSize\textcolor{BrickRed}{,}\ 0LL\textcolor{BrickRed}{);} \\
\mbox{}\ \ \textcolor{Red}{\}} \\
\mbox{} \\
\mbox{}\ \ \textcolor{ForestGreen}{void}\ \textbf{\textcolor{Black}{add}}\textcolor{BrickRed}{(}\textcolor{ForestGreen}{int}\ where\textcolor{BrickRed}{,}\ \textcolor{ForestGreen}{long}\ \textcolor{ForestGreen}{long}\ what\textcolor{BrickRed}{)}\textcolor{Red}{\{} \\
\mbox{}\ \ \ \ \textbf{\textcolor{Blue}{for}}\ \textcolor{BrickRed}{(}where\textcolor{BrickRed}{++;}\ where\ \textcolor{BrickRed}{$<$=}\ maxSize\textcolor{BrickRed}{;}\ where\ \textcolor{BrickRed}{+=}\ where\ \textcolor{BrickRed}{\&}\ \textcolor{BrickRed}{-}where\textcolor{BrickRed}{)}\textcolor{Red}{\{} \\
\mbox{}\ \ \ \ \ \ v\textcolor{BrickRed}{[}where\textcolor{BrickRed}{]}\ \textcolor{BrickRed}{+=}\ what\textcolor{BrickRed}{;} \\
\mbox{}\ \ \ \ \textcolor{Red}{\}} \\
\mbox{}\ \ \textcolor{Red}{\}} \\
\mbox{} \\
\mbox{}\ \ \textcolor{ForestGreen}{long}\ \textcolor{ForestGreen}{long}\ \textbf{\textcolor{Black}{query}}\textcolor{BrickRed}{(}\textcolor{ForestGreen}{int}\ where\textcolor{BrickRed}{)}\textcolor{Red}{\{} \\
\mbox{}\ \ \ \ \textcolor{ForestGreen}{long}\ \textcolor{ForestGreen}{long}\ sum\ \textcolor{BrickRed}{=}\ v\textcolor{BrickRed}{[}\textcolor{Purple}{0}\textcolor{BrickRed}{];} \\
\mbox{}\ \ \ \ \textbf{\textcolor{Blue}{for}}\ \textcolor{BrickRed}{(}where\textcolor{BrickRed}{++;}\ where\ \textcolor{BrickRed}{$>$}\ \textcolor{Purple}{0}\textcolor{BrickRed}{;}\ where\ \textcolor{BrickRed}{-=}\ where\ \textcolor{BrickRed}{\&}\ \textcolor{BrickRed}{-}where\textcolor{BrickRed}{)}\textcolor{Red}{\{} \\
\mbox{}\ \ \ \ \ \ sum\ \textcolor{BrickRed}{+=}\ v\textcolor{BrickRed}{[}where\textcolor{BrickRed}{];} \\
\mbox{}\ \ \ \ \textcolor{Red}{\}} \\
\mbox{}\ \ \ \ \textbf{\textcolor{Blue}{return}}\ sum\textcolor{BrickRed}{;} \\
\mbox{}\ \ \textcolor{Red}{\}} \\
\mbox{} \\
\mbox{}\ \ \textcolor{ForestGreen}{long}\ \textcolor{ForestGreen}{long}\ \textbf{\textcolor{Black}{query}}\textcolor{BrickRed}{(}\textcolor{ForestGreen}{int}\ from\textcolor{BrickRed}{,}\ \textcolor{ForestGreen}{int}\ to\textcolor{BrickRed}{)}\textcolor{Red}{\{} \\
\mbox{}\ \ \ \ \textbf{\textcolor{Blue}{return}}\ \textbf{\textcolor{Black}{query}}\textcolor{BrickRed}{(}to\textcolor{BrickRed}{)}\ \textcolor{BrickRed}{-}\ \textbf{\textcolor{Black}{query}}\textcolor{BrickRed}{(}from\textcolor{BrickRed}{-}\textcolor{Purple}{1}\textcolor{BrickRed}{);} \\
\mbox{}\ \ \textcolor{Red}{\}} \\
\mbox{} \\
\mbox{}\textcolor{Red}{\}}\textcolor{BrickRed}{;} \\

} \normalfont\normalsize
%.tex

\subsection{Segment tree}
% Generator: GNU source-highlight, by Lorenzo Bettini, http://www.gnu.org/software/src-highlite
{\ttfamily \raggedright {
\noindent
\mbox{}\textbf{\textcolor{RoyalBlue}{\#include}}\ \texttt{\textcolor{Red}{$<$vector$>$}} \\
\mbox{}\textbf{\textcolor{Blue}{using}}\ \textbf{\textcolor{Blue}{namespace}}\ std\textcolor{BrickRed}{;} \\
\mbox{} \\
\mbox{}\textbf{\textcolor{Blue}{class}}\ \textcolor{ForestGreen}{SegmentTree}\textcolor{Red}{\{} \\
\mbox{}\textbf{\textcolor{Blue}{public}}\textcolor{BrickRed}{:} \\
\mbox{}\ \ vector\textcolor{BrickRed}{$<$}\textcolor{ForestGreen}{int}\textcolor{BrickRed}{$>$}\ arr\textcolor{BrickRed}{,}\ tree\textcolor{BrickRed}{;} \\
\mbox{}\ \ \textcolor{ForestGreen}{int}\ n\textcolor{BrickRed}{;} \\
\mbox{} \\
\mbox{}\ \ \textbf{\textcolor{Black}{SegmentTree}}\textcolor{BrickRed}{()}\textcolor{Red}{\{\}} \\
\mbox{}\ \ \textbf{\textcolor{Black}{SegmentTree}}\textcolor{BrickRed}{(}\textbf{\textcolor{Blue}{const}}\ vector\textcolor{BrickRed}{$<$}\textcolor{ForestGreen}{int}\textcolor{BrickRed}{$>$}\ \textcolor{BrickRed}{\&}arr\textcolor{BrickRed}{)}\ \textcolor{BrickRed}{:}\ \textbf{\textcolor{Black}{arr}}\textcolor{BrickRed}{(}arr\textcolor{BrickRed}{)}\ \textcolor{Red}{\{} \\
\mbox{}\ \ \ \ \textbf{\textcolor{Black}{initialize}}\textcolor{BrickRed}{();} \\
\mbox{}\ \ \textcolor{Red}{\}} \\
\mbox{} \\
\mbox{}\ \ \textit{\textcolor{Brown}{//must\ be\ called\ after\ assigning\ a\ new\ arr.}} \\
\mbox{}\ \ \textcolor{ForestGreen}{void}\ \textbf{\textcolor{Black}{initialize}}\textcolor{BrickRed}{()}\textcolor{Red}{\{} \\
\mbox{}\ \ \ \ n\ \textcolor{BrickRed}{=}\ arr\textcolor{BrickRed}{.}\textbf{\textcolor{Black}{size}}\textcolor{BrickRed}{();} \\
\mbox{}\ \ \ \ tree\textcolor{BrickRed}{.}\textbf{\textcolor{Black}{resize}}\textcolor{BrickRed}{(}\textcolor{Purple}{4}\textcolor{BrickRed}{*}n\ \textcolor{BrickRed}{+}\ \textcolor{Purple}{1}\textcolor{BrickRed}{);} \\
\mbox{}\ \ \ \ \textbf{\textcolor{Black}{initialize}}\textcolor{BrickRed}{(}\textcolor{Purple}{0}\textcolor{BrickRed}{,}\ \textcolor{Purple}{0}\textcolor{BrickRed}{,}\ n\textcolor{BrickRed}{-}\textcolor{Purple}{1}\textcolor{BrickRed}{);} \\
\mbox{}\ \ \textcolor{Red}{\}} \\
\mbox{} \\
\mbox{}\ \ \textcolor{ForestGreen}{int}\ \textbf{\textcolor{Black}{query}}\textcolor{BrickRed}{(}\textcolor{ForestGreen}{int}\ query$\_$left\textcolor{BrickRed}{,}\ \textcolor{ForestGreen}{int}\ query$\_$right\textcolor{BrickRed}{)}\ \textbf{\textcolor{Blue}{const}}\textcolor{Red}{\{} \\
\mbox{}\ \ \ \ \textbf{\textcolor{Blue}{return}}\ \textbf{\textcolor{Black}{query}}\textcolor{BrickRed}{(}\textcolor{Purple}{0}\textcolor{BrickRed}{,}\ \textcolor{Purple}{0}\textcolor{BrickRed}{,}\ n\textcolor{BrickRed}{-}\textcolor{Purple}{1}\textcolor{BrickRed}{,}\ query$\_$left\textcolor{BrickRed}{,}\ query$\_$right\textcolor{BrickRed}{);} \\
\mbox{}\ \ \textcolor{Red}{\}} \\
\mbox{} \\
\mbox{}\ \ \textcolor{ForestGreen}{void}\ \textbf{\textcolor{Black}{update}}\textcolor{BrickRed}{(}\textcolor{ForestGreen}{int}\ where\textcolor{BrickRed}{,}\ \textcolor{ForestGreen}{int}\ what\textcolor{BrickRed}{)}\textcolor{Red}{\{} \\
\mbox{}\ \ \ \ \textbf{\textcolor{Black}{update}}\textcolor{BrickRed}{(}\textcolor{Purple}{0}\textcolor{BrickRed}{,}\ \textcolor{Purple}{0}\textcolor{BrickRed}{,}\ n\textcolor{BrickRed}{-}\textcolor{Purple}{1}\textcolor{BrickRed}{,}\ where\textcolor{BrickRed}{,}\ what\textcolor{BrickRed}{);} \\
\mbox{}\ \ \textcolor{Red}{\}} \\
\mbox{} \\
\mbox{}\textbf{\textcolor{Blue}{private}}\textcolor{BrickRed}{:} \\
\mbox{}\ \ \textcolor{ForestGreen}{int}\ \textbf{\textcolor{Black}{initialize}}\textcolor{BrickRed}{(}\textcolor{ForestGreen}{int}\ node\textcolor{BrickRed}{,}\ \textcolor{ForestGreen}{int}\ node$\_$left\textcolor{BrickRed}{,}\ \textcolor{ForestGreen}{int}\ node$\_$right\textcolor{BrickRed}{);} \\
\mbox{}\ \ \textcolor{ForestGreen}{int}\ \textbf{\textcolor{Black}{query}}\textcolor{BrickRed}{(}\textcolor{ForestGreen}{int}\ node\textcolor{BrickRed}{,}\ \textcolor{ForestGreen}{int}\ node$\_$left\textcolor{BrickRed}{,}\ \textcolor{ForestGreen}{int}\ node$\_$right\textcolor{BrickRed}{,}\ \textcolor{ForestGreen}{int}\ query$\_$left\textcolor{BrickRed}{,}\ \textcolor{ForestGreen}{int}\ query$\_$right\textcolor{BrickRed}{)}\ \textbf{\textcolor{Blue}{const}}\textcolor{BrickRed}{;} \\
\mbox{}\ \ \textcolor{ForestGreen}{void}\ \textbf{\textcolor{Black}{update}}\textcolor{BrickRed}{(}\textcolor{ForestGreen}{int}\ node\textcolor{BrickRed}{,}\ \textcolor{ForestGreen}{int}\ node$\_$left\textcolor{BrickRed}{,}\ \textcolor{ForestGreen}{int}\ node$\_$right\textcolor{BrickRed}{,}\ \textcolor{ForestGreen}{int}\ where\textcolor{BrickRed}{,}\ \textcolor{ForestGreen}{int}\ what\textcolor{BrickRed}{);} \\
\mbox{}\textcolor{Red}{\}}\textcolor{BrickRed}{;} \\
\mbox{} \\
\mbox{}\textcolor{ForestGreen}{int}\ SegmentTree\textcolor{BrickRed}{::}\textbf{\textcolor{Black}{initialize}}\textcolor{BrickRed}{(}\textcolor{ForestGreen}{int}\ node\textcolor{BrickRed}{,}\ \textcolor{ForestGreen}{int}\ node$\_$left\textcolor{BrickRed}{,}\ \textcolor{ForestGreen}{int}\ node$\_$right\textcolor{BrickRed}{)}\textcolor{Red}{\{} \\
\mbox{}\ \ \textbf{\textcolor{Blue}{if}}\ \textcolor{BrickRed}{(}node$\_$left\ \textcolor{BrickRed}{==}\ node$\_$right\textcolor{BrickRed}{)}\textcolor{Red}{\{} \\
\mbox{}\ \ \ \ tree\textcolor{BrickRed}{[}node\textcolor{BrickRed}{]}\ \textcolor{BrickRed}{=}\ node$\_$left\textcolor{BrickRed}{;} \\
\mbox{}\ \ \ \ \textbf{\textcolor{Blue}{return}}\ tree\textcolor{BrickRed}{[}node\textcolor{BrickRed}{];} \\
\mbox{}\ \ \textcolor{Red}{\}} \\
\mbox{}\ \ \textcolor{ForestGreen}{int}\ half\ \textcolor{BrickRed}{=}\ \textcolor{BrickRed}{(}node$\_$left\ \textcolor{BrickRed}{+}\ node$\_$right\textcolor{BrickRed}{)}\ \textcolor{BrickRed}{/}\ \textcolor{Purple}{2}\textcolor{BrickRed}{;} \\
\mbox{}\ \ \textcolor{ForestGreen}{int}\ ans$\_$left\ \textcolor{BrickRed}{=}\ \textbf{\textcolor{Black}{initialize}}\textcolor{BrickRed}{(}\textcolor{Purple}{2}\textcolor{BrickRed}{*}node\textcolor{BrickRed}{+}\textcolor{Purple}{1}\textcolor{BrickRed}{,}\ node$\_$left\textcolor{BrickRed}{,}\ half\textcolor{BrickRed}{);} \\
\mbox{}\ \ \textcolor{ForestGreen}{int}\ ans$\_$right\ \textcolor{BrickRed}{=}\ \textbf{\textcolor{Black}{initialize}}\textcolor{BrickRed}{(}\textcolor{Purple}{2}\textcolor{BrickRed}{*}node\textcolor{BrickRed}{+}\textcolor{Purple}{2}\textcolor{BrickRed}{,}\ half\textcolor{BrickRed}{+}\textcolor{Purple}{1}\textcolor{BrickRed}{,}\ node$\_$right\textcolor{BrickRed}{);} \\
\mbox{} \\
\mbox{}\ \ \textbf{\textcolor{Blue}{if}}\ \textcolor{BrickRed}{(}arr\textcolor{BrickRed}{[}ans$\_$left\textcolor{BrickRed}{]}\ \textcolor{BrickRed}{$<$=}\ arr\textcolor{BrickRed}{[}ans$\_$right\textcolor{BrickRed}{])}\textcolor{Red}{\{} \\
\mbox{}\ \ \ \ tree\textcolor{BrickRed}{[}node\textcolor{BrickRed}{]}\ \textcolor{BrickRed}{=}\ ans$\_$left\textcolor{BrickRed}{;} \\
\mbox{}\ \ \textcolor{Red}{\}}\textbf{\textcolor{Blue}{else}}\textcolor{Red}{\{} \\
\mbox{}\ \ \ \ tree\textcolor{BrickRed}{[}node\textcolor{BrickRed}{]}\ \textcolor{BrickRed}{=}\ ans$\_$right\textcolor{BrickRed}{;} \\
\mbox{}\ \ \textcolor{Red}{\}} \\
\mbox{}\ \ \textbf{\textcolor{Blue}{return}}\ tree\textcolor{BrickRed}{[}node\textcolor{BrickRed}{];} \\
\mbox{}\textcolor{Red}{\}} \\
\mbox{} \\
\mbox{}\textcolor{ForestGreen}{int}\ SegmentTree\textcolor{BrickRed}{::}\textbf{\textcolor{Black}{query}}\textcolor{BrickRed}{(}\textcolor{ForestGreen}{int}\ node\textcolor{BrickRed}{,}\ \textcolor{ForestGreen}{int}\ node$\_$left\textcolor{BrickRed}{,}\ \textcolor{ForestGreen}{int}\ node$\_$right\textcolor{BrickRed}{,}\ \textcolor{ForestGreen}{int}\ query$\_$left\textcolor{BrickRed}{,}\ \textcolor{ForestGreen}{int}\ query$\_$right\textcolor{BrickRed}{)}\ \textbf{\textcolor{Blue}{const}}\textcolor{Red}{\{} \\
\mbox{}\ \ \textbf{\textcolor{Blue}{if}}\ \textcolor{BrickRed}{(}node$\_$right\ \textcolor{BrickRed}{$<$}\ query$\_$left\ \textcolor{BrickRed}{$|$$|$}\ query$\_$right\ \textcolor{BrickRed}{$<$}\ node$\_$left\textcolor{BrickRed}{)}\ \textbf{\textcolor{Blue}{return}}\ \textcolor{BrickRed}{-}\textcolor{Purple}{1}\textcolor{BrickRed}{;} \\
\mbox{}\ \ \textbf{\textcolor{Blue}{if}}\ \textcolor{BrickRed}{(}query$\_$left\ \textcolor{BrickRed}{$<$=}\ node$\_$left\ \textcolor{BrickRed}{\&\&}\ node$\_$right\ \textcolor{BrickRed}{$<$=}\ query$\_$right\textcolor{BrickRed}{)}\ \textbf{\textcolor{Blue}{return}}\ tree\textcolor{BrickRed}{[}node\textcolor{BrickRed}{];} \\
\mbox{} \\
\mbox{}\ \ \textcolor{ForestGreen}{int}\ half\ \textcolor{BrickRed}{=}\ \textcolor{BrickRed}{(}node$\_$left\ \textcolor{BrickRed}{+}\ node$\_$right\textcolor{BrickRed}{)}\ \textcolor{BrickRed}{/}\ \textcolor{Purple}{2}\textcolor{BrickRed}{;} \\
\mbox{}\ \ \textcolor{ForestGreen}{int}\ ans$\_$left\ \textcolor{BrickRed}{=}\ \textbf{\textcolor{Black}{query}}\textcolor{BrickRed}{(}\textcolor{Purple}{2}\textcolor{BrickRed}{*}node\textcolor{BrickRed}{+}\textcolor{Purple}{1}\textcolor{BrickRed}{,}\ node$\_$left\textcolor{BrickRed}{,}\ half\textcolor{BrickRed}{,}\ query$\_$left\textcolor{BrickRed}{,}\ query$\_$right\textcolor{BrickRed}{);} \\
\mbox{}\ \ \textcolor{ForestGreen}{int}\ ans$\_$right\ \textcolor{BrickRed}{=}\ \textbf{\textcolor{Black}{query}}\textcolor{BrickRed}{(}\textcolor{Purple}{2}\textcolor{BrickRed}{*}node\textcolor{BrickRed}{+}\textcolor{Purple}{2}\textcolor{BrickRed}{,}\ half\textcolor{BrickRed}{+}\textcolor{Purple}{1}\textcolor{BrickRed}{,}\ node$\_$right\textcolor{BrickRed}{,}\ query$\_$left\textcolor{BrickRed}{,}\ query$\_$right\textcolor{BrickRed}{);} \\
\mbox{} \\
\mbox{}\ \ \textbf{\textcolor{Blue}{if}}\ \textcolor{BrickRed}{(}ans$\_$left\ \textcolor{BrickRed}{==}\ \textcolor{BrickRed}{-}\textcolor{Purple}{1}\textcolor{BrickRed}{)}\ \textbf{\textcolor{Blue}{return}}\ ans$\_$right\textcolor{BrickRed}{;} \\
\mbox{}\ \ \textbf{\textcolor{Blue}{if}}\ \textcolor{BrickRed}{(}ans$\_$right\ \textcolor{BrickRed}{==}\ \textcolor{BrickRed}{-}\textcolor{Purple}{1}\textcolor{BrickRed}{)}\ \textbf{\textcolor{Blue}{return}}\ ans$\_$left\textcolor{BrickRed}{;} \\
\mbox{} \\
\mbox{}\ \ \textbf{\textcolor{Blue}{return}}\ \textcolor{BrickRed}{(}arr\textcolor{BrickRed}{[}ans$\_$left\textcolor{BrickRed}{]}\ \textcolor{BrickRed}{$<$=}\ arr\textcolor{BrickRed}{[}ans$\_$right\textcolor{BrickRed}{]}\ \textcolor{BrickRed}{?}\ ans$\_$left\ \textcolor{BrickRed}{:}\ ans$\_$right\textcolor{BrickRed}{);} \\
\mbox{}\textcolor{Red}{\}} \\
\mbox{} \\
\mbox{}\textcolor{ForestGreen}{void}\ SegmentTree\textcolor{BrickRed}{::}\textbf{\textcolor{Black}{update}}\textcolor{BrickRed}{(}\textcolor{ForestGreen}{int}\ node\textcolor{BrickRed}{,}\ \textcolor{ForestGreen}{int}\ node$\_$left\textcolor{BrickRed}{,}\ \textcolor{ForestGreen}{int}\ node$\_$right\textcolor{BrickRed}{,}\ \textcolor{ForestGreen}{int}\ where\textcolor{BrickRed}{,}\ \textcolor{ForestGreen}{int}\ what\textcolor{BrickRed}{)}\textcolor{Red}{\{} \\
\mbox{}\ \ \textbf{\textcolor{Blue}{if}}\ \textcolor{BrickRed}{(}where\ \textcolor{BrickRed}{$<$}\ node$\_$left\ \textcolor{BrickRed}{$|$$|$}\ node$\_$right\ \textcolor{BrickRed}{$<$}\ where\textcolor{BrickRed}{)}\ \textbf{\textcolor{Blue}{return}}\textcolor{BrickRed}{;} \\
\mbox{}\ \ \textbf{\textcolor{Blue}{if}}\ \textcolor{BrickRed}{(}node$\_$left\ \textcolor{BrickRed}{==}\ where\ \textcolor{BrickRed}{\&\&}\ where\ \textcolor{BrickRed}{==}\ node$\_$right\textcolor{BrickRed}{)}\textcolor{Red}{\{} \\
\mbox{}\ \ \ \ arr\textcolor{BrickRed}{[}where\textcolor{BrickRed}{]}\ \textcolor{BrickRed}{=}\ what\textcolor{BrickRed}{;} \\
\mbox{}\ \ \ \ tree\textcolor{BrickRed}{[}node\textcolor{BrickRed}{]}\ \textcolor{BrickRed}{=}\ where\textcolor{BrickRed}{;} \\
\mbox{}\ \ \ \ \textbf{\textcolor{Blue}{return}}\textcolor{BrickRed}{;} \\
\mbox{}\ \ \textcolor{Red}{\}} \\
\mbox{}\ \ \textcolor{ForestGreen}{int}\ half\ \textcolor{BrickRed}{=}\ \textcolor{BrickRed}{(}node$\_$left\ \textcolor{BrickRed}{+}\ node$\_$right\textcolor{BrickRed}{)}\ \textcolor{BrickRed}{/}\ \textcolor{Purple}{2}\textcolor{BrickRed}{;} \\
\mbox{}\ \ \textbf{\textcolor{Blue}{if}}\ \textcolor{BrickRed}{(}where\ \textcolor{BrickRed}{$<$=}\ half\textcolor{BrickRed}{)}\textcolor{Red}{\{} \\
\mbox{}\ \ \ \ \textbf{\textcolor{Black}{update}}\textcolor{BrickRed}{(}\textcolor{Purple}{2}\textcolor{BrickRed}{*}node\textcolor{BrickRed}{+}\textcolor{Purple}{1}\textcolor{BrickRed}{,}\ node$\_$left\textcolor{BrickRed}{,}\ half\textcolor{BrickRed}{,}\ where\textcolor{BrickRed}{,}\ what\textcolor{BrickRed}{);} \\
\mbox{}\ \ \textcolor{Red}{\}}\textbf{\textcolor{Blue}{else}}\textcolor{Red}{\{} \\
\mbox{}\ \ \ \ \textbf{\textcolor{Black}{update}}\textcolor{BrickRed}{(}\textcolor{Purple}{2}\textcolor{BrickRed}{*}node\textcolor{BrickRed}{+}\textcolor{Purple}{2}\textcolor{BrickRed}{,}\ half\textcolor{BrickRed}{+}\textcolor{Purple}{1}\textcolor{BrickRed}{,}\ node$\_$right\textcolor{BrickRed}{,}\ where\textcolor{BrickRed}{,}\ what\textcolor{BrickRed}{);} \\
\mbox{}\ \ \textcolor{Red}{\}} \\
\mbox{}\ \ \textbf{\textcolor{Blue}{if}}\ \textcolor{BrickRed}{(}arr\textcolor{BrickRed}{[}tree\textcolor{BrickRed}{[}\textcolor{Purple}{2}\textcolor{BrickRed}{*}node\textcolor{BrickRed}{+}\textcolor{Purple}{1}\textcolor{BrickRed}{]]}\ \textcolor{BrickRed}{$<$=}\ arr\textcolor{BrickRed}{[}tree\textcolor{BrickRed}{[}\textcolor{Purple}{2}\textcolor{BrickRed}{*}node\textcolor{BrickRed}{+}\textcolor{Purple}{2}\textcolor{BrickRed}{]])}\textcolor{Red}{\{} \\
\mbox{}\ \ \ \ tree\textcolor{BrickRed}{[}node\textcolor{BrickRed}{]}\ \textcolor{BrickRed}{=}\ tree\textcolor{BrickRed}{[}\textcolor{Purple}{2}\textcolor{BrickRed}{*}node\textcolor{BrickRed}{+}\textcolor{Purple}{1}\textcolor{BrickRed}{];} \\
\mbox{}\ \ \textcolor{Red}{\}}\textbf{\textcolor{Blue}{else}}\textcolor{Red}{\{} \\
\mbox{}\ \ \ \ tree\textcolor{BrickRed}{[}node\textcolor{BrickRed}{]}\ \textcolor{BrickRed}{=}\ tree\textcolor{BrickRed}{[}\textcolor{Purple}{2}\textcolor{BrickRed}{*}node\textcolor{BrickRed}{+}\textcolor{Purple}{2}\textcolor{BrickRed}{];} \\
\mbox{}\ \ \textcolor{Red}{\}} \\
\mbox{}\textcolor{Red}{\}} \\
\mbox{}
} \normalfont\normalsize
%.tex

%---------------------------------------------------------------
\section{Misceláneo}
\subsection{El \textit{parser} más rápido del mundo}
\begin{itemize}
\item Cada no-terminal: un método
\item Cada lado derecho: 
\begin{itemize}
\item invocar los métodos de los no-terminales o
\item Cada terminal: invocar proceso \textit{match}
\end{itemize}
\item Alternativas en una producción: se hace un \textit{if}
\end{itemize}
\medskip
No funciona con gramáticas recursivas por izquierda ó en las que en algún momento haya
varias posibles escogencias que empiezan por el mismo caracter (En ambos casos la gramática se puede factorizar).

\medskip
\textbf{Ejemplo:} Para la gramática:
$$
A \longrightarrow (A)A 
$$ $$
A \longrightarrow \epsilon
$$

% Generator: GNU source-highlight, by Lorenzo Bettini, http://www.gnu.org/software/src-highlite

{\ttfamily \raggedright {
\noindent
\mbox{}\textit{\textcolor{Brown}{//A\ -$>$\ (A)A\ $|$\ Epsilon}} \\
\mbox{} \\
\mbox{}\textbf{\textcolor{RoyalBlue}{\#include}}\ \texttt{\textcolor{Red}{$<$iostream$>$}} \\
\mbox{}\textbf{\textcolor{RoyalBlue}{\#include}}\ \texttt{\textcolor{Red}{$<$string$>$}} \\
\mbox{} \\
\mbox{}\textbf{\textcolor{Blue}{using}}\ \textbf{\textcolor{Blue}{namespace}}\ std\textcolor{BrickRed}{;} \\
\mbox{} \\
\mbox{}\textcolor{ForestGreen}{bool}\ ok\textcolor{BrickRed}{;} \\
\mbox{}\textcolor{ForestGreen}{char}\ sgte\textcolor{BrickRed}{;} \\
\mbox{}\textcolor{ForestGreen}{int}\ i\textcolor{BrickRed}{;} \\
\mbox{}string\ s\textcolor{BrickRed}{;} \\
\mbox{} \\
\mbox{}\textcolor{ForestGreen}{bool}\ \textbf{\textcolor{Black}{match}}\textcolor{BrickRed}{(}\textcolor{ForestGreen}{char}\ c\textcolor{BrickRed}{)}\textcolor{Red}{\{} \\
\mbox{}\ \ \textbf{\textcolor{Blue}{if}}\ \textcolor{BrickRed}{(}sgte\ \textcolor{BrickRed}{!=}\ c\textcolor{BrickRed}{)}\textcolor{Red}{\{} \\
\mbox{}\ \ \ \ ok\ \textcolor{BrickRed}{=}\ \textbf{\textcolor{Blue}{false}}\textcolor{BrickRed}{;} \\
\mbox{}\ \ \textcolor{Red}{\}} \\
\mbox{}\ \ sgte\ \textcolor{BrickRed}{=}\ s\textcolor{BrickRed}{[++}i\textcolor{BrickRed}{];} \\
\mbox{}\textcolor{Red}{\}} \\
\mbox{} \\
\mbox{}\textcolor{ForestGreen}{void}\ \textbf{\textcolor{Black}{A}}\textcolor{BrickRed}{()}\textcolor{Red}{\{} \\
\mbox{}\ \ \textbf{\textcolor{Blue}{if}}\ \textcolor{BrickRed}{(}sgte\ \textcolor{BrickRed}{==}\ \texttt{\textcolor{Red}{'('}}\textcolor{BrickRed}{)}\textcolor{Red}{\{} \\
\mbox{}\ \ \ \ \textbf{\textcolor{Black}{match}}\textcolor{BrickRed}{(}\texttt{\textcolor{Red}{'('}}\textcolor{BrickRed}{);} \\
\mbox{}\ \ \ \ \textbf{\textcolor{Black}{A}}\textcolor{BrickRed}{();}\ \textbf{\textcolor{Black}{match}}\textcolor{BrickRed}{(}\texttt{\textcolor{Red}{')'}}\textcolor{BrickRed}{);}\ \textbf{\textcolor{Black}{A}}\textcolor{BrickRed}{();} \\
\mbox{}\ \ \textcolor{Red}{\}}\textbf{\textcolor{Blue}{else}}\ \textbf{\textcolor{Blue}{if}}\ \textcolor{BrickRed}{(}sgte\ \textcolor{BrickRed}{==}\ \texttt{\textcolor{Red}{'\$'}}\ \textcolor{BrickRed}{$|$$|$}\ sgte\ \textcolor{BrickRed}{==}\ \texttt{\textcolor{Red}{')'}}\textcolor{BrickRed}{)}\textcolor{Red}{\{} \\
\mbox{}\ \ \ \ \textit{\textcolor{Brown}{//nada}} \\
\mbox{}\ \ \textcolor{Red}{\}}\textbf{\textcolor{Blue}{else}}\textcolor{Red}{\{} \\
\mbox{}\ \ \ \ ok\ \textcolor{BrickRed}{=}\ \textbf{\textcolor{Blue}{false}}\textcolor{BrickRed}{;} \\
\mbox{}\ \ \textcolor{Red}{\}} \\
\mbox{}\textcolor{Red}{\}} \\
\mbox{} \\
\mbox{}\textcolor{ForestGreen}{int}\ \textbf{\textcolor{Black}{main}}\textcolor{BrickRed}{()}\textcolor{Red}{\{} \\
\mbox{}\ \ \textbf{\textcolor{Blue}{while}}\textcolor{BrickRed}{(}\textbf{\textcolor{Black}{getline}}\textcolor{BrickRed}{(}cin\textcolor{BrickRed}{,}\ s\textcolor{BrickRed}{)}\ \textcolor{BrickRed}{\&\&}\ s\ \textcolor{BrickRed}{!=}\ \texttt{\textcolor{Red}{"{}"{}}}\textcolor{BrickRed}{)}\textcolor{Red}{\{} \\
\mbox{}\ \ \ \ ok\ \textcolor{BrickRed}{=}\ \textbf{\textcolor{Blue}{true}}\textcolor{BrickRed}{;} \\
\mbox{}\ \ \ \ s\ \textcolor{BrickRed}{+=}\ \texttt{\textcolor{Red}{'\$'}}\textcolor{BrickRed}{;} \\
\mbox{}\ \ \ \ sgte\ \textcolor{BrickRed}{=}\ s\textcolor{BrickRed}{[(}i\ \textcolor{BrickRed}{=}\ \textcolor{Purple}{0}\textcolor{BrickRed}{)];} \\
\mbox{}\ \ \ \ \textbf{\textcolor{Black}{A}}\textcolor{BrickRed}{();} \\
\mbox{}\ \ \ \ \textbf{\textcolor{Blue}{if}}\ \textcolor{BrickRed}{(}i\ \textcolor{BrickRed}{$<$}\ s\textcolor{BrickRed}{.}\textbf{\textcolor{Black}{length}}\textcolor{BrickRed}{()-}\textcolor{Purple}{1}\textcolor{BrickRed}{)}\ ok\ \textcolor{BrickRed}{=}\ \textbf{\textcolor{Blue}{false}}\textcolor{BrickRed}{;}\ \textit{\textcolor{Brown}{//No\ consumi\ toda\ la\ cadena}} \\
\mbox{}\ \ \ \ \textbf{\textcolor{Blue}{if}}\ \textcolor{BrickRed}{(}ok\textcolor{BrickRed}{)}\textcolor{Red}{\{} \\
\mbox{}\ \ \ \ \ \ cout\ \textcolor{BrickRed}{$<$$<$}\ \texttt{\textcolor{Red}{"{}Accepted}}\texttt{\textcolor{CarnationPink}{\textbackslash{}n}}\texttt{\textcolor{Red}{"{}}}\textcolor{BrickRed}{;} \\
\mbox{}\ \ \ \ \textcolor{Red}{\}}\textbf{\textcolor{Blue}{else}}\textcolor{Red}{\{} \\
\mbox{}\ \ \ \ \ \ cout\ \textcolor{BrickRed}{$<$$<$}\ \texttt{\textcolor{Red}{"{}Not\ accepted}}\texttt{\textcolor{CarnationPink}{\textbackslash{}n}}\texttt{\textcolor{Red}{"{}}}\textcolor{BrickRed}{;} \\
\mbox{}\ \ \ \ \textcolor{Red}{\}} \\
\mbox{}\ \ \ \  \\
\mbox{}\ \ \textcolor{Red}{\}} \\
\mbox{}\textcolor{Red}{\}} \\

} \normalfont\normalsize
%.tex


\section{Java}
\subsection{Entrada desde entrada estándar}
Este primer método es muy fácil pero es mucho más ineficiente porque utiliza Scanner en vez de BufferedReader: \\
% Generator: GNU source-highlight, by Lorenzo Bettini, http://www.gnu.org/software/src-highlite

{\ttfamily \raggedright {
\noindent
\mbox{}\textbf{\textcolor{RoyalBlue}{import}}\ java\textcolor{BrickRed}{.}io\textcolor{BrickRed}{.*;} \\
\mbox{}\textbf{\textcolor{RoyalBlue}{import}}\ java\textcolor{BrickRed}{.}util\textcolor{BrickRed}{.*;} \\
\mbox{} \\
\mbox{}\textbf{\textcolor{Blue}{class}}\ Main\textcolor{Red}{\{} \\
\mbox{}\ \ \ \ \textbf{\textcolor{Blue}{public}}\ \textbf{\textcolor{Blue}{static}}\ \textcolor{ForestGreen}{void}\ \textbf{\textcolor{Black}{main}}\textcolor{BrickRed}{(}String\textcolor{BrickRed}{[]}\ args\textcolor{BrickRed}{)}\textcolor{Red}{\{} \\
\mbox{}\ \ \ \ \ \ \ \ Scanner\ sc\ \textcolor{BrickRed}{=}\ \textbf{\textcolor{Blue}{new}}\ \textbf{\textcolor{Black}{Scanner}}\textcolor{BrickRed}{(}System\textcolor{BrickRed}{.}in\textcolor{BrickRed}{);} \\
\mbox{}\ \ \ \ \ \ \ \ \textbf{\textcolor{Blue}{while}}\ \textcolor{BrickRed}{(}sc\textcolor{BrickRed}{.}\textbf{\textcolor{Black}{hasNextLine}}\textcolor{BrickRed}{())}\textcolor{Red}{\{} \\
\mbox{}\ \ \ \ \ \ \ \ \ \ \ \ String\ s\textcolor{BrickRed}{=}\ sc\textcolor{BrickRed}{.}\textbf{\textcolor{Black}{nextLine}}\textcolor{BrickRed}{();} \\
\mbox{}\ \ \ \ \ \ \ \ \ \ \ \ System\textcolor{BrickRed}{.}out\textcolor{BrickRed}{.}\textbf{\textcolor{Black}{println}}\textcolor{BrickRed}{(}\texttt{\textcolor{Red}{"{}Leí:\ "{}}}\ \textcolor{BrickRed}{+}\ s\textcolor{BrickRed}{);} \\
\mbox{}\ \ \ \ \ \ \ \ \textcolor{Red}{\}} \\
\mbox{}\ \ \ \ \textcolor{Red}{\}} \\
\mbox{}\textcolor{Red}{\}}
} \normalfont\normalsize
%.tex

\bigskip

Este segundo es más rápido: \\
% Generator: GNU source-highlight, by Lorenzo Bettini, http://www.gnu.org/software/src-highlite

{\ttfamily \raggedright {
\noindent
\mbox{}\textbf{\textcolor{RoyalBlue}{import}}\ java\textcolor{BrickRed}{.}util\textcolor{BrickRed}{.*;} \\
\mbox{}\textbf{\textcolor{RoyalBlue}{import}}\ java\textcolor{BrickRed}{.}io\textcolor{BrickRed}{.*;} \\
\mbox{}\textbf{\textcolor{RoyalBlue}{import}}\ java\textcolor{BrickRed}{.}math\textcolor{BrickRed}{.*;} \\
\mbox{}\  \\
\mbox{}\textbf{\textcolor{Blue}{class}}\ Main\ \textcolor{Red}{\{} \\
\mbox{}\ \ \ \ \textbf{\textcolor{Blue}{public}}\ \textbf{\textcolor{Blue}{static}}\ \textcolor{ForestGreen}{void}\ \textbf{\textcolor{Black}{main}}\textcolor{BrickRed}{(}String\textcolor{BrickRed}{[]}\ args\textcolor{BrickRed}{)}\ \textbf{\textcolor{Blue}{throws}}\ IOException\ \textcolor{Red}{\{} \\
\mbox{}\ \ \ \ \ \ \ \ BufferedReader\ reader\ \textcolor{BrickRed}{=}\ \textbf{\textcolor{Blue}{new}}\ \textbf{\textcolor{Black}{BufferedReader}}\textcolor{BrickRed}{(}\textbf{\textcolor{Blue}{new}}\ \textbf{\textcolor{Black}{InputStreamReader}}\textcolor{BrickRed}{(}System\textcolor{BrickRed}{.}in\textcolor{BrickRed}{));} \\
\mbox{}\ \ \ \ \ \ \ \ String\ line\ \textcolor{BrickRed}{=}\ reader\textcolor{BrickRed}{.}\textbf{\textcolor{Black}{readLine}}\textcolor{BrickRed}{();} \\
\mbox{}\ \ \ \ \ \ \ \ StringTokenizer\ tokenizer\ \textcolor{BrickRed}{=}\ \textbf{\textcolor{Blue}{new}}\ \textbf{\textcolor{Black}{StringTokenizer}}\textcolor{BrickRed}{(}line\textcolor{BrickRed}{);} \\
\mbox{}\ \ \ \ \ \ \ \ \textcolor{ForestGreen}{int}\ N\ \textcolor{BrickRed}{=}\ Integer\textcolor{BrickRed}{.}\textbf{\textcolor{Black}{valueOf}}\textcolor{BrickRed}{(}tokenizer\textcolor{BrickRed}{.}\textbf{\textcolor{Black}{nextToken}}\textcolor{BrickRed}{());} \\
\mbox{}\ \ \ \ \ \ \ \ \textbf{\textcolor{Blue}{while}}\ \textcolor{BrickRed}{(}N\textcolor{BrickRed}{-\/-}\ \textcolor{BrickRed}{$>$}\ \textcolor{Purple}{0}\textcolor{BrickRed}{)}\textcolor{Red}{\{} \\
\mbox{}\ \ \ \ \ \ \ \ \ \ \ \ String\ a\textcolor{BrickRed}{,}\ b\textcolor{BrickRed}{;} \\
\mbox{}\ \ \ \ \ \ \ \ \ \ \ \ a\ \textcolor{BrickRed}{=}\ reader\textcolor{BrickRed}{.}\textbf{\textcolor{Black}{readLine}}\textcolor{BrickRed}{();} \\
\mbox{}\ \ \ \ \ \ \ \ \ \ \ \ b\ \textcolor{BrickRed}{=}\ reader\textcolor{BrickRed}{.}\textbf{\textcolor{Black}{readLine}}\textcolor{BrickRed}{();} \\
\mbox{}\  \\
\mbox{}\ \ \ \ \ \ \ \ \ \ \ \ \textcolor{ForestGreen}{int}\ A\ \textcolor{BrickRed}{=}\ a\textcolor{BrickRed}{.}\textbf{\textcolor{Black}{length}}\textcolor{BrickRed}{(),}\ B\ \textcolor{BrickRed}{=}\ b\textcolor{BrickRed}{.}\textbf{\textcolor{Black}{length}}\textcolor{BrickRed}{();} \\
\mbox{}\ \ \ \ \ \ \ \ \ \ \ \ \textbf{\textcolor{Blue}{if}}\ \textcolor{BrickRed}{(}B\ \textcolor{BrickRed}{$>$}\ A\textcolor{BrickRed}{)}\textcolor{Red}{\{} \\
\mbox{}\ \ \ \ \ \ \ \ \ \ \ \ \ \ \ \ System\textcolor{BrickRed}{.}out\textcolor{BrickRed}{.}\textbf{\textcolor{Black}{println}}\textcolor{BrickRed}{(}\texttt{\textcolor{Red}{"{}0"{}}}\textcolor{BrickRed}{);} \\
\mbox{}\ \ \ \ \ \ \ \ \ \ \ \ \textcolor{Red}{\}}\textbf{\textcolor{Blue}{else}}\textcolor{Red}{\{} \\
\mbox{}\ \ \ \ \ \ \ \ \ \ \ \ \ \ \ \ BigInteger\ dp\textcolor{BrickRed}{[][]}\ \textcolor{BrickRed}{=}\ \textbf{\textcolor{Blue}{new}}\ BigInteger\textcolor{BrickRed}{[}\textcolor{Purple}{2}\textcolor{BrickRed}{][}A\textcolor{BrickRed}{];} \\
\mbox{}\ \ \ \ \ \ \ \ \ \ \ \ \ \ \ \ \textit{\textcolor{Brown}{/*}} \\
\mbox{}\textit{\textcolor{Brown}{dp[i][j]\ =\ cantidad\ de\ maneras\ diferentes}} \\
\mbox{}\textit{\textcolor{Brown}{en\ que\ puedo\ distribuir\ las\ primeras\ i}} \\
\mbox{}\textit{\textcolor{Brown}{letras\ de\ la\ subsecuencia\ (b)\ terminando}} \\
\mbox{}\textit{\textcolor{Brown}{en\ la\ letra\ j\ de\ la\ secuencia\ original\ (a)}} \\
\mbox{}\textit{\textcolor{Brown}{*/}} \\
\mbox{}\  \\
\mbox{}\ \ \ \ \ \ \ \ \ \ \ \ \ \ \ \ \textbf{\textcolor{Blue}{if}}\ \textcolor{BrickRed}{(}a\textcolor{BrickRed}{.}\textbf{\textcolor{Black}{charAt}}\textcolor{BrickRed}{(}\textcolor{Purple}{0}\textcolor{BrickRed}{)}\ \textcolor{BrickRed}{==}\ b\textcolor{BrickRed}{.}\textbf{\textcolor{Black}{charAt}}\textcolor{BrickRed}{(}\textcolor{Purple}{0}\textcolor{BrickRed}{))}\textcolor{Red}{\{} \\
\mbox{}\ \ \ \ \ \ \ \ \ \ \ \ \ \ \ \ \ \ \ \ dp\textcolor{BrickRed}{[}\textcolor{Purple}{0}\textcolor{BrickRed}{][}\textcolor{Purple}{0}\textcolor{BrickRed}{]}\ \textcolor{BrickRed}{=}\ BigInteger\textcolor{BrickRed}{.}ONE\textcolor{BrickRed}{;} \\
\mbox{}\ \ \ \ \ \ \ \ \ \ \ \ \ \ \ \ \textcolor{Red}{\}}\textbf{\textcolor{Blue}{else}}\textcolor{Red}{\{} \\
\mbox{}\ \ \ \ \ \ \ \ \ \ \ \ \ \ \ \ \ \ \ \ dp\textcolor{BrickRed}{[}\textcolor{Purple}{0}\textcolor{BrickRed}{][}\textcolor{Purple}{0}\textcolor{BrickRed}{]}\ \textcolor{BrickRed}{=}\ BigInteger\textcolor{BrickRed}{.}ZERO\textcolor{BrickRed}{;} \\
\mbox{}\ \ \ \ \ \ \ \ \ \ \ \ \ \ \ \ \textcolor{Red}{\}} \\
\mbox{}\ \ \ \ \ \ \ \ \ \ \ \ \ \ \ \ \textbf{\textcolor{Blue}{for}}\ \textcolor{BrickRed}{(}\textcolor{ForestGreen}{int}\ j\textcolor{BrickRed}{=}\textcolor{Purple}{1}\textcolor{BrickRed}{;}\ j\textcolor{BrickRed}{$<$}A\textcolor{BrickRed}{;}\ \textcolor{BrickRed}{++}j\textcolor{BrickRed}{)}\textcolor{Red}{\{} \\
\mbox{}\ \ \ \ \ \ \ \ \ \ \ \ \ \ \ \ \ \ \ \ dp\textcolor{BrickRed}{[}\textcolor{Purple}{0}\textcolor{BrickRed}{][}j\textcolor{BrickRed}{]}\ \textcolor{BrickRed}{=}\ dp\textcolor{BrickRed}{[}\textcolor{Purple}{0}\textcolor{BrickRed}{][}j\textcolor{BrickRed}{-}\textcolor{Purple}{1}\textcolor{BrickRed}{];} \\
\mbox{}\ \ \ \ \ \ \ \ \ \ \ \ \ \ \ \ \ \ \ \ \textbf{\textcolor{Blue}{if}}\ \textcolor{BrickRed}{(}a\textcolor{BrickRed}{.}\textbf{\textcolor{Black}{charAt}}\textcolor{BrickRed}{(}j\textcolor{BrickRed}{)}\ \textcolor{BrickRed}{==}\ b\textcolor{BrickRed}{.}\textbf{\textcolor{Black}{charAt}}\textcolor{BrickRed}{(}\textcolor{Purple}{0}\textcolor{BrickRed}{))}\textcolor{Red}{\{} \\
\mbox{}\ \ \ \ \ \ \ \ \ \ \ \ \ \ \ \ \ \ \ \ \ \ \ \ dp\textcolor{BrickRed}{[}\textcolor{Purple}{0}\textcolor{BrickRed}{][}j\textcolor{BrickRed}{]}\ \textcolor{BrickRed}{=}\ dp\textcolor{BrickRed}{[}\textcolor{Purple}{0}\textcolor{BrickRed}{][}j\textcolor{BrickRed}{].}\textbf{\textcolor{Black}{add}}\textcolor{BrickRed}{(}BigInteger\textcolor{BrickRed}{.}ONE\textcolor{BrickRed}{);} \\
\mbox{}\ \ \ \ \ \ \ \ \ \ \ \ \ \ \ \ \ \ \ \ \textcolor{Red}{\}} \\
\mbox{}\ \ \ \ \ \ \ \ \ \ \ \ \ \ \ \ \textcolor{Red}{\}} \\
\mbox{}\  \\
\mbox{}\ \ \ \ \ \ \ \ \ \ \ \ \ \ \ \ \textbf{\textcolor{Blue}{for}}\ \textcolor{BrickRed}{(}\textcolor{ForestGreen}{int}\ i\textcolor{BrickRed}{=}\textcolor{Purple}{1}\textcolor{BrickRed}{;}\ i\textcolor{BrickRed}{$<$}B\textcolor{BrickRed}{;}\ \textcolor{BrickRed}{++}i\textcolor{BrickRed}{)}\textcolor{Red}{\{} \\
\mbox{}\ \ \ \ \ \ \ \ \ \ \ \ \ \ \ \ \ \ \ \ dp\textcolor{BrickRed}{[}i\textcolor{BrickRed}{\%}\textcolor{Purple}{2}\textcolor{BrickRed}{][}\textcolor{Purple}{0}\textcolor{BrickRed}{]}\ \textcolor{BrickRed}{=}\ BigInteger\textcolor{BrickRed}{.}ZERO\textcolor{BrickRed}{;} \\
\mbox{}\ \ \ \ \ \ \ \ \ \ \ \ \ \ \ \ \ \ \ \ \textbf{\textcolor{Blue}{for}}\ \textcolor{BrickRed}{(}\textcolor{ForestGreen}{int}\ j\textcolor{BrickRed}{=}\textcolor{Purple}{1}\textcolor{BrickRed}{;}\ j\textcolor{BrickRed}{$<$}A\textcolor{BrickRed}{;}\ \textcolor{BrickRed}{++}j\textcolor{BrickRed}{)}\textcolor{Red}{\{} \\
\mbox{}\ \ \ \ \ \ \ \ \ \ \ \ \ \ \ \ \ \ \ \ \ \ \ \ dp\textcolor{BrickRed}{[}i\textcolor{BrickRed}{\%}\textcolor{Purple}{2}\textcolor{BrickRed}{][}j\textcolor{BrickRed}{]}\ \textcolor{BrickRed}{=}\ dp\textcolor{BrickRed}{[}i\textcolor{BrickRed}{\%}\textcolor{Purple}{2}\textcolor{BrickRed}{][}j\textcolor{BrickRed}{-}\textcolor{Purple}{1}\textcolor{BrickRed}{];} \\
\mbox{}\ \ \ \ \ \ \ \ \ \ \ \ \ \ \ \ \ \ \ \ \ \ \ \ \textbf{\textcolor{Blue}{if}}\ \textcolor{BrickRed}{(}a\textcolor{BrickRed}{.}\textbf{\textcolor{Black}{charAt}}\textcolor{BrickRed}{(}j\textcolor{BrickRed}{)}\ \textcolor{BrickRed}{==}\ b\textcolor{BrickRed}{.}\textbf{\textcolor{Black}{charAt}}\textcolor{BrickRed}{(}i\textcolor{BrickRed}{))}\textcolor{Red}{\{} \\
\mbox{}\ \ \ \ \ \ \ \ \ \ \ \ \ \ \ \ \ \ \ \ \ \ \ \ \ \ \ \ dp\textcolor{BrickRed}{[}i\textcolor{BrickRed}{\%}\textcolor{Purple}{2}\textcolor{BrickRed}{][}j\textcolor{BrickRed}{]}\ \textcolor{BrickRed}{=}\ dp\textcolor{BrickRed}{[}i\textcolor{BrickRed}{\%}\textcolor{Purple}{2}\textcolor{BrickRed}{][}j\textcolor{BrickRed}{].}\textbf{\textcolor{Black}{add}}\textcolor{BrickRed}{(}dp\textcolor{BrickRed}{[(}i\textcolor{BrickRed}{+}\textcolor{Purple}{1}\textcolor{BrickRed}{)\%}\textcolor{Purple}{2}\textcolor{BrickRed}{][}j\textcolor{BrickRed}{-}\textcolor{Purple}{1}\textcolor{BrickRed}{]);} \\
\mbox{}\ \ \ \ \ \ \ \ \ \ \ \ \ \ \ \ \ \ \ \ \ \ \ \ \textcolor{Red}{\}} \\
\mbox{}\ \ \ \ \ \ \ \ \ \ \ \ \ \ \ \ \ \ \ \ \textcolor{Red}{\}} \\
\mbox{}\ \ \ \ \ \ \ \ \ \ \ \ \ \ \ \ \textcolor{Red}{\}} \\
\mbox{}\ \ \ \ \ \ \ \ \ \ \ \ \ \ \ \ System\textcolor{BrickRed}{.}out\textcolor{BrickRed}{.}\textbf{\textcolor{Black}{println}}\textcolor{BrickRed}{(}dp\textcolor{BrickRed}{[(}B\textcolor{BrickRed}{-}\textcolor{Purple}{1}\textcolor{BrickRed}{)\%}\textcolor{Purple}{2}\textcolor{BrickRed}{][}A\textcolor{BrickRed}{-}\textcolor{Purple}{1}\textcolor{BrickRed}{].}\textbf{\textcolor{Black}{toString}}\textcolor{BrickRed}{());} \\
\mbox{}\ \ \ \ \ \ \ \ \ \ \ \ \textcolor{Red}{\}} \\
\mbox{}\ \ \ \ \ \ \ \ \textcolor{Red}{\}} \\
\mbox{}\ \ \ \ \textcolor{Red}{\}} \\
\mbox{}\textcolor{Red}{\}} \\

} \normalfont\normalsize
%.tex
\subsection{Entrada desde archivo}
% Generator: GNU source-highlight, by Lorenzo Bettini, http://www.gnu.org/software/src-highlite

{\ttfamily \raggedright {
\noindent
\mbox{}\textbf{\textcolor{RoyalBlue}{import}}\ java\textcolor{BrickRed}{.}io\textcolor{BrickRed}{.*;} \\
\mbox{}\textbf{\textcolor{RoyalBlue}{import}}\ java\textcolor{BrickRed}{.}util\textcolor{BrickRed}{.*;} \\
\mbox{}\textbf{\textcolor{Blue}{public}}\ \textbf{\textcolor{Blue}{class}}\ BooleanTree\ \textcolor{Red}{\{} \\
\mbox{}\ \ \ \ \ \ \ \ \textbf{\textcolor{Blue}{public}}\ \textbf{\textcolor{Blue}{static}}\ \textcolor{ForestGreen}{void}\ \textbf{\textcolor{Black}{main}}\textcolor{BrickRed}{(}String\textcolor{BrickRed}{[]}\ args\textcolor{BrickRed}{)}\ \textbf{\textcolor{Blue}{throws}}\ FileNotFoundException\ \textcolor{Red}{\{} \\
\mbox{}\ \ \ \ \ \ \ \ \ \ \ \ \ \ \ \ System\textcolor{BrickRed}{.}\textbf{\textcolor{Black}{setIn}}\textcolor{BrickRed}{(}\textbf{\textcolor{Blue}{new}}\ \textbf{\textcolor{Black}{FileInputStream}}\textcolor{BrickRed}{(}\texttt{\textcolor{Red}{"{}tree.in"{}}}\textcolor{BrickRed}{));} \\
\mbox{}\ \ \ \ \ \ \ \ \ \ \ \ \ \ \ \ System\textcolor{BrickRed}{.}\textbf{\textcolor{Black}{setOut}}\textcolor{BrickRed}{(}\textbf{\textcolor{Blue}{new}}\ \textbf{\textcolor{Black}{PrintStream}}\textcolor{BrickRed}{(}\texttt{\textcolor{Red}{"{}tree.out"{}}}\textcolor{BrickRed}{));} \\
\mbox{}\ \ \ \ \ \ \ \ \ \ \ \ \ \ \ \ Scanner\ reader\ \textcolor{BrickRed}{=}\ \textbf{\textcolor{Blue}{new}}\ \textbf{\textcolor{Black}{Scanner}}\textcolor{BrickRed}{(}System\textcolor{BrickRed}{.}in\textcolor{BrickRed}{);} \\
\mbox{}\ \ \ \ \ \ \ \ \ \ \ \ \ \ \ \ N\ \textcolor{BrickRed}{=}\ reader\textcolor{BrickRed}{.}\textbf{\textcolor{Black}{nextInt}}\textcolor{BrickRed}{();} \\
\mbox{}\ \ \ \ \ \ \ \ \ \ \ \ \ \ \ \ \textbf{\textcolor{Blue}{for}}\ \textcolor{BrickRed}{(}\textcolor{ForestGreen}{int}\ c\ \textcolor{BrickRed}{=}\ \textcolor{Purple}{1}\textcolor{BrickRed}{;}\ c\ \textcolor{BrickRed}{$<$=}\ N\textcolor{BrickRed}{;}\ \textcolor{BrickRed}{++}c\textcolor{BrickRed}{)}\ \textcolor{Red}{\{} \\
\mbox{}\ \ \ \ \ \ \ \ \ \ \ \ \ \ \ \ \ \ \ \ \ \ \ \ \textcolor{ForestGreen}{int}\ res\ \textcolor{BrickRed}{=}\ \textcolor{Purple}{100}\textcolor{BrickRed}{;} \\
\mbox{}\ \ \ \ \ \ \ \ \ \ \ \ \ \ \ \ \ \ \ \ \ \ \ \ \textbf{\textcolor{Blue}{if}}\ \textcolor{BrickRed}{(}res\ \textcolor{BrickRed}{$<$}\ \textcolor{Purple}{1000}\textcolor{BrickRed}{)} \\
\mbox{}\ \ \ \ \ \ \ \ \ \ \ \ \ \ \ \ \ \ \ \ \ \ \ \ \ \ \ \ \ \ \ \ System\textcolor{BrickRed}{.}out\textcolor{BrickRed}{.}\textbf{\textcolor{Black}{println}}\textcolor{BrickRed}{(}\texttt{\textcolor{Red}{"{}Case\ \#"{}}}\ \textcolor{BrickRed}{+}\ c\ \textcolor{BrickRed}{+}\ \texttt{\textcolor{Red}{"{}:\ "{}}}\ \textcolor{BrickRed}{+}\ res\textcolor{BrickRed}{);} \\
\mbox{}\ \ \ \ \ \ \ \ \ \ \ \ \ \ \ \ \ \ \ \ \ \ \ \ \textbf{\textcolor{Blue}{else}} \\
\mbox{}\ \ \ \ \ \ \ \ \ \ \ \ \ \ \ \ \ \ \ \ \ \ \ \ \ \ \ \ \ \ \ \ System\textcolor{BrickRed}{.}out\textcolor{BrickRed}{.}\textbf{\textcolor{Black}{println}}\textcolor{BrickRed}{(}\texttt{\textcolor{Red}{"{}Case\ \#"{}}}\ \textcolor{BrickRed}{+}\ c\ \textcolor{BrickRed}{+}\ \texttt{\textcolor{Red}{"{}:\ IMPOSSIBLE"{}}}\textcolor{BrickRed}{);} \\
\mbox{}\ \ \ \ \ \ \ \ \ \ \ \ \ \ \ \ \textcolor{Red}{\}} \\
\mbox{}\ \ \ \ \ \ \ \ \textcolor{Red}{\}} \\
\mbox{}\textcolor{Red}{\}} \\

} \normalfont\normalsize
%.tex

\subsection{Mapas y sets}
Programa de ejemplo:
% Generator: GNU source-highlight, by Lorenzo Bettini, http://www.gnu.org/software/src-highlite

{\ttfamily \raggedright {
\noindent
\mbox{}\textbf{\textcolor{RoyalBlue}{import}}\ java\textcolor{BrickRed}{.}util\textcolor{BrickRed}{.*;} \\
\mbox{} \\
\mbox{}\textbf{\textcolor{Blue}{public}}\ \textbf{\textcolor{Blue}{class}}\ Ejemplo\ \textcolor{Red}{\{} \\
\mbox{}\ \ \ \ \textbf{\textcolor{Blue}{public}}\ \textbf{\textcolor{Blue}{static}}\ \textcolor{ForestGreen}{void}\ \textbf{\textcolor{Black}{main}}\textcolor{BrickRed}{(}String\textcolor{BrickRed}{[]}\ args\textcolor{BrickRed}{)}\textcolor{Red}{\{} \\
\mbox{}\ \ \ \ \ \ \ \ \textit{\textcolor{Brown}{/*}} \\
\mbox{}\textit{\textcolor{Brown}{\ \ \ \ \ \ \ \ \ *\ Mapas}} \\
\mbox{}\textit{\textcolor{Brown}{\ \ \ \ \ \ \ \ \ *\ Tanto\ el\ HashMap\ como\ el\ TreeMap\ funcionan,\ pero\ tienen\ diferentes\ detalles}} \\
\mbox{}\textit{\textcolor{Brown}{\ \ \ \ \ \ \ \ \ *\ y\ difieren\ en\ algunos\ métodos\ (Ver\ API).}} \\
\mbox{}\textit{\textcolor{Brown}{\ \ \ \ \ \ \ \ \ */}} \\
\mbox{}\ \ \ \ \ \ \ \ System\textcolor{BrickRed}{.}out\textcolor{BrickRed}{.}\textbf{\textcolor{Black}{println}}\textcolor{BrickRed}{(}\texttt{\textcolor{Red}{"{}Maps"{}}}\textcolor{BrickRed}{);} \\
\mbox{}\ \ \ \ \ \ \ \ \textit{\textcolor{Brown}{//TreeMap$<$String,\ Integer$>$\ m\ =\ new\ TreeMap$<$String,\ Integer$>$();}} \\
\mbox{}\ \ \ \ \ \ \ \ HashMap\textcolor{BrickRed}{$<$}String\textcolor{BrickRed}{,}\ Integer\textcolor{BrickRed}{$>$}\ m\ \textcolor{BrickRed}{=}\ \textbf{\textcolor{Blue}{new}}\ HashMap\textcolor{BrickRed}{$<$}String\textcolor{BrickRed}{,}\ Integer\textcolor{BrickRed}{$>$();} \\
\mbox{}\ \ \ \ \ \ \ \ m\textcolor{BrickRed}{.}\textbf{\textcolor{Black}{put}}\textcolor{BrickRed}{(}\texttt{\textcolor{Red}{"{}Hola"{}}}\textcolor{BrickRed}{,}\ \textbf{\textcolor{Blue}{new}}\ \textbf{\textcolor{Black}{Integer}}\textcolor{BrickRed}{(}\textcolor{Purple}{465}\textcolor{BrickRed}{));} \\
\mbox{}\ \ \ \ \ \ \ \ System\textcolor{BrickRed}{.}out\textcolor{BrickRed}{.}\textbf{\textcolor{Black}{println}}\textcolor{BrickRed}{(}\texttt{\textcolor{Red}{"{}m.size()\ =\ "{}}}\ \textcolor{BrickRed}{+}\ m\textcolor{BrickRed}{.}\textbf{\textcolor{Black}{size}}\textcolor{BrickRed}{());} \\
\mbox{}\ \ \ \ \ \ \ \ \ \ \ \ \ \ \ \  \\
\mbox{}\ \ \ \ \ \ \ \ \textbf{\textcolor{Blue}{if}}\ \textcolor{BrickRed}{(}m\textcolor{BrickRed}{.}\textbf{\textcolor{Black}{containsKey}}\textcolor{BrickRed}{(}\texttt{\textcolor{Red}{"{}Hola"{}}}\textcolor{BrickRed}{))}\textcolor{Red}{\{} \\
\mbox{}\ \ \ \ \ \ \ \ \ \ \ \ System\textcolor{BrickRed}{.}out\textcolor{BrickRed}{.}\textbf{\textcolor{Black}{println}}\textcolor{BrickRed}{(}m\textcolor{BrickRed}{.}\textbf{\textcolor{Black}{get}}\textcolor{BrickRed}{(}\texttt{\textcolor{Red}{"{}Hola"{}}}\textcolor{BrickRed}{));} \\
\mbox{}\ \ \ \ \ \ \ \ \textcolor{Red}{\}} \\
\mbox{}\ \ \ \ \ \ \ \ \ \ \ \ \ \ \ \  \\
\mbox{}\ \ \ \ \ \ \ \ System\textcolor{BrickRed}{.}out\textcolor{BrickRed}{.}\textbf{\textcolor{Black}{println}}\textcolor{BrickRed}{(}m\textcolor{BrickRed}{.}\textbf{\textcolor{Black}{get}}\textcolor{BrickRed}{(}\texttt{\textcolor{Red}{"{}Objeto\ inexistente"{}}}\textcolor{BrickRed}{));} \\
\mbox{}\ \ \ \ \ \ \ \ \ \ \ \ \ \ \ \  \\
\mbox{}\ \ \ \ \ \ \ \ \ \ \ \ \ \ \ \  \\
\mbox{}\ \ \ \ \ \ \ \ \textit{\textcolor{Brown}{/*}} \\
\mbox{}\textit{\textcolor{Brown}{\ \ \ \ \ \ \ \ \ *\ Sets}} \\
\mbox{}\textit{\textcolor{Brown}{\ \ \ \ \ \ \ \ \ *\ La\ misma\ diferencia\ entre\ TreeSet\ y\ HashSet.}} \\
\mbox{}\textit{\textcolor{Brown}{\ \ \ \ \ \ \ \ \ */}} \\
\mbox{}\ \ \ \ \ \ \ \ System\textcolor{BrickRed}{.}out\textcolor{BrickRed}{.}\textbf{\textcolor{Black}{println}}\textcolor{BrickRed}{(}\texttt{\textcolor{Red}{"{}}}\texttt{\textcolor{CarnationPink}{\textbackslash{}n}}\texttt{\textcolor{Red}{Sets"{}}}\textcolor{BrickRed}{);} \\
\mbox{}\ \ \ \ \ \ \ \ \textit{\textcolor{Brown}{/*}} \\
\mbox{}\textit{\textcolor{Brown}{\ \ \ \ \ \ \ \ \ *\ *OJO:\ El\ HashSet\ no\ está\ en\ orden,\ el\ TreeSet\ sí.}} \\
\mbox{}\textit{\textcolor{Brown}{\ \ \ \ \ \ \ \ \ */}} \\
\mbox{}\ \ \ \ \ \ \ \ \textit{\textcolor{Brown}{//HashSet$<$Integer$>$\ s\ =\ new\ HashSet$<$Integer$>$();}} \\
\mbox{}\ \ \ \ \ \ \ \ TreeSet\textcolor{BrickRed}{$<$}Integer\textcolor{BrickRed}{$>$}\ s\ \textcolor{BrickRed}{=}\ \textbf{\textcolor{Blue}{new}}\ TreeSet\textcolor{BrickRed}{$<$}Integer\textcolor{BrickRed}{$>$();} \\
\mbox{}\ \ \ \ \ \ \ \ s\textcolor{BrickRed}{.}\textbf{\textcolor{Black}{add}}\textcolor{BrickRed}{(}\textcolor{Purple}{3576}\textcolor{BrickRed}{);}\ s\textcolor{BrickRed}{.}\textbf{\textcolor{Black}{add}}\textcolor{BrickRed}{(}\textbf{\textcolor{Blue}{new}}\ \textbf{\textcolor{Black}{Integer}}\textcolor{BrickRed}{(}\texttt{\textcolor{Red}{"{}54"{}}}\textcolor{BrickRed}{));}\ s\textcolor{BrickRed}{.}\textbf{\textcolor{Black}{add}}\textcolor{BrickRed}{(}\textbf{\textcolor{Blue}{new}}\ \textbf{\textcolor{Black}{Integer}}\textcolor{BrickRed}{(}\textcolor{Purple}{1000000007}\textcolor{BrickRed}{));} \\
\mbox{}\ \ \ \ \ \ \ \ \ \ \ \ \ \ \ \ \  \\
\mbox{}\ \ \ \ \ \ \ \ \textbf{\textcolor{Blue}{if}}\ \textcolor{BrickRed}{(}s\textcolor{BrickRed}{.}\textbf{\textcolor{Black}{contains}}\textcolor{BrickRed}{(}\textcolor{Purple}{54}\textcolor{BrickRed}{))}\textcolor{Red}{\{} \\
\mbox{}\ \ \ \ \ \ \ \ \ \ \ \ System\textcolor{BrickRed}{.}out\textcolor{BrickRed}{.}\textbf{\textcolor{Black}{println}}\textcolor{BrickRed}{(}\texttt{\textcolor{Red}{"{}54\ presente."{}}}\textcolor{BrickRed}{);} \\
\mbox{}\ \ \ \ \ \ \ \ \textcolor{Red}{\}} \\
\mbox{}\ \ \ \ \ \ \ \ \ \ \ \ \ \ \ \ \  \\
\mbox{}\ \ \ \ \ \ \ \ \textbf{\textcolor{Blue}{if}}\ \textcolor{BrickRed}{(}s\textcolor{BrickRed}{.}\textbf{\textcolor{Black}{isEmpty}}\textcolor{BrickRed}{()}\ \textcolor{BrickRed}{==}\ \textbf{\textcolor{Blue}{false}}\textcolor{BrickRed}{)}\textcolor{Red}{\{} \\
\mbox{}\ \ \ \ \ \ \ \ \ \ \ \ System\textcolor{BrickRed}{.}out\textcolor{BrickRed}{.}\textbf{\textcolor{Black}{println}}\textcolor{BrickRed}{(}\texttt{\textcolor{Red}{"{}s.size()\ =\ "{}}}\ \textcolor{BrickRed}{+}\ s\textcolor{BrickRed}{.}\textbf{\textcolor{Black}{size}}\textcolor{BrickRed}{());} \\
\mbox{}\ \ \ \ \ \ \ \ \ \ \ \ Iterator\textcolor{BrickRed}{$<$}Integer\textcolor{BrickRed}{$>$}\ i\ \textcolor{BrickRed}{=}\ s\textcolor{BrickRed}{.}\textbf{\textcolor{Black}{iterator}}\textcolor{BrickRed}{();} \\
\mbox{}\ \ \ \ \ \ \ \ \ \ \ \ \textbf{\textcolor{Blue}{while}}\ \textcolor{BrickRed}{(}i\textcolor{BrickRed}{.}\textbf{\textcolor{Black}{hasNext}}\textcolor{BrickRed}{())}\textcolor{Red}{\{} \\
\mbox{}\ \ \ \ \ \ \ \ \ \ \ \ \ \ \ \ System\textcolor{BrickRed}{.}out\textcolor{BrickRed}{.}\textbf{\textcolor{Black}{println}}\textcolor{BrickRed}{(}i\textcolor{BrickRed}{.}\textbf{\textcolor{Black}{next}}\textcolor{BrickRed}{());} \\
\mbox{}\ \ \ \ \ \ \ \ \ \ \ \ \ \ \ \ i\textcolor{BrickRed}{.}\textbf{\textcolor{Black}{remove}}\textcolor{BrickRed}{();} \\
\mbox{}\ \ \ \ \ \ \ \ \ \ \ \ \textcolor{Red}{\}} \\
\mbox{}\ \ \ \ \ \ \ \ \ \ \ \ System\textcolor{BrickRed}{.}out\textcolor{BrickRed}{.}\textbf{\textcolor{Black}{println}}\textcolor{BrickRed}{(}\texttt{\textcolor{Red}{"{}s.size()\ =\ "{}}}\ \textcolor{BrickRed}{+}\ s\textcolor{BrickRed}{.}\textbf{\textcolor{Black}{size}}\textcolor{BrickRed}{());} \\
\mbox{}\ \ \ \ \ \ \ \ \textcolor{Red}{\}} \\
\mbox{}\ \ \ \ \textcolor{Red}{\}} \\
\mbox{}\textcolor{Red}{\}} \\
\mbox{} \\

} \normalfont\normalsize
%.tex
\bigskip
La salida de este programa es: \\

\ttfamily 
\fbox{\parbox{2.0in}{
Maps\\
m.size() = 1\\
465\\
null\\
\\
Sets\\
54 presente.\\
s.size() = 3\\
54\\
3576\\
1000000007\\
s.size() = 0\\
}
}
\\ \normalfont\normalsize
\bigskip 
Si quiere usarse una clase propia como llave del mapa o como elemento del set, la clase debe implementar
algunos métodos especiales: Si va a usarse un TreeMap ó TreeSet hay que implementar los métodos \texttt{compareTo} y 
\texttt{equals} de la interfaz \texttt{Comparable} como en la sección \ref{colas_de_prioridad_java}. Si va a usarse
un HashMap ó HashSet hay más complicaciones.\\
\smallskip
\textbf{Sugerencia:} Inventar una manera de codificar y decodificar la clase en una String o un Integer y meter esa representación en el mapa o set: esas clases ya tienen los métodos implementados.

\subsection{Colas de prioridad}
\label{colas_de_prioridad_java}
Hay que implementar unos métodos. Veamos un ejemplo: \\
% Generator: GNU source-highlight, by Lorenzo Bettini, http://www.gnu.org/software/src-highlite

{\ttfamily \raggedright {
\noindent
\mbox{}\textbf{\textcolor{RoyalBlue}{import}}\ java\textcolor{BrickRed}{.}util\textcolor{BrickRed}{.*;} \\
\mbox{} \\
\mbox{} \\
\mbox{}\textbf{\textcolor{Blue}{class}}\ Item\ \textbf{\textcolor{Blue}{implements}}\ Comparable\textcolor{BrickRed}{$<$}Item\textcolor{BrickRed}{$>$}\textcolor{Red}{\{} \\
\mbox{}\ \ \ \ \textcolor{ForestGreen}{int}\ destino\textcolor{BrickRed}{,}\ peso\textcolor{BrickRed}{;} \\
\mbox{} \\
\mbox{}\ \ \ \ \textbf{\textcolor{Black}{Item}}\textcolor{BrickRed}{(}\textcolor{ForestGreen}{int}\ destino\textcolor{BrickRed}{,}\ \textcolor{ForestGreen}{int}\ peso\textcolor{BrickRed}{)}\textcolor{Red}{\{} \\
\mbox{}\ \ \ \ \ \ \ \ \textbf{\textcolor{Blue}{this}}\textcolor{BrickRed}{.}peso\ \textcolor{BrickRed}{=}\ peso\textcolor{BrickRed}{;} \\
\mbox{}\ \ \ \ \ \ \ \ \textbf{\textcolor{Blue}{this}}\textcolor{BrickRed}{.}destino\ \textcolor{BrickRed}{=}\ destino\textcolor{BrickRed}{;} \\
\mbox{}\ \ \ \ \textcolor{Red}{\}} \\
\mbox{}\ \ \ \ \textit{\textcolor{Brown}{/*}} \\
\mbox{}\textit{\textcolor{Brown}{\ \ \ \ \ *\ Implementamos\ toda\ la\ javazofia.}} \\
\mbox{}\textit{\textcolor{Brown}{\ \ \ \ \ */}} \\
\mbox{}\ \ \ \ \textbf{\textcolor{Blue}{public}}\ \textcolor{ForestGreen}{int}\ \textbf{\textcolor{Black}{compareTo}}\textcolor{BrickRed}{(}Item\ otro\textcolor{BrickRed}{)}\textcolor{Red}{\{} \\
\mbox{}\ \ \ \ \ \ \ \ \textit{\textcolor{Brown}{//\ Return\ $<$\ 0\ si\ this\ $<$\ otro}} \\
\mbox{}\ \ \ \ \ \ \ \ \textit{\textcolor{Brown}{//\ Return\ \ \ 0\ si\ this\ ==\ otro}} \\
\mbox{}\ \ \ \ \ \ \ \ \textit{\textcolor{Brown}{//\ Return\ $>$\ 0\ si\ this\ $>$\ otro\ \ }} \\
\mbox{}\ \ \ \ \ \ \ \ \textbf{\textcolor{Blue}{return}}\ peso\ \textcolor{BrickRed}{-}\ otro\textcolor{BrickRed}{.}peso\textcolor{BrickRed}{;}\ \textit{\textcolor{Brown}{/*\ Un\ nodo\ es\ menor\ que\ otro\ si\ tiene\ menos\ peso\ */}} \\
\mbox{}\ \ \ \ \textcolor{Red}{\}} \\
\mbox{}\ \ \ \ \textbf{\textcolor{Blue}{public}}\ \textcolor{ForestGreen}{boolean}\ \textbf{\textcolor{Black}{equals}}\textcolor{BrickRed}{(}Object\ otro\textcolor{BrickRed}{)}\textcolor{Red}{\{} \\
\mbox{}\ \ \ \ \ \ \ \ \textbf{\textcolor{Blue}{if}}\ \textcolor{BrickRed}{(}otro\ \textbf{\textcolor{Blue}{instanceof}}\ Item\textcolor{BrickRed}{)}\textcolor{Red}{\{} \\
\mbox{}\ \ \ \ \ \ \ \ \ \ \ \ Item\ ese\ \textcolor{BrickRed}{=}\ \textcolor{BrickRed}{(}Item\textcolor{BrickRed}{)}otro\textcolor{BrickRed}{;} \\
\mbox{}\ \ \ \ \ \ \ \ \ \ \ \ \textbf{\textcolor{Blue}{return}}\ destino\ \textcolor{BrickRed}{==}\ ese\textcolor{BrickRed}{.}destino\ \textcolor{BrickRed}{\&\&}\ peso\ \textcolor{BrickRed}{==}\ ese\textcolor{BrickRed}{.}peso\textcolor{BrickRed}{;} \\
\mbox{}\ \ \ \ \ \ \ \ \textcolor{Red}{\}} \\
\mbox{}\ \ \ \ \ \ \ \ \textbf{\textcolor{Blue}{return}}\ \textbf{\textcolor{Blue}{false}}\textcolor{BrickRed}{;} \\
\mbox{}\ \ \ \ \textcolor{Red}{\}} \\
\mbox{}\ \ \ \ \textbf{\textcolor{Blue}{public}}\ String\ \textbf{\textcolor{Black}{toString}}\textcolor{BrickRed}{()}\textcolor{Red}{\{} \\
\mbox{}\ \ \ \ \ \ \ \ \textbf{\textcolor{Blue}{return}}\ \texttt{\textcolor{Red}{"{}peso\ =\ "{}}}\ \textcolor{BrickRed}{+}\ peso\ \textcolor{BrickRed}{+}\ \texttt{\textcolor{Red}{"{},\ destino\ =\ "{}}}\ \textcolor{BrickRed}{+}\ destino\textcolor{BrickRed}{;} \\
\mbox{}\ \ \ \ \textcolor{Red}{\}} \\
\mbox{}\textcolor{Red}{\}} \\
\mbox{} \\
\mbox{}\textbf{\textcolor{Blue}{class}}\ Ejemplo\ \textcolor{Red}{\{} \\
\mbox{}\ \ \ \ \textbf{\textcolor{Blue}{public}}\ \textbf{\textcolor{Blue}{static}}\ \textcolor{ForestGreen}{void}\ \textbf{\textcolor{Black}{main}}\textcolor{BrickRed}{(}String\textcolor{BrickRed}{[]}\ args\textcolor{BrickRed}{)}\ \textcolor{Red}{\{} \\
\mbox{}\ \ \ \ \ \ \ \ PriorityQueue\textcolor{BrickRed}{$<$}Item\textcolor{BrickRed}{$>$}\ q\ \textcolor{BrickRed}{=}\ \textbf{\textcolor{Blue}{new}}\ PriorityQueue\textcolor{BrickRed}{$<$}Item\textcolor{BrickRed}{$>$();} \\
\mbox{}\ \ \ \ \ \ \ \ q\textcolor{BrickRed}{.}\textbf{\textcolor{Black}{add}}\textcolor{BrickRed}{(}\textbf{\textcolor{Blue}{new}}\ \textbf{\textcolor{Black}{Item}}\textcolor{BrickRed}{(}\textcolor{Purple}{12}\textcolor{BrickRed}{,}\ \textcolor{Purple}{0}\textcolor{BrickRed}{));} \\
\mbox{}\ \ \ \ \ \ \ \ q\textcolor{BrickRed}{.}\textbf{\textcolor{Black}{add}}\textcolor{BrickRed}{(}\textbf{\textcolor{Blue}{new}}\ \textbf{\textcolor{Black}{Item}}\textcolor{BrickRed}{(}\textcolor{Purple}{4}\textcolor{BrickRed}{,}\ \textcolor{Purple}{1876}\textcolor{BrickRed}{));} \\
\mbox{}\ \ \ \ \ \ \ \ q\textcolor{BrickRed}{.}\textbf{\textcolor{Black}{add}}\textcolor{BrickRed}{(}\textbf{\textcolor{Blue}{new}}\ \textbf{\textcolor{Black}{Item}}\textcolor{BrickRed}{(}\textcolor{Purple}{13}\textcolor{BrickRed}{,}\ \textcolor{Purple}{0}\textcolor{BrickRed}{));} \\
\mbox{}\ \ \ \ \ \ \ \ q\textcolor{BrickRed}{.}\textbf{\textcolor{Black}{add}}\textcolor{BrickRed}{(}\textbf{\textcolor{Blue}{new}}\ \textbf{\textcolor{Black}{Item}}\textcolor{BrickRed}{(}\textcolor{Purple}{8}\textcolor{BrickRed}{,}\ \textcolor{Purple}{0}\textcolor{BrickRed}{));} \\
\mbox{}\ \ \ \ \ \ \ \ q\textcolor{BrickRed}{.}\textbf{\textcolor{Black}{add}}\textcolor{BrickRed}{(}\textbf{\textcolor{Blue}{new}}\ \textbf{\textcolor{Black}{Item}}\textcolor{BrickRed}{(}\textcolor{Purple}{7}\textcolor{BrickRed}{,}\ \textcolor{Purple}{3}\textcolor{BrickRed}{));} \\
\mbox{}\ \ \ \ \ \ \ \ \textbf{\textcolor{Blue}{while}}\ \textcolor{BrickRed}{(!}q\textcolor{BrickRed}{.}\textbf{\textcolor{Black}{isEmpty}}\textcolor{BrickRed}{())}\textcolor{Red}{\{} \\
\mbox{}\ \ \ \ \ \ \ \ \ \ \ \ System\textcolor{BrickRed}{.}out\textcolor{BrickRed}{.}\textbf{\textcolor{Black}{println}}\textcolor{BrickRed}{(}q\textcolor{BrickRed}{.}\textbf{\textcolor{Black}{poll}}\textcolor{BrickRed}{());} \\
\mbox{}\ \ \ \ \ \ \ \ \textcolor{Red}{\}} \\
\mbox{}\ \ \ \ \ \ \ \ \ \ \  \\
\mbox{}\ \ \ \ \textcolor{Red}{\}} \\
\mbox{}\textcolor{Red}{\}} \\

} \normalfont\normalsize
%.tex
\bigskip
La salida de este programa es: \\

\ttfamily 
\fbox{\parbox{2.0in}{
peso = 0, destino = 12\\
peso = 0, destino = 8\\
peso = 0, destino = 13\\
peso = 3, destino = 7\\
peso = 1876, destino = 4\\
}
}
\\ \normalfont\normalsize
\medskip
Vemos que la función de comparación que definimos no tiene en cuenta \texttt{destino},
por eso no desempata cuando dos \texttt{Items} tienen el mismo \texttt{peso} si no que escoge
cualquiera de manera arbitraria.

\section{C++}
\subsection{Entrada desde archivo}
% Generator: GNU source-highlight, by Lorenzo Bettini, http://www.gnu.org/software/src-highlite

{\ttfamily \raggedright {
\noindent
\mbox{}\textbf{\textcolor{RoyalBlue}{\#include}}\ \texttt{\textcolor{Red}{$<$iostream$>$}} \\
\mbox{}\textbf{\textcolor{RoyalBlue}{\#include}}\ \texttt{\textcolor{Red}{$<$fstream$>$}} \\
\mbox{} \\
\mbox{}\textbf{\textcolor{Blue}{using}}\ \textbf{\textcolor{Blue}{namespace}}\ std\textcolor{BrickRed}{;} \\
\mbox{} \\
\mbox{}\textcolor{ForestGreen}{int}\ \textbf{\textcolor{Black}{$\_$main}}\textcolor{BrickRed}{()}\textcolor{Red}{\{} \\
\mbox{}\ \ \textbf{\textcolor{Black}{freopen}}\textcolor{BrickRed}{(}\texttt{\textcolor{Red}{"{}entrada.in"{}}}\textcolor{BrickRed}{,}\ \texttt{\textcolor{Red}{"{}r"{}}}\textcolor{BrickRed}{,}\ stdin\textcolor{BrickRed}{);} \\
\mbox{}\ \ \textbf{\textcolor{Black}{freopen}}\textcolor{BrickRed}{(}\texttt{\textcolor{Red}{"{}entrada.out"{}}}\textcolor{BrickRed}{,}\ \texttt{\textcolor{Red}{"{}w"{}}}\textcolor{BrickRed}{,}\ stdout\textcolor{BrickRed}{);} \\
\mbox{} \\
\mbox{}\ \ string\ s\textcolor{BrickRed}{;} \\
\mbox{}\ \ \textbf{\textcolor{Blue}{while}}\ \textcolor{BrickRed}{(}cin\ \textcolor{BrickRed}{$>$$>$}\ s\textcolor{BrickRed}{)}\textcolor{Red}{\{} \\
\mbox{}\ \ \ \ cout\ \textcolor{BrickRed}{$<$$<$}\ \texttt{\textcolor{Red}{"{}Leí\ "{}}}\ \textcolor{BrickRed}{$<$$<$}\ s\ \textcolor{BrickRed}{$<$$<$}\ endl\textcolor{BrickRed}{;} \\
\mbox{}\ \ \textcolor{Red}{\}} \\
\mbox{}\ \ \textbf{\textcolor{Blue}{return}}\ \textcolor{Purple}{0}\textcolor{BrickRed}{;} \\
\mbox{}\textcolor{Red}{\}} \\
\mbox{} \\
\mbox{} \\
\mbox{}\textcolor{ForestGreen}{int}\ \textbf{\textcolor{Black}{main}}\textcolor{BrickRed}{()}\textcolor{Red}{\{} \\
\mbox{}\ \ ifstream\ \textbf{\textcolor{Black}{fin}}\textcolor{BrickRed}{(}\texttt{\textcolor{Red}{"{}entrada.in"{}}}\textcolor{BrickRed}{);} \\
\mbox{}\ \ ofstream\ \textbf{\textcolor{Black}{fout}}\textcolor{BrickRed}{(}\texttt{\textcolor{Red}{"{}entrada.out"{}}}\textcolor{BrickRed}{);} \\
\mbox{} \\
\mbox{}\ \ string\ s\textcolor{BrickRed}{;} \\
\mbox{}\ \ \textbf{\textcolor{Blue}{while}}\ \textcolor{BrickRed}{(}fin\ \textcolor{BrickRed}{$>$$>$}\ s\textcolor{BrickRed}{)}\textcolor{Red}{\{} \\
\mbox{}\ \ \ \ fout\ \textcolor{BrickRed}{$<$$<$}\ \texttt{\textcolor{Red}{"{}Leí\ "{}}}\ \textcolor{BrickRed}{$<$$<$}\ s\ \textcolor{BrickRed}{$<$$<$}\ endl\textcolor{BrickRed}{;} \\
\mbox{}\ \ \textcolor{Red}{\}} \\
\mbox{}\ \ \textbf{\textcolor{Blue}{return}}\ \textcolor{Purple}{0}\textcolor{BrickRed}{;} \\
\mbox{}\textcolor{Red}{\}} \\

} \normalfont\normalsize
%.tex

\subsection{Strings con caractéres especiales}
% Generator: GNU source-highlight, by Lorenzo Bettini, http://www.gnu.org/software/src-highlite

{\ttfamily \raggedright {
\noindent
\mbox{}\textbf{\textcolor{RoyalBlue}{\#include}}\ \texttt{\textcolor{Red}{$<$iostream$>$}} \\
\mbox{}\textbf{\textcolor{RoyalBlue}{\#include}}\ \texttt{\textcolor{Red}{$<$cassert$>$}} \\
\mbox{}\textbf{\textcolor{RoyalBlue}{\#include}}\ \texttt{\textcolor{Red}{$<$stdio.h$>$}} \\
\mbox{}\textbf{\textcolor{RoyalBlue}{\#include}}\ \texttt{\textcolor{Red}{$<$assert.h$>$}} \\
\mbox{}\textbf{\textcolor{RoyalBlue}{\#include}}\ \texttt{\textcolor{Red}{$<$wchar.h$>$}} \\
\mbox{}\textbf{\textcolor{RoyalBlue}{\#include}}\ \texttt{\textcolor{Red}{$<$wctype.h$>$}} \\
\mbox{}\textbf{\textcolor{RoyalBlue}{\#include}}\ \texttt{\textcolor{Red}{$<$locale.h$>$}} \\
\mbox{} \\
\mbox{}\textbf{\textcolor{Blue}{using}}\ \textbf{\textcolor{Blue}{namespace}}\ std\textcolor{BrickRed}{;} \\
\mbox{} \\
\mbox{}\textcolor{ForestGreen}{int}\ \textbf{\textcolor{Black}{main}}\textcolor{BrickRed}{()}\textcolor{Red}{\{} \\
\mbox{}\ \ \textbf{\textcolor{Black}{assert}}\textcolor{BrickRed}{(}\textbf{\textcolor{Black}{setlocale}}\textcolor{BrickRed}{(}LC$\_$ALL\textcolor{BrickRed}{,}\ \texttt{\textcolor{Red}{"{}en$\_$US.UTF-8"{}}}\textcolor{BrickRed}{)}\ \textcolor{BrickRed}{!=}\ NULL\textcolor{BrickRed}{);} \\
\mbox{}\ \ \textcolor{ForestGreen}{wchar$\_$t}\ c\textcolor{BrickRed}{;} \\
\mbox{} \\
\mbox{}\ \ wstring\ s\textcolor{BrickRed}{;} \\
\mbox{}\ \ \textbf{\textcolor{Blue}{while}}\ \textcolor{BrickRed}{(}\textbf{\textcolor{Black}{getline}}\textcolor{BrickRed}{(}wcin\textcolor{BrickRed}{,}\ s\textcolor{BrickRed}{))}\textcolor{Red}{\{} \\
\mbox{}\ \ \ \ wcout\ \textcolor{BrickRed}{$<$$<$}\ L\texttt{\textcolor{Red}{"{}Leí\ :\ "{}}}\ \textcolor{BrickRed}{$<$$<$}\ s\ \textcolor{BrickRed}{$<$$<$}\ endl\textcolor{BrickRed}{;} \\
\mbox{}\ \ \ \ \textbf{\textcolor{Blue}{for}}\ \textcolor{BrickRed}{(}\textcolor{ForestGreen}{int}\ i\textcolor{BrickRed}{=}\textcolor{Purple}{0}\textcolor{BrickRed}{;}\ i\textcolor{BrickRed}{$<$}s\textcolor{BrickRed}{.}\textbf{\textcolor{Black}{size}}\textcolor{BrickRed}{();}\ \textcolor{BrickRed}{++}i\textcolor{BrickRed}{)}\textcolor{Red}{\{} \\
\mbox{}\ \ \ \ \ \ c\ \textcolor{BrickRed}{=}\ s\textcolor{BrickRed}{[}i\textcolor{BrickRed}{];} \\
\mbox{}\ \ \ \ \ \ \textbf{\textcolor{Black}{wprintf}}\textcolor{BrickRed}{(}L\texttt{\textcolor{Red}{"{}\%lc\ \%lc}}\texttt{\textcolor{CarnationPink}{\textbackslash{}n}}\texttt{\textcolor{Red}{"{}}}\textcolor{BrickRed}{,}\ \textbf{\textcolor{Black}{towlower}}\textcolor{BrickRed}{(}s\textcolor{BrickRed}{[}i\textcolor{BrickRed}{]),}\ \textbf{\textcolor{Black}{towupper}}\textcolor{BrickRed}{(}s\textcolor{BrickRed}{[}i\textcolor{BrickRed}{]));} \\
\mbox{}\ \ \ \ \textcolor{Red}{\}} \\
\mbox{}\ \ \textcolor{Red}{\}} \\
\mbox{} \\
\mbox{}\ \ \textbf{\textcolor{Blue}{return}}\ \textcolor{Purple}{0}\textcolor{BrickRed}{;} \\
\mbox{}\textcolor{Red}{\}} \\
\mbox{} \\
\mbox{} \\

} \normalfont\normalsize
%.tex

\emph{Nota}: Como alternativa a la función getline, se pueden utilizar las funciones fgetws y fputws, y más adelante swscanf y wprintf:
% Generator: GNU source-highlight, by Lorenzo Bettini, http://www.gnu.org/software/src-highlite

{\ttfamily \raggedright {
\noindent
\mbox{}\textbf{\textcolor{RoyalBlue}{\#include}}\ \texttt{\textcolor{Red}{$<$iostream$>$}} \\
\mbox{}\textbf{\textcolor{RoyalBlue}{\#include}}\ \texttt{\textcolor{Red}{$<$cassert$>$}} \\
\mbox{}\textbf{\textcolor{RoyalBlue}{\#include}}\ \texttt{\textcolor{Red}{$<$stdio.h$>$}} \\
\mbox{}\textbf{\textcolor{RoyalBlue}{\#include}}\ \texttt{\textcolor{Red}{$<$assert.h$>$}} \\
\mbox{}\textbf{\textcolor{RoyalBlue}{\#include}}\ \texttt{\textcolor{Red}{$<$wchar.h$>$}} \\
\mbox{}\textbf{\textcolor{RoyalBlue}{\#include}}\ \texttt{\textcolor{Red}{$<$wctype.h$>$}} \\
\mbox{}\textbf{\textcolor{RoyalBlue}{\#include}}\ \texttt{\textcolor{Red}{$<$locale.h$>$}} \\
\mbox{} \\
\mbox{} \\
\mbox{}\textbf{\textcolor{Blue}{using}}\ \textbf{\textcolor{Blue}{namespace}}\ std\textcolor{BrickRed}{;} \\
\mbox{} \\
\mbox{}\textcolor{ForestGreen}{int}\ \textbf{\textcolor{Black}{main}}\textcolor{BrickRed}{()}\textcolor{Red}{\{} \\
\mbox{}\ \ \textbf{\textcolor{Black}{assert}}\textcolor{BrickRed}{(}\textbf{\textcolor{Black}{setlocale}}\textcolor{BrickRed}{(}LC$\_$ALL\textcolor{BrickRed}{,}\ \texttt{\textcolor{Red}{"{}en$\_$US.UTF-8"{}}}\textcolor{BrickRed}{)}\ \textcolor{BrickRed}{!=}\ NULL\textcolor{BrickRed}{);} \\
\mbox{}\ \ \textcolor{ForestGreen}{wchar$\_$t}\ in$\_$buf\textcolor{BrickRed}{[}\textcolor{Purple}{512}\textcolor{BrickRed}{],}\ out$\_$buf\textcolor{BrickRed}{[}\textcolor{Purple}{512}\textcolor{BrickRed}{];} \\
\mbox{}\ \ \textbf{\textcolor{Black}{swprintf}}\textcolor{BrickRed}{(}out$\_$buf\textcolor{BrickRed}{,}\ \textcolor{Purple}{512}\textcolor{BrickRed}{,}\ L\texttt{\textcolor{Red}{"{}¿Podrías\ escribir\ un\ número?,\ Por\ ejemplo\ \%d.\ ¡Gracias,\ pingüino\ español!}}\texttt{\textcolor{CarnationPink}{\textbackslash{}n}}\texttt{\textcolor{Red}{"{}}}\textcolor{BrickRed}{,}\ \textcolor{Purple}{3}\textcolor{BrickRed}{);} \\
\mbox{}\ \ \textbf{\textcolor{Black}{fputws}}\textcolor{BrickRed}{(}out$\_$buf\textcolor{BrickRed}{,}\ stdout\textcolor{BrickRed}{);} \\
\mbox{} \\
\mbox{}\ \ \textbf{\textcolor{Black}{fgetws}}\textcolor{BrickRed}{(}in$\_$buf\textcolor{BrickRed}{,}\ \textcolor{Purple}{512}\textcolor{BrickRed}{,}\ stdin\textcolor{BrickRed}{);} \\
\mbox{}\ \ \textcolor{ForestGreen}{int}\ n\textcolor{BrickRed}{;} \\
\mbox{}\ \ \textbf{\textcolor{Black}{swscanf}}\textcolor{BrickRed}{(}in$\_$buf\textcolor{BrickRed}{,}\ L\texttt{\textcolor{Red}{"{}\%d"{}}}\textcolor{BrickRed}{,}\ \textcolor{BrickRed}{\&}n\textcolor{BrickRed}{);} \\
\mbox{} \\
\mbox{}\ \ \textbf{\textcolor{Black}{swprintf}}\textcolor{BrickRed}{(}out$\_$buf\textcolor{BrickRed}{,}\ \textcolor{Purple}{512}\textcolor{BrickRed}{,}\ L\texttt{\textcolor{Red}{"{}Escribiste\ \%d,\ yo\ escribo\ ¿ÔÏàÚÑ\textasciitilde{}}}\texttt{\textcolor{CarnationPink}{\textbackslash{}n}}\texttt{\textcolor{Red}{"{}}}\textcolor{BrickRed}{,}\ n\textcolor{BrickRed}{);} \\
\mbox{}\ \ \textbf{\textcolor{Black}{fputws}}\textcolor{BrickRed}{(}out$\_$buf\textcolor{BrickRed}{,}\ stdout\textcolor{BrickRed}{);} \\
\mbox{} \\
\mbox{}\ \ \textbf{\textcolor{Blue}{return}}\ \textcolor{Purple}{0}\textcolor{BrickRed}{;} \\
\mbox{}\textcolor{Red}{\}} \\
\mbox{} \\
\mbox{} \\

} \normalfont\normalsize
%.tex

\end{document}